\chapter*{}


----------------------------------------------------

\textit{When we consider and reflect upon nature at large or the history of mankind or our own intellectual activity, at first we see the picture of an endless entanglement of relations and reactions in which nothing remains what, where and as it was, but everything moves, changes, comes into being and passes away. This primitive, naive but intrinsically correct conception of the world is that of ancient Greek philosophy, and was first clearly formulated by Heraclitus: everything is and is not, for everything is fluid, is constantly changing, constantly coming into being and passing away.}

-------------------------

\textit{Dialectics constitutes the most important form of thinking for present-day natural science, for it alone offers the analogue for, and thereby the method of explaining, the evolutionary processes occurring in nature, inter-connections in general, and transitions from one field of investigation to another.}

------------------------------

\textit{Nature is the proof of dialectics, and it must be said for modern science that it has furnished this proof with very rich materials increasingly daily, and thus has shown that, in the last resort, Nature works dialectically and not metaphysically; that she does not move in the eternal oneness of a perpetually recurring circle, but goes through a real historical evolution.}

--------------------------------

\textit{It is precisely the alteration of nature by men, not solely nature as such, which is the most essential and immediate basis of human thought.}

---------------------------------

thesis on feuerbarch i, ii

---------------------------------

\textit{The analysis of Nature into its individual parts, the grouping of the different natural processes and natural objects in definite classes, the study of the internal anatomy of organic bodies in their manifold forms—these were the fundamental conditions of the gigantic strides in our knowledge of Nature which have been made during the last four hundred years. But this method of investigation has also left us as a legacy the habit of observing natural objects and natural processes in their isolation, detached from the whole vast interconnection of things; and therefore not in their motion, but in their repose; not as essentially changing, but fixed constants; not in their life, but in their death. }


