\chapter{QCD on the lattice}\addcontentsline{toc}{chapter}{QCD on the lattice}

%%%%%%%%%%%%%%%%%%%%%%%%%%%%%%%%%%%%%%%%%%%%%%%%%%%%%%%%%%%
%%%%%%%%%%%%%%%%%%%%%%%%%%%%%%%%%%%%%%%%%%%%%%%%%%%%%%%%%%%
%%%%%%%%%%%%%%%%%%%%%%%%%%%%%%%%%%%%%%%%%%%%%%%%%%%%%%%%%%%
%%%%%%%%%%%%%%%%%%%%%%%%%%%%%%%%%%%%%%%%%%%%%%%%%%%%%%%%%%%

\label{ch_foundation}

%%%%%%%%%%%%%%%%%%%%%%%%%%%%%%%%%%%%%%%%%%%%%%%%%%%%%%%%%%%
%%%%%%%%%%%%%%%%%%%%%%%%%%%%%%%%%%%%%%%%%%%%%%%%%%%%%%%%%%%
%%%%%%%%%%%%%%%%%%%%%%%%%%%%%%%%%%%%%%%%%%%%%%%%%%%%%%%%%%%
%%%%%%%%%%%%%%%%%%%%%%%%%%%%%%%%%%%%%%%%%%%%%%%%%%%%%%%%%%%

\section{Motivation}
\label{ch_foundation:sec:general}

The theory that describes the strong interactions between quarks and gluons is called Quantum Chromodynamics, or QCD. 

QCD has many remarkable properties. Among others, it is an asymptotically free theory, which means that at high energies (or short distances) the fundamental particles interact very weakly and can be treated as free. On the other hand, this implies the phenomenon of confinement at low energies (or large distances), phenomenon by which no color charged particles can be observed in Nature at these energy scales, but only color singlets. These are composite particles called mesons and baryons, which are made of quarks and gluons. Furthermore, at low energies the coupling constant grows logarithmically and perturbation theory fails to make reliable predictions. Another important feature of QCD is spontaneous chiral symmetry breaking, which is responsible for the light mass of pions, among other phenomena.

QCD is a non-abelian theory associated to the SU($N_c=3$) Lie group. SU(3) rotations are described by unitary $3\times 3$ matrices $\Omega(x)$ with ${\textrm{det}}\left(\Omega(x)\right)=1$. This is a non-abelian group since its elements do not commute under multiplication. The elements of the group can be written as
\begin{equation}
\Omega(x)=e^{i\alpha^{(a)}(x)T^{(a)}},
\end{equation}
where summation over $a=1,...,N_c^2-1=8$ is implicit and $T^{(a)}$ are the 8 generators of the SU(3) Lie group. Quarks are described by elements of the group in the fundamental representation, and gluons in the adjoint, both of them carrying the QCD charge: color.

Quarks are described by the fermionic fields $\psi,\;\bar{\psi}$. They carry a Dirac spinor index $\alpha=1,2,3,4$, a flavor index $i=1,...,N_f$ and a color index $c=1,2,3$,
\begin{equation}
\psi^i(x)_{\alpha c}.
\end{equation} 
In Nature, $N_f=6$ and each flavor of quark has a different mass $m_i$. Under SU(3) quarks transform as
\begin{gather}
\psi(x)\rightarrow\Omega(x)\psi(x), \quad
\bar{\psi}(x)\to\bar{\psi}(x)\Omega^{\dagger}(x).
\end{gather}

Gluons are described by the gauge fields $A_{\mu}(x)$, which are $3\times 3$ hermitian and traceless matrices belonging to the Lie algebra su(3). Gluons, being gauge bosons, carry a Lorentz index $\mu=0,1,2,3$ and two color indices $c,d=1,2,3$,
\begin{equation}
A_{\mu}(x)_{cd}.
\end{equation}
Since $A_{\mu}(x)$ are in the Lie algebra su(3) we can decompose it into
\begin{equation}
A_{\mu}(x)=\sum_{a=1}^8A_{\mu}^{(a)}(x)T^{(a)},
\end{equation}
$A_{\mu}^{(a)}(x)$ being real-valued fields.

Omitting color and spinor indices, the continuum QCD action takes the form
\begin{align}
\label{ch_foundation:eq:QCD}
S_{\textrm{QCD}}&=\sum_{i=1}^{N_f}\int d^4x\;\bar{\psi}^i(x)\left(\gamma_{\mu}D_{\mu}(x)+m_i\right)\psi^i(x)\\&+\frac{1}{2g_0^2}{\textrm{tr}}\left(F_{\mu\nu}(x)F_{\mu\nu}(x)\right),
\end{align}
where the covariant derivative is given by
\begin{equation}
D_{\mu}(x)=\partial_{\mu}+iA_{\mu}(x)=\partial_{\mu}+iA_{\mu}^{(a)}T^{(a)}(x),
\end{equation}
and the strength tensor
\begin{align}
F_{\mu\nu}(x)&=-i\left[D_{\mu}(x),D_{\nu}(x)\right]\\&=\partial_{\mu}A_{\nu}(x)-\partial_{\nu}A_{\mu}(x)+i\left[A_{\mu}(x),A_{\nu}(x)\right] \\
&=\left(\partial_{\mu}A_{\nu}^{(a)}(x)-\partial_{\nu}A_{\mu}^{(a)}(x)\right)T^{(a)}-f_{abc}A_{\mu}^{(b)}(x)A_{\nu}^{(c)}(x)T^{(a)} \\
&\equiv F_{\mu\nu}^{(a)}T^{(a)},
\end{align}
with $f_{abc}$ the structure constants of the SU(3) group, which are anti-symmetric and given by
\begin{equation}
\left[T^{(b)},T^{(c)}\right]=if_{bca}T^{(a)}.
\end{equation}
This allows to write the gauge action as
\begin{equation}
\frac{1}{4g_0^2}\int d^4x\;F_{\mu\nu}^{(a)}(x)F_{\mu\nu}^{(a)}(x).
\end{equation}
Finally, we have used the euclidean Dirac matrices $\gamma_{\mu}$, which fulfill 
\begin{equation}
\{\gamma_{\mu},\gamma_{\nu}\}=2\delta_{\mu\nu}.
\end{equation}

In order for the QCD action to be invariant under SU(3) rotations, the gauge fields must transform as
\begin{equation}
A_{\mu}(x)\to\Omega(x)A_{\mu}(x)\Omega^{\dagger}(x)+i\left(\partial_{\mu}\Omega(x)\right)\Omega^{\dagger}(x),
\end{equation}
and so they are in the adjoint representation. From this we obtain the transformation relations
\begin{align}
D_{\mu}(x)&\to\Omega(x)D_{\mu}(x)\Omega^{\dagger}(x), \\
F_{\mu\nu}(x)&\to\Omega(x)F_{\mu\nu}(x)\Omega^{\dagger}(x).
\end{align}

As mentioned above, at low energies or large distances QCD cannot be studied using perturbation theory. The only other known first-principles method to perform theoretical predictions is Lattice Field Theory, or Lattice QCD when applied to the study of Quantum Chromodynamics. This method is based on the discretization of spacetime into a hypercubic box or lattice
\begin{equation}
\Lambda=\{n_0,n_1,n_2,n_3|n_0=0,...,T/a-1;n_i=0,...,L/a-1;i=1,2,3\},
\end{equation} 
where $a$ is the lattice spacing between two adjacent sites, and $L,T$ are the spatial and temporal lattice extents (in physical units) respectively. The discretization of spacetime and introduction of a finite lattice spacing $a$ provides a natural energy cutoff for momenta $\sim a^{-1}$, removing UV divergences. On the other hand, the finite volume lattice ensures the absence of IR divergences. This means that the lattice formulation can be seen as a way to regularize any particular Quantum Field Theory. However, this also implies the presence of finite volume and discretization effects, which should be removed from the results. To do this, after having computed the physical observables in the lattice setup, one must perform a continuum extrapolation to obtain results at $a\rightarrow0$ and simulate large enough volumes in order to be able to neglect the effects associated with finite volume. If the theory is renormalizable, physical quantities will remain finite in the continuum limit.

Having discretized spacetime, fields are placed at the lattice sites $n\in\Lambda$. Fermions thus look like
\begin{gather}
\psi(n),\bar{\psi}(n), \quad
n\in\Lambda.
\end{gather}
For the gauge fields, it will be helpful to use the definition of a parallel transporter in SU($N$). An $N$-component unit vector $\boldsymbol{v}$ is parallel transported along a curve parameterized by $z_{\mu}(t)$ from point $z_{\mu}(a)=x$ to $z_{\mu}(b)=y$ as
\begin{align}
\boldsymbol{v}(b)&=P(y,x)\boldsymbol{v}(a),\\
P(y,x)&=e^{i\int_x^yA_{\mu}(z)dz_{\mu}},
\end{align}
with $A_{\mu}$ the SU($N$) gauge fields. This means that a fermion in the fundamental representation picks up a phase of $P(y,x)$ when going from $x$ to $y$. This parallel transporter is called a gauge link and its discrete version will be the variable to be used in the lattice for the gauge fields. It transforms in the fundamental representation of SU($N$),
\begin{equation}
P(x,y)\to\Omega(x)P(x,y)\Omega^{\dagger}(y).
\end{equation}

Having defined the fields in the lattice, one needs to discretize the QCD action, formulating it in a finite box $\Lambda$ in terms of said fields in such a way that in the continuum limit $a\rightarrow 0$ the continuum QCD action is recovered. We discuss this in the following sections.	

The Chapter is structured as follows. In Sec.~\ref{ch_foundation:sec:Gauge} we show the Wilson formulation of the gauge action in the lattice, in terms of the link variables. In Sec.~\ref{ch_foundation:sec:Fermions} we show different ways of discretizing the fermion action. In Sec.~\ref{ch_foundation:subsec:Naive} we discuss the problem of doublers that appear with a naive fermion discretization and its relation with formulating chiral symmetry in the lattice, briefly commenting on Ginsparg-Wilson fermions. In Sec.~\ref{ch_foundation:subsec:Wilson} we show the solution to the doublers problem proposed by Wilson, which implies adding a term which explicitly breaks chiral symmetry. This term gives a heavy mass to the doublers that grows with the inverse of the lattice spacing $a$, and thus they decouple in the continuum. In Sec.~\ref{ch_foundation:subsec:tm} we discuss a modification of Wilson fermions which adds a chirally rotated mass term. This regularization poses some advantages which will be of importance for our study. In Sec.~\ref{ch_foundation:sec:path} we review some concepts of the path integral formalism and how expectation values are computed numerically in the lattice. In Sec.~\ref{ch_foundation:sec:continuum-limit} we review some concepts of renormalizability and the continuum limit in the lattice. In Sec.~\ref{ch_foundation:sec:impr} we discuss the Symanzik improvement program, which allows to reduce cutoff effects associated to the lattice action and fields, helping in the task of performing the continuum limit. Finally, in Sec.~\ref{ch_foundation:sec:ss} the procedure to set the scale in the lattice is discussed. This is necessary in order to extract lattice predictions in physical units.

%%%%%%%%%%%%%%%%%%%%%%%%%%%%%%%%%%%%%%%%%%%%%%%%%%%%%%%%%%%
%%%%%%%%%%%%%%%%%%%%%%%%%%%%%%%%%%%%%%%%%%%%%%%%%%%%%%%%%%%
%%%%%%%%%%%%%%%%%%%%%%%%%%%%%%%%%%%%%%%%%%%%%%%%%%%%%%%%%%%
%%%%%%%%%%%%%%%%%%%%%%%%%%%%%%%%%%%%%%%%%%%%%%%%%%%%%%%%%%%

\section{Pure gauge SU(3) in the lattice}
\label{ch_foundation:sec:Gauge}

In the lattice, gluon fields can be defined by the link variables $U_{\mu}(x)\in$ SU(3) that act as a discrete version of the gauge transporters connecting points $x$ and $x+\hat{\mu}$, with $\hat{\mu}=\{a\hat{x}_0,a\hat{x}_1,a\hat{x}_2,a\hat{x}_3\}$
\begin{equation}
\label{ch_foundation:eq:U}
U_{\mu}(x)=\exp\left(iaA_{\mu}(x)\right).
\end{equation}
The gauge fields transform as 
\begin{equation}
\label{ch_foundation:eq:U_transf}
U_{\mu}(x)\to\Omega(x)U_{\mu}(x)\Omega^{\dagger}(x+\hat{\mu}),
\end{equation}
These fields live on the links of the lattice that connect sites $x$ and $x+\hat{\mu}$.

A common discretization of the gluonic action is the Wilson gauge action~\cite{Wilson:1974sk}, which is expressed in terms of the link variables $U_{\mu}(x)$
\begin{equation}
S_{\textrm{G}}=\frac{1}{g_0^2}\sum_x\sum_{\mu,\nu}{\textrm{Re\; tr}}\left(1-U_{\mu\nu}(x)\right),
\end{equation} 
where $U_{\mu\nu}(x)$ is the plaquette centered on the lattice site $x$
\begin{equation}
\label{ch_foundation:eq:plaq}
U_{\mu\nu}(x)=U_{\mu}(x)U_{\nu}(x+\hat{\mu})U_{\mu}^{\dagger}(x+\hat{\nu})U_{\nu}^{\dagger}(x),
\end{equation}
and we have used 
\begin{equation}
U_{\mu}^{\dagger}(x)=U_{-\mu}(x+\hat{\mu}).
\end{equation}
Using the Baker-Campbell-Hausdorff formula iteratively
\begin{equation}
\exp\left(A\right)\exp\left(B\right)=\exp\left(A+B+\frac{1}{2}\left[A,B\right]+...\right),
\end{equation}
and using eq.~(\ref{ch_foundation:eq:U}) we get
\begin{equation}
\label{ch_foundation:eq:YM-latt}
S_{\textrm{G}}=a^4\frac{\beta}{3}\sum_x\sum_{\mu,\nu}{\textrm{tr}}\left(F_{\mu\nu}^2(x)\right)+\mathcal{O}(a^2),
\end{equation}
where we introduced the inverse coupling
\begin{equation}
\beta=\frac{6}{g_0^2}.
\end{equation}
Taking the continuum limit $a^4\sum_x\rightarrow\int d^4x$ we recover the continuum Yang-Mills action.

Eq.~(\ref{ch_foundation:eq:YM-latt}) shows that the effects related to the discretization of spacetime are of order $\mathcal{O}(a^2)$ for the Wilson gauge action. The discretization of the SU(3) pure Yang-Mills action is not unique, and different choices result in different cutoff effects.

The $\mathcal{O}(a^2)$ cutoff effects present in the Wilson regularization of the gauge action can be further reduced by adding additional terms that respect the symmetries of the theory following the Symanzik improvement program. One such choice is the Lüscher-Weisz action~\cite{}, which we discuss in Sec.~\ref{ch_foundation:sec:impr}.

%%%%%%%%%%%%%%%%%%%%%%%%%%%%%%%%%%%%%%%%%%%%%%%%%%%%%%%%%%%
%%%%%%%%%%%%%%%%%%%%%%%%%%%%%%%%%%%%%%%%%%%%%%%%%%%%%%%%%%%


\section{Introducing fermions in the lattice}
\label{ch_foundation:sec:Fermions}

%%%%%%%%%%%%%%%%%%%%%%%%%%%%%%%%%%%%%%%%%%%%%%%%%%%%%%%%%%%
%%%%%%%%%%%%%%%%%%%%%%%%%%%%%%%%%%%%%%%%%%%%%%%%%%%%%%%%%%%

After discretizing the SU(3) gauge action, we still need to find a suitable discrete version of the fermion action in eq.~(\ref{ch_foundation:eq:QCD}) to fully formulate QCD on the lattice. It will be shown that theoretical challenges arise with the naive fermion discretization and how these can be addressed with alternative formulations.

\subsection{Naive fermions}
\label{ch_foundation:subsec:Naive}

To discretize the continuum fermionic action in the absence of gauge fields, considering only one flavor with mass $m$,
\begin{equation}
S_{\textrm{F}}=\int d^4x\bar{\psi}(x)\left(\gamma_{\mu}\partial_{\mu}+m\right)\psi(x),
\end{equation}
the derivative $\partial_{\mu}$ needs to take a discrete form, which can be done easily by
\begin{equation}
\partial_{\mu}\psi(x)\rightarrow\hat{\partial}_{\mu}\psi(x)=\frac{1}{2a}\left(\psi(x+\hat{\mu})-\psi(x-\hat{\mu})\right).
\end{equation}
To respect gauge symmetry in our action, we must promote the derivative $\hat{\partial}_{\mu}$ to a covariant one, as in the continuum case. To achieve this, we note that terms like
\begin{equation}
\bar{\psi}(x)\psi(x+\hat{\mu}),
\end{equation}
which arise from $\bar{\psi}(x)\hat{\partial}_{\mu}\psi(x)$ are not gauge invariant
\begin{equation}
\bar{\psi}(x)\psi(x+\hat{\mu})\to\bar{\psi}(x)\Omega^{\dagger}(x)\Omega(x+\hat{\mu})\psi(x+\hat{\mu}).
\end{equation}
The solution is again to introduce the link variable or parallel transporter $U_{\mu}(x)$ from site $x$ to $x+\hat{\mu}$ defined in eq.~(\ref{ch_foundation:eq:U}) which transforms as in eq.~(\ref{ch_foundation:eq:U_transf}). This way, the discretized fermion action reads
\begin{equation}
\label{ch_foundation:eq:naive}
S_{\textrm{F}}=a^4\sum_x\bar{\psi}(x)\left(\gamma_{\mu}\frac{U_{\mu}(x)\psi(x+\hat{\mu})-U_{\mu}^{\dagger}(x-\hat{\mu})\psi(x-\hat{\mu})}{2a}+m\psi(x)\right).
\end{equation}
From this the Dirac operator $D$ can be defined such that
\begin{equation}
S_{\textrm{F}}=\int d^4x\bar{\psi}(x)D\psi(x).
\end{equation}

The lattice fermion action used here suffers from the problem of doublers, which are unwanted poles that appear in the Kallen-Lehman representation of the fermion propagator $\left<\psi(x)\bar{\psi}(y)\right>$. These doublers appear due to the presence of two zeros of the Dirac operator in the First Brillouin Zone (FBZ) in each lattice direction, all with the same vacuum energy.  Thus, for $d$ dimensions, there are $2^d$ one particle states with the same ground state energy. The problem of the doublers is related to chiral symmetry and its implementation in the lattice.

Chiral symmetry in continuum QCD can be expressed as 
\begin{equation}
\label{ch_foundation:eq:chirality}
\{D,\gamma_5\}=0,
\end{equation}
with $D$ the Dirac operator. There's a theorem by Nielsen and Ninomiya~\cite{} that states that one cannot implement chiral symmetry in this way in the lattice without the appearance of doublers.
In this lattice formulation of chiral symmetry, there must be an equal number of right movers and left movers. In particular, this means having just one pole is not possible. Ginsparg and Wilson~\cite{} proposed a suitable version of chiral symmetry for the lattice as
\begin{equation}
\label{ch_foundation:eq:chirality-latt}
\{D,\gamma_5\}=aD\gamma_5D,
\end{equation}
such that in the continuum eq.~(\ref{ch_foundation:eq:chirality}) is recovered. With this definition of chiral symmetry in the lattice, it is possible to construct Dirac operators that satisfy eq.~(\ref{ch_foundation:eq:chirality-latt}) and are free of doublers. An example of such an operator is the overlap operator~\cite{}.

If one is not interested in studying physics related to chiral symmetry, another choice is to build a Dirac operator that explicitly breaks chiral symmetry but removes the doublers. Wilson fermions and Wilson twisted mass fermions are examples of such a choice, which we will now study.

%%%%%%%%%%%%%%%%%%%%%%%%%%%%%%%%%%%%%%%%%%%%%%%%%%%%%%%%%%%
%%%%%%%%%%%%%%%%%%%%%%%%%%%%%%%%%%%%%%%%%%%%%%%%%%%%%%%%%%%

\subsection{Wilson fermions}
\label{ch_foundation:subsec:Wilson}

Wilson proposed~\cite{Wilson:1974sk} adding an extra term to the naive fermionic action in eq.~\eqref{ch_foundation:eq:naive} to eliminate doublers. This term vanishes in the continuum limit, but at finite lattice spacing it gives an additional mass to the doublers. This mass becomes infinitely heavy in the continuum limit, and thus the doublers decouple from the theory. The Wilson fermion action reads
\begin{equation}
\label{ch_foundation:eq:Wil_fer}
S_{\textrm{W}}=a^4\sum_x\bar{\psi}(x)\frac{1}{2}\left(\gamma_{\mu}\left(\nabla_{\mu}+\nabla_{\mu}^*\right)+2m-a\nabla_{\mu}\nabla_{\mu}^*\right)\psi(x),
\end{equation}
where we have defined the forward and backward discrete covariant derivatives as
\begin{align}
\nabla_{\mu}\psi(x)&=\frac{U_{\mu}(x)\psi(x+\hat{\mu})-\psi(x)}{a},\\
\nabla_{\mu}^*\psi(x)&=\frac{\psi(x)-U_{\mu}^{\dagger}(x-\hat{\mu})\psi(x-\hat{\mu})}{a}.
\end{align}

From the Wilson fermion action~\eqref{ch_foundation:eq:Wil_fer} the Wilson Dirac operator reads
\begin{equation}
\label{ch_foundation:eq:DW}
D_{\textrm{W}}+m=\frac{1}{2}\left(\gamma_{\mu}\left(\nabla_{\mu}+\nabla_{\mu}^*\right)-a\nabla_{\mu}\nabla_{\mu}^*\right)+m,
\end{equation}
and the action can be written as
\begin{equation}
S_{\textrm{W}}=a^4\sum_x\bar{\psi}(x)\left(D_{\textrm{W}}+m\right)\psi(x).
\end{equation}
For $N_f$ flavors, an additional sum over a flavor index $i=1,...,N_f$ is required, and $m$ is promoted to a diagonal matrix in flavor space, whose diagonal elements are $m_i$. The fermion mass $m_i$ is commonly expressed in terms of the $\kappa$ parameter
\begin{equation}
\label{ch_foundation:eq:kappa}
\kappa_i=\frac{1}{2am_i+8}.
\end{equation}

The Wilson term $a\nabla_{\mu}\nabla_{\mu}^*$ in the Wilson Dirac operator manifestly breaks chiral symmetry, even in the $m_i=0$ limit, and this symmetry is only restored in the continuum limit. Consequently, the quark mass receives additive renormalization contributions, 
\begin{equation}
m_i^{\textrm{R}}=Z_m\left(m_i-m_{\textrm{cr}}\right),
\end{equation}
since it is not protected against renormalization by the axial symmetry.

The Wilson fermion action has $\mathcal{O}(a)$ cutoff effects, which again can be systematically reduced by using the Symanzik improvement program detailed in Sec.~\ref{ch_foundation:sec:impr}.

%%%%%%%%%%%%%%%%%%%%%%%%%%%%%%%%%%%%%%%%%%%%%%%%%%%%%%%%%%%
%%%%%%%%%%%%%%%%%%%%%%%%%%%%%%%%%%%%%%%%%%%%%%%%%%%%%%%%%%%

\subsection{Wilson twisted mass fermions}
\label{ch_foundation:subsec:tm}

Wilson twisted mass (tm) fermions~\cite{} introduce an imaginary mass term to the Wilson Dirac operator in eq.~\eqref{ch_foundation:eq:DW} of the form
\begin{equation}
i\bar{\psi}(x)\boldsymbol{\mu}\gamma_5\psi(x),
\end{equation}
with the twisted quark mass matrix in flavor space with four flavors of quarks (as will be our case of interest) given by
\begin{equation}
\boldsymbol{\mu}={\textrm{diag}}\left(\mu_u,-\mu_d,-\mu_s,\mu_c\right).
\end{equation}
More generally, the Wilson tm Dirac operator reads
\begin{equation}
D_{\textrm{W}}+\boldsymbol{m}+i\boldsymbol{\mu}\gamma_5.
\end{equation}
By rotating the fields
\begin{gather}
\label{ch_foundation:eq:chiral_rot}
\psi~\to~\psi' = e^{-i\frac{\pi}{2}\gamma_5 \frac{T}{2}}\psi, \quad
\bar{\psi}~\to~\bar{\psi}' = \bar{\psi} e^{-i\frac{\pi}{2}\gamma_5 \frac{T}{2}}, \quad \\
T = {\textrm{diag}}(\eta_u,\eta_d,\eta_s,\eta_c),
\end{gather}
with $\alpha_i\equiv\frac{\pi}{2}\eta_i$ the so called twist angles, defined by
\begin{equation}
{\textrm{cot}}\;\alpha_i=\frac{m_i^{\textrm{R}}}{\mu_i^{\textrm{R}}},
\end{equation}
one retrieves the usual physical (standard) formulation with real fermionic mass
\begin{equation}
M_i^2=m_i^2+\mu_i^2,
\end{equation}
and a chirally rotated Wilson term. The rotated fields $\phi',\bar{\psi}'$ define the so called physical basis, while the unrotated ones $\psi,\bar{\psi}$ define the twisted basis.

In practice we will be working with Wilson tm fermions at full twist
\begin{equation}
\eta_u=\eta_c=-\eta_s=-\eta_d=1,
\end{equation}
which can be obtained by setting the renormalized standard masses $m_i^{\textrm{R}}$ to zero. The procedure to achieve this is explained in Sec.~\ref{ch_ma:sec:matching}.

Considering for simplicity the light sector of mass-degenerate $u$ and $d$ quarks, at full twist the symmetry group $SU(2)_V\times SU(2)_A$ is broken into
\begin{equation}
SU(2)_V\times SU(2)_A\rightarrow[U(1)_A]_1\times[U(1)_A]_2\times[U(1)_V]_3,
\end{equation}
with 
\begin{align}
[U(1)_A]_a=\left\{\begin{matrix}
\psi(x)\rightarrow e^{i\alpha_A^a\gamma_5\frac{\tau^a}{2}}\psi(x) & a=1,2 \\ 
\bar{\psi}(x)\rightarrow \bar{\psi}(x)e^{i\alpha_A^a\gamma_5\frac{\tau^a}{2}} & a=1,2
\end{matrix}\right.,
\end{align}
and
\begin{equation}
[U(1)_V]_3=\left\{\begin{matrix}
\psi(x)\rightarrow e^{i\alpha_A^3\frac{\tau^3}{2}}\psi(x) \\ 
\bar{\psi}(x)\rightarrow \bar{\psi}(x)e^{-i\alpha_A^3\frac{\tau^3}{2}}
\end{matrix}\right..
\end{equation}
This means that at full twist axial symmetries are not completely broken, and thus the twisted mass is protected against additive renormalization,
\begin{equation}
\label{ch_foundation:eq:muR}
\mu_i^{\textrm{R}}=Z_{\mu}(g_0^2,a\mu)\mu_i.
\end{equation}

An important role in our setup is played by the Ward-Takahashi identities (WTI). They will be used to tune the Wilson twisted mass parameters to ensure full twist and to determine the quark masses in the Wilson regularization. Furthermore, they allow to identify the renormalization constant of the twist masses $Z_{\mu}$. For the non-singlet case ($i\neq j$) the WTI for the axial and vector currents, in the continuum limit and in the twisted basis, are as follows (see Appendix~\ref{apex_currents} for the definition of the currents)
\begin{align}
\label{ch_foundation:eq:WTI}
\partial_{\mu}V_{\mu}^{ij}&=(m_i-m_j)S^{ij}+i(\eta_i\mu_i-\eta_j\mu_j)P^{ij},\\
\partial_{\mu}A_{\mu}^{ij}&=(m_i+m_j)P^{ij}+i(\eta_i\mu_i+\eta_j\mu_j)S^{ij}.
\end{align}
Note that at zero twist angle $\eta_u=\eta_d=\eta_s=\eta_c=0$ the twisted and physical basis are the same, and the standard WTIs are recovered. However, at full twist the renormalized standard masses $m_i^{\textrm{R}}$ vanish, which in turn means that the current masses $m_{i}$ in eqs.~(\ref{ch_foundation:eq:WTI}) also vanish (up to cutoff effects). Moreover, the exact flavor symmetry of massless Wilson fermions implies the existence of a point-split vector current $\tilde{V}_{\mu}^{ij}$ on the lattice such that the vector WTI holds exactly on the lattice. In the twisted basis, this current takes the form
\begin{equation}
\tilde{V}_{\mu}^{ij}=\frac{1}{2}\left[\bar{\psi}^i(x)(\gamma_{\mu}-1)U_{\mu}(x)\psi^j(x+\hat{\mu})+\bar{\psi}^i(x+\hat{\mu})(\gamma_{\mu}+1)U_{\mu}^{\dagger}(x)\psi^j(x)\right].
\end{equation}
The conservation of this WTI on the lattice for $\tilde{V}_{\mu}^{ij}$ means that the point-split vector current renormalizes trivially with
\begin{equation}
\label{ch_foundation:eq:ZV=1}
Z_{\tilde{V}}=1.
\end{equation}
Looking at eq.~(\ref{ch_foundation:eq:WTI}) this means that for all flavors
\begin{equation}
\label{ch_foundation:eq:Zmu}
Z_{\mu}=Z_P^{-1}.
\end{equation}


%%%%%%%%%%%%%%%%%%%%%%%%%%%%%%%%%%%%%%%%%%%%%%%%%%%%%%%%%%%
%%%%%%%%%%%%%%%%%%%%%%%%%%%%%%%%%%%%%%%%%%%%%%%%%%%%%%%%%%%

\section{Path integral regularization}
\label{ch_foundation:sec:path}

Having formulated the QCD action in the lattice, we need to see how physical quantities are computed. To do so, we review some aspects of the path integral formulation in Euclidean spacetime. In this formalism, physical quantities are expressed as expectation values of operators
\begin{align}
\left<\mathcal{O}(x_1,...,x_n)\right>&=\frac{1}{\mathcal{Z}}\int\mathcal{D}[\psi,\bar{\psi},U]\mathcal{O}(x_1,...,x_n)e^{-S[\psi,\bar{\psi},U]}, \\
\mathcal{Z}&=\int\mathcal{D}[\psi,\bar{\psi},U]e^{-S[\psi,\bar{\psi},U]}.
\end{align}
This is equivalent to expectation values in statistical mechanics with a Boltzmann factor of $e^{-S[\psi,\bar{\psi},U]}$. The action can be decomposed into its gluon and fermion components $S[\psi,\bar{\psi},U]=S_{\textrm G}[U]+S_{\textrm F}[\psi,\bar{\psi}]$, and fermion variables can be integrated out as
\begin{align}
\left<\mathcal{O}(x_1,...,x_n)\right>&=\frac{1}{\mathcal{Z}}\int\mathcal{D}[U]e^{-S_{\textrm G}[U]}\mathcal{Z}_{\textrm F}\times\\&\left[\frac{1}{\mathcal{Z}_{\textrm F}}\int\mathcal{D}[\psi,\bar{\psi}]\mathcal{O}(x_1,...,x_n)e^{-S_{\textrm F}[\psi,\bar{\psi}]}\right] \\
&=\frac{1}{\mathcal{Z}}\int\mathcal{D}[U]e^{-S_{\textrm G}[U]}\mathcal{Z}_{\textrm F}\left<\mathcal{O}(x_1,...,x_n)\right>_{\textrm F}, \\
\mathcal{Z}_{\textrm F}&=\int\mathcal{D}[\psi,\bar{\psi}]e^{-S_{\textrm F}[\psi,\bar{\psi}]}=\Pi_{i=1}^{N_f}{\textrm{det}}\left(D+m_i\right),
\end{align}
with $D$ the massless Dirac operator. This fermionic determinant can be expressed as an effective action as
\begin{align}
\left<\mathcal{O}(x_1,...,x_n)\right>&=\frac{1}{\mathcal{Z}}\int\mathcal{D}[U]e^{-S_{\textrm G}[U]-S_{\textrm{eff}}[U]}\left<\mathcal{O}(x_1,...,x_n)\right>_{\textrm F}, \\
\mathcal{Z}&=\int\mathcal{D}[U]e^{-S_{\textrm G}[U]-S_{\textrm{eff}}[U]}, \\
S_{\textrm{eff}}[U]&=-\sum_{i=1}^{N_f}{\textrm{log}}\;{\textrm{det}}\left(D+m_i\right).
\end{align}

In order to compute meson observables we will use meson interpolators, which are composite fermionic observables that share the same quantum numbers as the desired meson state. A generic meson interpolator has the form
\begin{equation}
\mathcal{O}_A^{ij}(x)=\bar{\psi}^i(x)\Gamma_A\psi^j(x),
\end{equation}	
with $\Gamma_A$ a Gamma matrix. This way, a meson two-point function reads
\begin{align}
\label{ch_foundation:eq:path_int}
\left<\mathcal{O}_A^{ij}(x_1)\mathcal{O}_B^{ji}(x_2)\right>&=\frac{1}{\mathcal{Z}}\int\mathcal{D}[U]e^{-S_{\textrm G}[U]-S_{\textrm{eff}}[U]}\times\\&\left<\bar{\psi}^i(x_1)\Gamma_A\psi^j(x_1)\psi^i(x_2)\Gamma_B\bar{\psi}^j(x_2)\right>_{\textrm F} \\
&=-\frac{1}{\mathcal{Z}}\int\mathcal{D}[U]e^{-S_{\textrm G}[U]-S_{\textrm{eff}}[U]}\times\\&{\textrm{tr}}\left(\Gamma_AD_i^{-1}(x_1,x_2)\Gamma_BD_j^{-1}(x_2,x_1)\right),
\end{align}
where the trace is over spin indices and $D_i$ the massive Dirac operator for flavor $i$. In order to perform this integral numerically, using the connection with statistical mechanics, a finite set of $N_{\textrm{cnfg}}$ gauge configurations is generated with Boltzmann distribution $e^{-S_{\textrm G}[U]-S_{\textrm{eff}}[U]}$ following a Markov process (see Appendices~\ref{appex_simulations},~\ref{appex_errors}). Then, measurements of the quantity
\begin{equation}
O=-{\textrm{tr}}\left(\Gamma D_i^{-1}(x_1,x_2)\Gamma D_j^{-1}(x_2,x_1)\right),
\end{equation}
are taken in each of these configurations, and the expectation value is computed as
\begin{equation}
\left<O\right>=\frac{1}{N_{\textrm{cnfg}}}\sum_{i}^{N_{\textrm{cnfg}}}O_i.
\end{equation}


%%%%%%%%%%%%%%%%%%%%%%%%%%%%%%%%%%%%%%%%%%%%%%%%%%%%%%%%%%%
%%%%%%%%%%%%%%%%%%%%%%%%%%%%%%%%%%%%%%%%%%%%%%%%%%%%%%%%%%%

\section{Continuum limit}
\label{ch_foundation:sec:continuum-limit}

The lattice regularization provides a natural energy cutoff $a^{-1}$, ensuring that any loop integral is finite in perturbation theory. In perturbative renormalization, it is necessary to take the cutoff to infinity, which in the lattice means taking the lattice spacing to $a\rightarrow0$. If the theory is renormalizable, any physical quantity (e.g. a mass $m_{\textrm{phys}}$) in units of the lattice spacing must vanish in the continuum limit
\begin{equation}
\label{ch_foundation:eq:ma0}
m_{\textrm{phys}}a\rightarrow0,
\end{equation}
since this means that $m_{\textrm{phys}}$ remains finite in this limit. 

Physical quantities are dependent on the couplings of the theory, $m_{\textrm{phys}}(g_0)$, and accordingly change with them. In turn, one can study how the couplings of the theory change in the lattice as one approaches the continuum limit by decreasing $a$. To do so and for simplicity, we assume a single coupling $g_0$, and write the most general local effective action at lattice spacing $a_1$
\begin{equation}
S(a_1)=g_0(a_1)\sum_i\mathcal{O}_i,
\end{equation}
where $\mathcal{O}_i$ are all possible local operators respecting the lattice symmetries. When the lattice spacing is reduced to $a_2<a_1$, all the short-range extra degrees of freedom can be integrated out and reabsorbed into a redefinition of the coupling, such that the new action has the same generic form but with different couplings
\begin{equation}
g_0(a_1)\rightarrow R(g_0(a_1))=g_0(a_2).
\end{equation}
$R$ here stands for the renormalization group (RG) transformation that defines the change in the couplings when varying the lattice spacing.

Since physical quantities change with the couplings, they also do with the lattice spacing, and we want to ensure eq.~(\ref{ch_foundation:eq:ma0}) to ensure renormalizability. It can be observed then that renormalizability corresponds to fixed points $g_0^*$ of the RG transformation
\begin{equation}
R(g_0^*)=g_0^*.
\end{equation}

In the context of SU(N) Yang-Mills theory, perturbation theory shows that at a fixed value of the renormalized coupling $g_R$ the bare coupling runs with the lattice spacing as
\begin{equation}
\label{ch_foundation:eq:beta-func}
a\frac{\partial g_0}{\partial a}\equiv\beta(g_0)=-\beta_0g_0^3-\beta_1g_0^5+...,
\end{equation}
where $\beta_{0,1}$ are universal coefficients (do not depend on the renormalization scheme) and positive for $N=3$, as in the case of QCD. This shows that $g_0=0$ is a fixed point of the RG transformations and thus corresponds to the continuum limit. As the fixed point is in the weak coupling regime, this perturbative argument is expected to be valid. Therefore the continuum limit corresponds to
\begin{equation}
g_0\rightarrow0,
\end{equation}
or in terms of the inverse coupling $\beta$
\begin{equation}
\beta\rightarrow\infty.
\end{equation}

In practice, one cannot numerically simulate at infinite inverse coupling $\beta$. Therefore, physical observables are computed at several finite values of $\beta$. This introduces $\mathcal{O}(a^n)$ cutoff effects in the results, with some power $n$. To obtain results in the continuum, one parameterizes these cutoff effects with some function of the lattice spacing and extrapolates to $a\rightarrow 0$. However, this task is far from trivial~\cite{}. To help in the continuum limit extrapolation, one can systematically reduce lattice artifacts, e.g. from $\mathcal{O}(a)$ to $\mathcal{O}(a^2)$ following the Symanzik improvement program. 

%%%%%%%%%%%%%%%%%%%%%%%%%%%%%%%%%%%%%%%%%%%%%%%%%%%%%%%%%%%
%%%%%%%%%%%%%%%%%%%%%%%%%%%%%%%%%%%%%%%%%%%%%%%%%%%%%%%%%%%

\section{Symanzik improvement program}
\label{ch_foundation:sec:impr}

Symanzik improvement requires improving both the action of the theory and the lattice interpolators that enter the different correlators. 

In order to improve a lattice action, one can describe the target continuum action in terms of an effective action
\begin{equation}
S_{\textrm{eff}}=\int d^4x\sum_kc_k\mathcal{L}_k(x)a^{k-4}.
\end{equation}
Here $\mathcal{L}_0(x)$ is the discretized lattice Lagrangian unimproved, and the higher-dimension terms $\mathcal{L}_k$(x) are all possible Lagrangians built from fermion and gluon field operators that preserve the symmetries of the regularized theory, i.e. the lattice theory, with mass dimension $4+k$, and $c_k$ are numerical coefficients.

In the case of Lattice QCD, we saw that in the Wilson gauge action in eq.~(\ref{ch_foundation:eq:YM-latt}) lattice artifacts appear at $\mathcal{O}(a^2)$, and therefore no $\mathcal{O}(a)$ improvement is required. However, these $\mathcal{O}(a^2)$ effects can be further reduced by adding all possible dimension $4+k=6$ operators that preserve the underlying symmetries of the gauge action. These dimension-6 operators are all three possible ways of writing a closed path in a rectangular lattice with 6 gauge links: planar, twisted and L-shaped rectangles. The action then reads
\begin{equation}
\label{ch_foundation:eq:SG_impr}
S_G=\frac{\beta}{3}\sum_{\mu\nu}\left[c_0\sum_p{\textrm{Re}}\left({\textrm{tr}}\left(1-U_{\mu\nu}(p)\right)\right)+\sum_{i=1}^3c_i\sum_r{\textrm{Re}}\left({\textrm{tr}}\left(1-U^{(i)}(r)\right)\right)\right],
\end{equation}
with $U^{(i)}$ said dimension-6 operators. Tuning the coefficients $c_i$ properly leads to $\mathcal{O}(a^2)$ improvement. In our study, the CLS ensembles we employ (see Sec.~\ref{ch_ma:sec:Sea}) use the so called Lüscher-Weisz gauge action~\cite{}, with these coefficients computed at tree-level
\begin{gather}
\label{ch_foundation:eq:LW}
c_0=\frac{5}{3}, \quad
c_1=-\frac{1}{12}, \quad
c_2=c_3=0.
\end{gather}
Thus, in the Lüscher-Weisz gauge action the only dimension-6 operators that survive are planar rectangles $U^{(1)}$.

We also need to improve the fermion action. Wilson fermions have $\mathcal{O}(a)$ cutoff effects. In order to improve the Wilson fermion action to $\mathcal{O}(a^2)$ we need to look for all possible operators with dimension $4+k=5$ that preserve the lattice symmetries. These are
\begin{gather}
\mathcal{L}_{k=1}^{(1)}=i\bar{\psi}\sigma_{\mu\nu}F_{\mu\nu}\psi,\\
\mathcal{L}_{k=1}^{(2)}=\boldsymbol{m}{\textrm{tr}}\left(F_{\mu\nu}(x)F_{\mu\nu}(x)\right),\\
\mathcal{L}_{k=1}^{(3)}=\boldsymbol{m}^2\bar{\psi}\psi,
\end{gather}
with
\begin{align}
\label{ch_foundation:eq:dim5-op}
\sigma_{\mu\nu}&=\frac{\left[\gamma_{\mu},\gamma_{\nu}\right]}{2i},\\
\hat{F}_{\mu\nu}(x)&=\frac{-i}{8a^2}\left(Q_{\mu\nu}(x)-Q_{\nu\mu}(x)\right),\\
Q_{\mu\nu}&=U_{\mu\nu}(x)+U_{\nu,-\mu}(x)+U_{-\mu,-\nu}(x)+U_{-\nu,\mu}(x).
\end{align}
$\mathcal{L}_{k=1}^{(1),(2)}$ are already present (up to numerical factors) in the original Wilson fermion action and can therefore be reabsorbed in those terms. The $\mathcal{O}(a)$ improved Wilson fermion action reads
\begin{equation}
\label{ch_foundation:eq:DW_impr}
D_{\textrm W}+\boldsymbol{m}+c_{\textrm{sw}}a\frac{1}{2}\sum_{\mu<\nu}\sigma_{\mu\nu}\hat{F}_{\mu\nu},
\end{equation}
with $c_{\textrm{sw}}$ the Sheikholeslami-Wohlert coefficient determined non-perturbatively in~\cite{}.

Improving the lattice action ensures improvement of on-shell quantities such as meson masses. However, if one is interested in matrix elements mediated by some current $\mathcal{J}_{\mu}$, it is also necessary to improve the lattice interpolators that enter into the definition of those currents. In analogy with the improvement of the action, a local operator $O$ is expressed in the Symanzik effective theory as
\begin{equation}
O_{\textrm{eff}}(x)=\sum_ka^kO_k(x).
\end{equation}
Again, $O_k$ are gauge invariant local operators with the right mass dimensions. Following this, a generic n-point function reads
\begin{equation}
\label{ch_foundation:eq:Oimpr}
\left<\Phi\right>=\left<\Phi_0\right>-a\int d^4y\left<\Phi_0\mathcal{L}_1(y)\right>+a\left<\Phi_1\right>+...,
\end{equation}
with 
\begin{align}
\left<\Phi_0\right>&=\left<O_0(x_1)...O_0(x_n)\right>, \\
\left<\Phi_1\right>&=\sum_{i=1}^n\left<O_0(x_1)...O_1(x_i)...O_0(x_n)\right>,
\end{align}
and vacuum expectation values taken in the continuum. The generic form of the $O_k$ operators is a sum over all possible operators $\Psi_k$ with the right mass dimension and that are local and gauge invariant,
\begin{equation}
O_k=\sum_ic_i\Psi_k,
\end{equation}
with $c_i$ some non-perturbatively determined coefficients required to suppress $\mathcal{O}(a^k)$ cutoff effects in the correlation functions. In Sec~\ref{ch_observables} we discuss the details of operator improvement for the case of standard Wilson fermions for the observables of interest.

The $\mathcal{O}(a)$ improved Wilson tm fermion action is analogous to the Wilson case,
\begin{equation}
D_{\textrm W}+\boldsymbol{m}+i\gamma_5\boldsymbol{\mu}+c_{\textrm{sw}}a\frac{1}{2}\sum_{\mu<\nu}\sigma_{\mu\nu}\hat{F}_{\mu\nu}.
\end{equation}
The advantage of Wilson tm fermions is that at full twist (vanishing renormalized standard quark mass) one achieves automatic $\mathcal{O}(a)$ improvement~\cite{}. This means that physical quantities are automatically improved without the need of any improvement coefficients for lattice operators. The following argument is based on the original work~\cite{} to which we refer for a complete proof.

At full twist, the Wilson tm Dirac operator reads
\begin{equation}
D_{\textrm W}+i\boldsymbol{\mu}\gamma_5.
\end{equation}
Working in the twisted basis, this action in the continuum is invariant under a discrete chiral symmetry 
\begin{equation}
\mathcal{R}_5^{1,2}=\left\{\begin{matrix}
\psi(x)\rightarrow i\gamma_5\tau^{1,2}\psi(x) \\ 
\bar{\psi}(x)\rightarrow \bar{\psi}(x)i\gamma_5\tau^{1,2}
\end{matrix}\right.,
\end{equation}
while $\mathcal{L}_{k=1}^{(1)}$ in eq.~(\ref{ch_foundation:eq:dim5-op}) is not 
\begin{equation}
\mathcal{L}_{k=1}^{(1)}\rightarrow-\mathcal{L}_{k=1}^{(1)}.
\end{equation}
This is key for automatic $\mathcal{O}(a)$ improvement. For correlation functions like eq.~(\ref{ch_foundation:eq:Oimpr}), we have that operators may be even or odd under $\mathcal{R}_5$, $\left<\Phi_0\right>$ and $\left<\Phi_1\right>$ having opposite $\mathcal{R}_5$-chirality
\begin{gather}
\left<\Phi_0\right>\rightarrow\pm\left<\Phi_0\right>, \quad \left<\Phi_1\right>\rightarrow\mp\left<\Phi_1\right>.
\end{gather}
This means that for even $\left<\Phi_0\right>$
\begin{gather}
\left<\Phi_0\right>=\left<\Phi_0\right>,
\quad \left<\Phi_0\mathcal{L}^{(1)}_{k=1}\right>=-\left<\Phi_0\mathcal{L}^{(1)}_{k=1}\right>=0, \quad \\ \left<\Phi_1\right>=-\left<\Phi_1\right>=0,
\end{gather}
and thus even operators are automatically $\mathcal{O}(a)$ improved. On the other hand, for odd operators what we have is
\begin{gather}
\left<\Phi_0\right>=-\left<\Phi_0\right>=0, \quad
\left<\Phi_0\mathcal{L}^{(1)}_{k=1}\right>=\left<\Phi_0\mathcal{L}^{(1)}_{k=1}\right>, \quad \\ \left<\Phi_1\right>=\left<\Phi_1\right>,
\end{gather}
and thus they vanish in the continuum. Summing up, the only tuning required for Wilson tm fermions to achieve $\mathcal{O}(a)$ improvement is to set the bare quark mass $m$ to its critical value $m_{\textrm{cr}}$ in order to obtain full twist.

In our particular case, we will be working with a mixed action setup employing standard Wilson quarks in the sea and fully twisted Wilson tm quarks in the valence (see Sec~\ref{ch_ma}). This means valence observables still get residual $\mathcal{O}(a)$ cutoff effects from the sea sector, and thus improvement is still needed. However, these effects are expected to be $\mathcal{O}(g_0^4)$ in perturbation theory.

Finally, we also need to improve the bare gauge coupling, which at $\mathcal{O}(a)$ reads
\begin{equation}
\tilde{g}_0^2=g_0^2\left(1+ab_g{\textrm{tr}}\left(M_q\right)\right),
\end{equation}
with $b_g$ the improvement coefficient, whose value at one-loop is given in~\cite{}.

%%%%%%%%%%%%%%%%%%%%%%%%%%%%%%%%%%%%%%%%%%%%%%%%%%%%%%%%%%%
%%%%%%%%%%%%%%%%%%%%%%%%%%%%%%%%%%%%%%%%%%%%%%%%%%%%%%%%%%%

\section{Scale setting}
\label{ch_foundation:sec:ss}

In the lattice, all physical observables are computed in units of the lattice spacing $a$. Thus, in order to make any prediction, it is necessary to determine $a$ in physical units. This task is called scale setting. It involves the precise determination of a reference observable, called the scale, in physical units, to which any other observable is compared in order to extract the value of the latter in physical units. As an example, we could use the proton mass $m_{\textrm{proton}}$ as a reference scale, and calculate the ratio of it to a given mass $m_i$
\begin{equation}
R_i=\frac{m_i}{m_{\textrm{proton}}}.
\end{equation}
After computing the continuum limit of $R_i$, we can extract the physical mass $m_i$ as
\begin{equation}
\label{ch_ss:eq:R}
m_i^{\textrm{ph}}=R_i(a=0)\times m_{\textrm{proton}}^{\textrm{exp}}.
\end{equation}
Here, the proton mass is used as a reference scale, and comparing any lattice observable to it allows to extract the latter in physical units, once the continuum limit is performed. This procedure is equivalent to finding the value of the lattice spacing in physical units, since it can be extracted as 
\begin{equation}
a=\frac{(am_{\textrm{proton}})^{\textrm latt}}{m_{\textrm{proton}}^{\textrm{exp}}}.
\end{equation}
From eq.~(\ref{ch_ss:eq:R}) it is clear that when aiming for precise lattice calculations of any physical observable like $m_i$, a reliable and precise scale setting is of the utmost importance. In this example this means being able to determine $m_{\textrm{proton}}$ with high accuracy in the lattice in order to compute the ratios $R_i$, controlling the continuum limit of $R_i$ and having a high precision in $m_{\textrm{proton}}^{\textrm{exp}}$.

Instead of using a phenomenological scale like $m_{\textrm{proton}}$, another choice is to use intermediate scales, like the gradient flow scale $t_0$ introduced in Sec.~\ref{ch_observables:sec:Flow}. This quantity can be computed with very high precision in the lattice while it cannot be measured experimentally. To obtain its physical value, one constructs a dimensionless quantity $(\sqrt{t_0}\Lambda)^{\textrm{latt}}$ with some phenomenological quantity $\Lambda$ in the lattice. After performing the continuum limit, the physical value of $t_0$ can be extracted as
\begin{equation}
\sqrt{t_0^{\textrm{ph}}}=\frac{\left.\begin{matrix}
\left(\sqrt{t_0}\Lambda\right)^{\textrm{latt}}
\end{matrix}\right|_{a=0}}{\Lambda^{\textrm{exp}}}.
\end{equation}
Once the physical value of $t_0$ is found, it can be used as an intermediate scale against which any other quantity $\Lambda'$ in the lattice can be compared in order to extract the latter in physical units. For this purpose, one performs a continuum extrapolation of $\sqrt{t_0}\Lambda'$ and obtains the physical value of $\Lambda'$ as
\begin{equation}
\Lambda^{\textrm{' ph}}=\frac{\left.\begin{matrix}
\left(\sqrt{t_0}\Lambda'\right)^{\textrm{latt}}
\end{matrix}\right|_{a=0}}{\sqrt{t_0^{\textrm{ph}}}}.
\end{equation}
This quantity is already a prediction of the lattice.

%%%%%%%%%%%%%%%%%%%%%%%%%%%%%%%%%%%%%%%%%%%%%%%%%%%%%%%%%%%
%%%%%%%%%%%%%%%%%%%%%%%%%%%%%%%%%%%%%%%%%%%%%%%%%%%%%%%%%%%
