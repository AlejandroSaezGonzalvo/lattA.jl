\chapter{QCD on the lattice}%\addcontentsline{toc}{chapter}{QCD on the lattice}

%%%%%%%%%%%%%%%%%%%%%%%%%%%%%%%%%%%%%%%%%%%%%%%%%%%%%%%%%%%
%%%%%%%%%%%%%%%%%%%%%%%%%%%%%%%%%%%%%%%%%%%%%%%%%%%%%%%%%%%
%%%%%%%%%%%%%%%%%%%%%%%%%%%%%%%%%%%%%%%%%%%%%%%%%%%%%%%%%%%
%%%%%%%%%%%%%%%%%%%%%%%%%%%%%%%%%%%%%%%%%%%%%%%%%%%%%%%%%%%

\label{ch_foundation}

%%%%%%%%%%%%%%%%%%%%%%%%%%%%%%%%%%%%%%%%%%%%%%%%%%%%%%%%%%%
%%%%%%%%%%%%%%%%%%%%%%%%%%%%%%%%%%%%%%%%%%%%%%%%%%%%%%%%%%%
%%%%%%%%%%%%%%%%%%%%%%%%%%%%%%%%%%%%%%%%%%%%%%%%%%%%%%%%%%%
%%%%%%%%%%%%%%%%%%%%%%%%%%%%%%%%%%%%%%%%%%%%%%%%%%%%%%%%%%%

\section{Introduction}
\label{ch_foundation:sec:general}

The theory that describes the strong interaction between quarks and gluons is called Quantum Chromodynamics or QCD. 

The underlying symmetry of QCD is associated with the non-abelian $SU(N=3)$ Lie group. The elements of this group are non-commuting, traceless unitary matrices $\Omega$ with unit determinant $\textrm{det}\;\Omega=1$. In order to have a gauge theory we must allow these elements to depend on space-time coordinates. The map
\begin{equation}
\Omega(x)=e^{i\alpha^{(a)}(x)T^{(a)}},
\end{equation}
provides a local parameterization of the group near the identity with coordinates $\alpha^{(a)}(x)$. Summation over $a=1,...,N^2-1=8$ is implicit and $T^{(a)}$ are the 8 generators of the $SU(3)$ Lie group. These live in the Lie algebra $su(3)$, which is the tangent space of the group $SU(3)$ at the identity $I\in SU(3)$. They satisfy the commutation relations
\begin{equation}
\left[T^{(a)}, T^{(b)}\right]=if_{abc}T^{(c)},
\end{equation}
where $f_{abc}$ are the structure constants of the group, given in Appendix~\ref{apex_SU3}. Unitarity of the group elements means that
\begin{equation}
\Omega^{\dagger}\Omega=1.
\end{equation}

The group elements $\Omega$ must be in some representation which determines how they act on a vector space where the degrees of freedom of the theory live. In QCD these are quarks and gluons. The former are described by spinor fields $\psi_{\alpha,i},\bar{\psi}_{\alpha,i}$. They carry a Dirac spinor index $\alpha=1,2,3,4$ and a flavor index $i=1,...,N_f$, to each flavor corresponding a different mass (in Nature $N_f=6$). They transform under SU(3) in the fundamental representation,
\begin{gather}
\psi(x)\to\Omega(x)\psi(x), \quad \bar{\psi}(x)\to\bar{\psi}(x)\Omega(x)^{\dagger}.
\end{gather}
In this representation the group generators $T^{(a)}$ are given by the Gell-Mann matrices (see Appendix~\ref{apex_SU3}), and quark fields live in a 3-dimensional vector space, having an additional index $c=1,2,3$ called color. As spinor fields, their dynamics is governed by the Dirac action, which in Euclidean metric $g_{\mu\nu}=\textrm{diag}(+1,+1,+1,+1)$ reads
\begin{equation}
S_{\textrm{F}}=\sum_{i=1}^{N_f}\int d^4x\;\bar{\psi}^{i}(x)\left(\gamma_{\mu}\partial_{\mu}+m_i\right)\psi^i(x).
\end{equation} 
Here we have implicitly summed over the repeated $\mu$ index and omitted the spinor and color indices. This action is invariant under global $SU(3)$ transformations ($\Omega$ independent of $x$). In order to promote this to a local or gauge symmetry, we must replace the derivative by a covariant one
\begin{equation}
\partial_{\mu}\psi(x)\to D_{\mu}\psi(x)=\partial_{\mu}\psi(x)+iA_{\mu}(x)\psi(x),
\end{equation}
with $A_{\mu}$ a new gauge field which must transform under $SU(3)$ in the adjoint representation
\begin{equation}
\label{ch_foundation:eq:Agauge}
A_{\mu}(x)\to\Omega(x)A_{\mu}(x)\Omega^{\dagger}(x)+i\Omega(x)\partial_{\mu}\Omega^{\dagger}(x),
\end{equation}
in order to ensure gauge invariance of the Dirac action. This field $A_{\mu}$ lives in the $su(3)$ algebra, and thus it is a hermitian, traceless matrix which can be decomposed as a linear combination of the algebra generators $T^{(a)}$
\begin{equation}
A_{\mu}=A_{\mu}^{(a)}T^{(a)},
\end{equation}
where we again implicitly sum over the repeated index $a$. The gauge or gluon fields must have a kinetic piece in the action for them to be dynamical. This is given by the Yang-Mills action
\begin{equation}
\label{ch_foundation:eq:SYM}
\frac{1}{2g_0^2}\int d^4x\;\textrm{tr}(F_{\mu\nu}(x)F_{\mu\nu}(x)),
\end{equation}
which describes dynamical gauge fields in the absence of matter. The dimensionless parameter $g_0$ is the coupling constant and the energy strength tensor $F_{\mu\nu}$ is given by
\begin{align}
F_{\mu\nu}(x)&=\partial_{\mu}A_{\nu}(x)-\partial_{\nu}A_{\mu}(x)+i\left[A_{\mu}(x),A_{\nu}(x)\right].
\end{align}
Again, this object lives in the $su(3)$ algebra and can be expressed as
\begin{equation}
F_{\mu\nu}=F_{\mu\nu}^{(a)}T^{(a)}.
\end{equation}
From the transformation in eq.~(\ref{ch_foundation:eq:Agauge}) we derive the transformation relations of $F_{\mu\nu}$
\begin{align}
F_{\mu\nu}(x)&\to\Omega(x)F_{\mu\nu}(x)\Omega^{\dagger}(x).
\end{align}

Finally, the QCD action in continuum space-time is given by 
\begin{align}
\label{ch_foundation:eq:QCD}
S_{\textrm{QCD}}&=\sum_{i=1}^{N_f}\int d^4x\;\bar{\psi}^i(x)\left(\gamma_{\mu}D_{\mu}+m_i\right)\psi^i(x) \\
&+\frac{1}{2g_0^2}\int d^4x\;{\textrm{tr}}\left(F_{\mu\nu}(x)F_{\mu\nu}(x)\right).
\end{align}
The only parameters of this action are the quark masses $m_i$ and the dimensionless coupling constant $g_0$.

As mentioned in the Introduction, QCD is strongly coupled at low energies or large distances. Consequently, one cannot rely on perturbation theory to compute physical observables, as an expansion on powers of the coupling does not converge. The only known first-principles method other than perturbation theory to perform theoretical predictions in Quantum Field Theory is Lattice Quantum Field Theory, Lattice QCD when applied to the study of Quantum Chromodynamics. This method is based on the discretization of space-time into a hypercubic box or lattice
\begin{equation}
\Lambda=\{n_0,n_1,n_2,n_3|n_0=0,...,T/a-1;n_i=0,...,L/a-1;i=1,2,3\},
\end{equation} 
where $a$ is the lattice spacing between two adjacent sites, and $L,T$ are the spatial and temporal lattice extents (in physical units) respectively. The discretization of space-time and the introduction of a finite lattice spacing $a$ provide a natural energy cutoff for momenta $\sim a^{-1}$, removing UV divergences. On the other hand, the finite volume lattice ensures the absence of IR divergences. This implies that the lattice formulation can be seen as a way to regularize any particular Quantum Field Theory. However, this also implies the presence of finite volume and discretization effects, which should be removed from any prediction. To do this, after computing some physical observable on the lattice setup, one must perform a continuum extrapolation to obtain results at $a\rightarrow0$ and simulate large enough volumes in order to be able to neglect the effects associated with finite volume. If the theory is renormalizable, physical quantities will remain finite in the continuum limit.

After discretizing space-time, fields are placed at the lattice sites $n\in\Lambda$. Fermion fields are represented by
\begin{gather}
\psi(n),\bar{\psi}(n), \quad
n\in\Lambda.
\end{gather}
For the gauge fields, it will be helpful to use the definition of a parallel transporter for $SU(N)$. An $N$-component unit vector $\boldsymbol{v}$ is parallel transported along a curve in space-time parameterized by $z_{\mu}(t)$ from point $z_{\mu}(a)=x$ to $z_{\mu}(b)=y$ as
\begin{align}
\boldsymbol{v}(b)&=P(y,x)\boldsymbol{v}(a),\\
P(y,x)&=\mathcal{P}e^{i\int_x^yA_{\mu}(z)dz_{\mu}},
\end{align}
with $A_{\mu}$ the $SU(N)$ gauge field. This implies that a fermion in the fundamental representation acquires a phase factor of $P(y,x)$ when going from $x$ to $y$. This parallel transporter is referred to as a gauge link and its discrete version will be used for the gauge degrees of freedom on the lattice. It is an element of the group and transforms as
\begin{equation}
P(x,y)\to\Omega(x)P(x,y)\Omega^{\dagger}(y).
\end{equation}

Once the fields on the lattice have been defined, the next step is to discretize the QCD action. This is done by formulating it in a finite box $\Lambda$ in terms of the aforementioned fields in such a way that in the continuum limit $a\rightarrow 0$ the continuum QCD action is recovered. We discuss this in the following sections.	

The Chapter is organized as follows. In Sec.~\ref{ch_foundation:sec:Gauge} we present the Wilson formulation of the gauge action on the lattice, expressed in terms of the link variables. In Sec.~\ref{ch_foundation:sec:Fermions} we present various methods for discretizing the fermion action. In Sec.~\ref{ch_foundation:subsec:Naive} we discuss the issue of fermion doublers that arise with a naive fermion discretization and its connection to the formulation of chiral symmetry on the lattice. We also provide some brief comments on Ginsparg-Wilson fermions. In Sec.~\ref{ch_foundation:subsec:Wilson} we present the solution to the doublers problem proposed by Wilson, which consists in adding a term that explicitly breaks chiral symmetry. This term gives an additional mass to the doublers that grows with the inverse of the lattice spacing $a$, which helps to distinguish them. In Sec.~\ref{ch_foundation:subsec:tm} we discuss a modification of Wilson fermions which adds a chirally rotated mass term. This regularization offers several advantages that will be crucial for our study. In Sec.~\ref{ch_foundation:sec:path} we review some of the fundamental concepts of the path integral formalism and how expectation values are computed numerically on the lattice. In Sec.~\ref{ch_foundation:sec:continuum-limit} we review some concepts of renormalizability and the continuum limit on the lattice. In Sec.~\ref{ch_foundation:sec:impr} we discuss the Symanzik improvement program, which allows to reduce cutoff effects associated with the lattice action and fields, thus facilitating the task of performing the continuum limit. Finally, in Sec.~\ref{ch_foundation:sec:ss} the procedure for setting the scale on the lattice is discussed. This is necessary in order to extract lattice predictions in physical units.

%%%%%%%%%%%%%%%%%%%%%%%%%%%%%%%%%%%%%%%%%%%%%%%%%%%%%%%%%%%
%%%%%%%%%%%%%%%%%%%%%%%%%%%%%%%%%%%%%%%%%%%%%%%%%%%%%%%%%%%
%%%%%%%%%%%%%%%%%%%%%%%%%%%%%%%%%%%%%%%%%%%%%%%%%%%%%%%%%%%
%%%%%%%%%%%%%%%%%%%%%%%%%%%%%%%%%%%%%%%%%%%%%%%%%%%%%%%%%%%

\section{Pure gauge SU(3) on the lattice}
\label{ch_foundation:sec:Gauge}

on the lattice, gluon fields can be defined by the link variables $U_{\mu}(x)\in SU(3)$ that act as a discrete version of the gauge transporters connecting points $x$ and $x+\hat{\mu}$, with $\hat{\mu}=\{a\hat{x}_0,a\hat{x}_1,a\hat{x}_2,a\hat{x}_3\}$
\begin{equation}
\label{ch_foundation:eq:U}
U_{\mu}(x)=\exp\left(iaA_{\mu}(x)\right).
\end{equation}
These fields transform as 
\begin{equation}
\label{ch_foundation:eq:U_transf}
U_{\mu}(x)\to\Omega(x)U_{\mu}(x)\Omega^{\dagger}(x+\hat{\mu}),
\end{equation}
and they live on the links of the lattice that connect sites $x$ and $x+\hat{\mu}$.

A common discretization of the gluonic action is the Wilson gauge action~\citep{Wilson:1974sk}, which is expressed in terms of the link variables $U_{\mu}(x)$
\begin{equation}
\label{ch_foundation:eq:SG}
S_{\textrm{G}}=\frac{1}{g_0^2}\sum_x\sum_{\mu,\nu}{\textrm{Re\; tr}}\left(1-U_{\mu\nu}(x)\right),
\end{equation} 
where $U_{\mu\nu}(x)$ is the plaquette centered on the lattice site $x$
\begin{equation}
\label{ch_foundation:eq:plaq}
U_{\mu\nu}(x)=U_{\mu}(x)U_{\nu}(x+\hat{\mu})U_{\mu}^{\dagger}(x+\hat{\nu})U_{\nu}^{\dagger}(x),
\end{equation}
and we have used 
\begin{equation}
U_{\mu}^{\dagger}(x)=U_{-\mu}(x+\hat{\mu}).
\end{equation}
Using the Baker-Campbell-Hausdorff formula iteratively
\begin{equation}
\exp\left(A\right)\exp\left(B\right)=\exp\left(A+B+\frac{1}{2}\left[A,B\right]+...\right),
\end{equation}
and using eq.~(\ref{ch_foundation:eq:U}) we get
\begin{equation}
\label{ch_foundation:eq:YM-latt}
S_{\textrm{G}}=a^4\frac{\beta}{6}\sum_x\sum_{\mu,\nu}{\textrm{tr}}\left(F_{\mu\nu}^2(x)\right)+\mathcal{O}(a^2),
\end{equation}
where we introduced the inverse coupling
\begin{equation}
\beta=\frac{6}{g_0^2}.
\end{equation}
Taking the continuum limit $a^4\sum_x\rightarrow\int d^4x$ we recover the continuum Yang-Mills action.

Eq.~(\ref{ch_foundation:eq:YM-latt}) shows that the effects associated with the discretization of space-time are of order $\mathcal{O}(a^2)$ for the Wilson gauge action. The discretization of the SU(3) pure Yang-Mills action is not unique, and different choices result in different cutoff effects.

The $\mathcal{O}(a^2)$ cutoff effects present in the Wilson regularization of the gauge action can be further reduced by adding additional terms that respect the symmetries of the theory following the Symanzik improvement program. One such choice is the Lüscher-Weisz action~\citep{Luscher:1984xn}, which we discuss in Sec.~\ref{ch_foundation:sec:impr}.

%%%%%%%%%%%%%%%%%%%%%%%%%%%%%%%%%%%%%%%%%%%%%%%%%%%%%%%%%%%
%%%%%%%%%%%%%%%%%%%%%%%%%%%%%%%%%%%%%%%%%%%%%%%%%%%%%%%%%%%


\section{Introducing fermions on the lattice}
\label{ch_foundation:sec:Fermions}

%%%%%%%%%%%%%%%%%%%%%%%%%%%%%%%%%%%%%%%%%%%%%%%%%%%%%%%%%%%
%%%%%%%%%%%%%%%%%%%%%%%%%%%%%%%%%%%%%%%%%%%%%%%%%%%%%%%%%%%

After discretizing the SU(3) gauge action, we still need to find a suitable discrete version of the fermion action in eq.~(\ref{ch_foundation:eq:QCD}) to fully formulate QCD on the lattice. It will be shown that theoretical challenges arise with the naive fermion discretization and how these can be addressed with alternative formulations.

\subsection{Naive fermions}
\label{ch_foundation:subsec:Naive}

To discretize the continuum fermion action in the absence of gauge fields, considering only one flavor with mass $m$,
\begin{equation}
S_{\textrm{F}}=\int d^4x\bar{\psi}(x)\left(\gamma_{\mu}\partial_{\mu}+m\right)\psi(x),
\end{equation}
the derivative $\partial_{\mu}$ needs to take a discrete form, which can be done easily by
\begin{equation}
\partial_{\mu}\psi(x)\rightarrow\hat{\partial}_{\mu}\psi(x)=\frac{1}{2a}\left(\psi(x+\hat{\mu})-\psi(x-\hat{\mu})\right).
\end{equation}
To respect gauge symmetry in our action, we must promote the derivative $\hat{\partial}_{\mu}$ to a covariant one, as in the continuum case. To achieve this, we note that terms like
\begin{equation}
\bar{\psi}(x)\psi(x+\hat{\mu}),
\end{equation}
which arise from $\bar{\psi}(x)\hat{\partial}_{\mu}\psi(x)$ are not gauge invariant
\begin{equation}
\bar{\psi}(x)\psi(x+\hat{\mu})\to\bar{\psi}(x)\Omega^{\dagger}(x)\Omega(x+\hat{\mu})\psi(x+\hat{\mu}).
\end{equation}
The solution is to introduce the link variable or parallel transporter $U_{\mu}(x)$ from site $x$ to $x+\hat{\mu}$ defined in eq.~(\ref{ch_foundation:eq:U}) which transforms as in eq.~(\ref{ch_foundation:eq:U_transf}). This way, the discretized fermion action reads
\begin{equation}
\label{ch_foundation:eq:naive}
S_{\textrm{F}}=a^4\sum_x\bar{\psi}(x)\left(\gamma_{\mu}\frac{U_{\mu}(x)\psi(x+\hat{\mu})-U_{\mu}^{\dagger}(x-\hat{\mu})\psi(x-\hat{\mu})}{2a}+m\psi(x)\right).
\end{equation}

However, this naive formulation of the fermion action exhibits the problem of doubling: despite the fact that we wrote our action to describe one fermion of mass $m$, for finite lattice spacing $a$ additional poles with the same ground state energy appear, spoiling the dynamics of the theory. These unwanted additional poles are known as doublers. To see how they appear, we consider the massive Dirac operator $D(x,y)$ in the continuum, defined such that
\begin{equation}
S_{\textrm{F}}=\int d^4xd^4y\;\bar{\psi}(x)D(x,y)\psi(y).
\end{equation}
on the lattice this takes the form
\begin{equation}
S_{\textrm{F}}=a^4\sum_{n,m}\bar{\psi}(n)D(n,m)\psi(m),
\end{equation}
with the Dirac operator for the naive fermion formulation given by
\begin{equation}
D(n,m)=\gamma_{\mu}\frac{U_{\mu}(n)\delta_{n+\hat{\mu},m}-U_{\mu}^{\dagger}(n-\hat{\mu})\delta_{n-\hat{\mu},m}}{2a}+m\delta_{n,m}.
\end{equation}
Restricting to the free massless fermion case $U_{\mu}=1$ for illustration, upon Fourier transform we get
\begin{align}
\tilde{D}(p,q)&=\frac{1}{V}\sum_{n,m}e^{-ip\times na}D(n,m)e^{iq\times ma}\\
&=\frac{1}{V}\sum_{n,m}e^{-i(p-q)na}\left(\gamma_{\mu}\frac{e^{iq_{\mu}a}-e^{-iq_{\mu}a}}{2a}\right)\\
&=\delta(p-q)\tilde{D}(p),
\end{align}
with $V$ the 4-dimensional volume of the lattice and
\begin{equation}
\tilde{D}(p)=\sum_{\mu}\frac{i}{a}\gamma_{\mu}\textrm{sin}(p_{\mu}a),
\end{equation}
where we made explicit again the sum over $\mu$. The inverse of this operator can be computed as
\begin{equation}
\label{ch_foundations:eq:Dinv}
\tilde{D}^{-1}(p)=\frac{ia^{-1}\sum_{\mu}\gamma_{\mu}sin(p_{\mu}a)}{a^{-2}\sum_{\mu}sin(p_{\mu}a)^2}.
\end{equation}
We can see that in the continuum $a\rightarrow0$ we recover the correct form of the Dirac operator
\begin{equation}
\tilde{D}(p)^{-1}|_{m=0}\rightarrow_{a\rightarrow0}\frac{-i\gamma_{\mu}p_{\mu}}{p^2}
\end{equation}
with one single pole at $p^2=0$. However, at finite lattice spacing, the denominator in eq.~(\ref{ch_foundations:eq:Dinv}) vanishes not only for $p=(0,0,0,0)$ but also for 
\begin{equation}
p=(\pi/a,0,0,0),\;(0,\pi/a,0,0),\;...,\;(\pi/a,\pi/a,\pi/a,\pi/a).
\end{equation}
These are 15 unwanted poles, the doublers, that only disappear in the continuum, once they become infinitely heavy. These doublers have the same ground energy as the true pole at $p^2=0$ and they affect the dynamics of the theory. 

The problem of doublers is related to chiral symmetry and its implementation on the lattice. Chiral symmetry in continuum QCD can be expressed as 
\begin{equation}
\label{ch_foundation:eq:chirality}
\{D,\gamma_5\}=0,
\end{equation}
with $D$ the Dirac operator. The Nielsen-Ninomiya~\citep{Nielsen:1980rz,Nielsen:1981hk} Theorem states that one cannot implement chiral symmetry in the way of eq.~(\ref{ch_foundation:eq:chirality}) on the lattice without the appearance of doublers. In this lattice formulation of chiral symmetry, there must be an equal number of right movers and left movers. In particular, this means having just one pole is not possible. Ginsparg and Wilson~\citep{Ginsparg:1981bj} proposed a suitable version of chiral symmetry for the lattice as
\begin{equation}
\label{ch_foundation:eq:chirality-latt}
\{D,\gamma_5\}=aD\gamma_5D,
\end{equation}
such that in the continuum eq.~(\ref{ch_foundation:eq:chirality}) is recovered. With this definition of chiral symmetry on the lattice, it is possible to construct Dirac operators that satisfy eq.~(\ref{ch_foundation:eq:chirality-latt}) and are free of doublers. 

If one is not interested in studying physics related to chiral symmetry, another choice is to build a Dirac operator that explicitly breaks chiral symmetry but removes the doublers. Wilson fermions and Wilson twisted mass fermions are examples of such a choice, which we will now study.

%%%%%%%%%%%%%%%%%%%%%%%%%%%%%%%%%%%%%%%%%%%%%%%%%%%%%%%%%%%
%%%%%%%%%%%%%%%%%%%%%%%%%%%%%%%%%%%%%%%%%%%%%%%%%%%%%%%%%%%

\subsection{Wilson fermions}
\label{ch_foundation:subsec:Wilson}

Wilson proposed~\citep{Wilson:1974sk} adding an extra term to the naive fermion action in eq.~\eqref{ch_foundation:eq:naive} to distinguish the doublers from the true pole. The Wilson fermion action reads
\begin{equation}
\label{ch_foundation:eq:Wil_fer}
S_{\textrm{W}}=a^4\sum_x\bar{\psi}(x)\frac{1}{2}\left(\gamma_{\mu}\left(\nabla_{\mu}+\nabla_{\mu}^*\right)+2m-a\nabla_{\mu}\nabla_{\mu}^*\right)\psi(x),
\end{equation}
where we have defined the forward and backward discrete covariant derivatives as
\begin{align}
\nabla_{\mu}\psi(x)&=\frac{U_{\mu}(x)\psi(x+\hat{\mu})-\psi(x)}{a},\\
\nabla_{\mu}^*\psi(x)&=\frac{\psi(x)-U_{\mu}^{\dagger}(x-\hat{\mu})\psi(x-\hat{\mu})}{a}.
\end{align}
From the Wilson fermion action~\eqref{ch_foundation:eq:Wil_fer} the Wilson Dirac operator reads
\begin{equation}
\label{ch_foundation:eq:DW}
D=D_{\textrm{W}}+m=\frac{1}{2}\left(\gamma_{\mu}\left(\nabla_{\mu}+\nabla_{\mu}^*\right)-a\nabla_{\mu}\nabla_{\mu}^*\right)+m,
\end{equation}
where we have introduced the massless Wilson Dirac operator $D_{\textrm{W}}$, and the action can be written as
\begin{equation}
S_{\textrm{W}}=a^4\sum_x\bar{\psi}(x)\left(D_{\textrm{W}}+m\right)\psi(x).
\end{equation}
For $N_f$ flavors, an additional sum over a flavor index $i=1,...,N_f$ is required, and $m$ is promoted to a diagonal matrix in flavor space, whose diagonal elements are $m_i$. The fermion mass $m_i$ is commonly expressed in terms of the $\kappa$ parameter
\begin{equation}
\label{ch_foundation:eq:kappa}
\kappa_i=\frac{1}{2am_i+8}.
\end{equation}
For the free case, the momentum space massless Dirac operator reads
\begin{equation}
\tilde{D}_{\textrm{W}}(p)=\frac{i}{a}\sum_{\mu}\gamma_{\mu}\textrm{sin}(p_{\mu}a)+\frac{1}{a}\sum_{\mu}\left(1-\textrm{cos}(p_{\mu}a)\right).
\end{equation}
The second summand in the right-hand side comes from the Wilson extra term $a\nabla_{\mu}\nabla_{\mu}^*$ in the action, and it is responsible for giving an additional mass term to the doublers
\begin{equation}
\frac{2l}{a},
\end{equation}
where $l$ is the number of momentum components with $p_{\mu}=\pi/a$ for the doubler. This additional mass distinguishes the doublers from the true pole and makes them decouple as we approach the continuum limit.

The Wilson term $a\nabla_{\mu}\nabla_{\mu}^*$ in the Wilson Dirac operator manifestly breaks chiral symmetry, even in the $m_i=0$ limit, and this symmetry is only restored in the continuum limit. Consequently, the quark mass receives additive renormalization contributions, 
\begin{equation}
m_i^{\textrm{R}}=Z_m\left(m_i-m_{\textrm{cr}}\right),
\end{equation}
since it is no longer protected against them by the axial symmetry.

The Wilson fermion action has $\mathcal{O}(a)$ cutoff effects, which can be systematically reduced by using the Symanzik improvement program detailed in Sec.~\ref{ch_foundation:sec:impr}.

%%%%%%%%%%%%%%%%%%%%%%%%%%%%%%%%%%%%%%%%%%%%%%%%%%%%%%%%%%%
%%%%%%%%%%%%%%%%%%%%%%%%%%%%%%%%%%%%%%%%%%%%%%%%%%%%%%%%%%%

\subsection{Wilson twisted mass fermions}
\label{ch_foundation:subsec:tm}

Wilson twisted mass (tm) fermions~\citep{Frezzotti:1999vv,Frezzotti:2000nk,Frezzotti:2001ea,Frezzotti:2003ni,Shindler:2007vp} introduce an imaginary mass term to the Wilson Dirac operator in eq.~\eqref{ch_foundation:eq:DW} of the form
\begin{equation}
i\bar{\psi}(x)\boldsymbol{\mu}\gamma_5\psi(x),
\end{equation}
with $\boldsymbol{\mu}$ the twisted quark mass matrix in flavor space. More generally, the Wilson tm Dirac operator reads
\begin{equation}
D=D_{\textrm{W}}+\boldsymbol{m}+i\boldsymbol{\mu}\gamma_5.
\end{equation}
Our case of interest for this thesis will be
\begin{align}
\boldsymbol{\mu}&={\textrm{diag}}\left(\mu_u,-\mu_d,-\mu_s,\mu_c\right), \\
\boldsymbol{m}&={\textrm{diag}}\left(m_u,m_d,m_s,m_c\right).
\end{align}
By rotating the fields
\begin{gather}
\label{ch_foundation:eq:chiral_rot}
\psi~\to~\psi' = e^{-i\frac{\pi}{2}\gamma_5 \frac{T}{2}}\psi, \quad
\bar{\psi}~\to~\bar{\psi}' = \bar{\psi} e^{-i\frac{\pi}{2}\gamma_5 \frac{T}{2}}, \quad \\
T = {\textrm{diag}}(\eta_u,\eta_d,\eta_s,\eta_c),
\end{gather}
with $\alpha_i\equiv\frac{\pi}{2}\eta_i$ the so called twist angles, defined with the renormalized standard and twisted quark masses as
\begin{equation}
{\textrm{cot}}\;\alpha_i=\frac{m_i^{\textrm{R}}}{\mu_i^{\textrm{R}}},
\end{equation}
one retrieves the usual physical (standard) formulation with real fermionic mass
\begin{equation}
M_i^2=m_i^2+\mu_i^2,
\end{equation}
and a chirally rotated Wilson term. The rotated fields $\psi',\bar{\psi}'$ define the so called physical basis, while the unrotated ones $\psi,\bar{\psi}$ define the twisted basis.

In practice we will be working with Wilson tm fermions at maximal twist
\begin{equation}
\label{ch_foundation:eq:mte}
\eta_u=\eta_c=-\eta_s=-\eta_d=1,
\end{equation}
which can be obtained by setting the renormalized standard masses $m_i^{\textrm{R}}$ to zero. The procedure to achieve this is explained in Sec.~\ref{ch_ma:sec:matching}.

Considering for simplicity the light sector of mass-degenerate up/down ($u$ and $d$) quarks, at maximal twist the symmetry group $SU(2)_V\times SU(2)_A$ is broken into
\begin{equation}
SU(2)_V\times SU(2)_A\rightarrow[U(1)_A]_1\times[U(1)_A]_2\times[U(1)_V]_3,
\end{equation}
with 
\begin{align}
\left[U(1)_A\right]_a=\left\{\begin{matrix}
\psi(x)\rightarrow e^{i\alpha_A^a\gamma_5\frac{\tau^a}{2}}\psi(x) & a=1,2 \\ 
\bar{\psi}(x)\rightarrow \bar{\psi}(x)e^{i\alpha_A^a\gamma_5\frac{\tau^a}{2}} & a=1,2
\end{matrix}\right.,
\end{align}
and
\begin{equation}
\left[U(1)_V\right]_3=\left\{\begin{matrix}
\psi(x)\rightarrow e^{i\alpha_A^3\frac{\tau^3}{2}}\psi(x) \\ 
\bar{\psi}(x)\rightarrow \bar{\psi}(x)e^{-i\alpha_A^3\frac{\tau^3}{2}}
\end{matrix}\right.,
\end{equation}
with $\tau^{a}$ the Pauli matrices. This means that at maximal twist axial symmetries are not completely broken, and thus the twisted mass is protected against additive renormalization,
\begin{equation}
\label{ch_foundation:eq:muR}
\mu_i^{\textrm{R}}=Z_{\mu}(g_0^2,a\mu)\mu_i.
\end{equation}

An important role in our setup is played by the Ward-Takahashi identities (WTI). They will be used to tune the Wilson twisted mass parameters to ensure maximal twist. Furthermore, they allow to identify the renormalization constant of the twisted masses $Z_{\mu}$. For the non-singlet case ($i\neq j$) the WTI for the axial and vector currents, in the continuum limit and in the twisted basis, read (see eqs.~(\ref{ch_observables:eq:P}-\ref{ch_observables:eq:A}) for the definition of the currents)
\begin{align}
\label{ch_foundation:eq:WTI_V}
\partial_{\mu}V_{\mu}^{ij}&=(m_i-m_j)S^{ij}+i(\eta_i\mu_i-\eta_j\mu_j)P^{ij},\\
\label{ch_foundation:eq:WTI_A}
\partial_{\mu}A_{\mu}^{ij}&=(m_i+m_j)P^{ij}+i(\eta_i\mu_i+\eta_j\mu_j)S^{ij}.
\end{align}
Note that at zero twist angle $\eta_u=\eta_d=\eta_s=\eta_c=0$ the twisted and physical basis are the same, and the standard WTIs are recovered. However, at maximal twist the renormalized standard masses $m_i^{\textrm{R}}$ vanish, which in turn means that the current masses $m_{i}$ in eqs.~(\ref{ch_foundation:eq:WTI_V}-\ref{ch_foundation:eq:WTI_A}) also vanish (up to cutoff effects). Moreover, the exact flavor symmetry of massless Wilson fermions implies the existence of a point-split vector current $\tilde{V}_{\mu}^{ij}$ on the lattice such that the vector WTI holds exactly on the lattice. In the twisted basis, this current takes the form
\begin{equation}
\tilde{V}_{\mu}^{ij}=\frac{1}{2}\left[\bar{\psi}^i(x)(\gamma_{\mu}-1)U_{\mu}(x)\psi^j(x+\hat{\mu})+\bar{\psi}^i(x+\hat{\mu})(\gamma_{\mu}+1)U_{\mu}^{\dagger}(x)\psi^j(x)\right].
\end{equation}
The conservation of this WTI on the lattice for $\tilde{V}_{\mu}^{ij}$ means that the point-split vector current renormalizes trivially with
\begin{equation}
\label{ch_foundation:eq:ZV=1}
Z_{\tilde{V}}=1.
\end{equation}
Looking at eq.~(\ref{ch_foundation:eq:WTI_V}) this means that for all flavors
\begin{equation}
\label{ch_foundation:eq:Zmu}
Z_{\mu}=Z_P^{-1}.
\end{equation}


%%%%%%%%%%%%%%%%%%%%%%%%%%%%%%%%%%%%%%%%%%%%%%%%%%%%%%%%%%%
%%%%%%%%%%%%%%%%%%%%%%%%%%%%%%%%%%%%%%%%%%%%%%%%%%%%%%%%%%%

\section{Path integral regularization}
\label{ch_foundation:sec:path}

Having formulated the QCD action on the lattice, we need to see how physical quantities are computed. To do so, we review some aspects of the path integral formulation in Euclidean space-time. In this formalism, physical quantities are expressed as expectation values of operators
\begin{align}
\left<{O}(x_1,...,x_n)\right>&=\frac{1}{\mathcal{Z}}\int\mathcal{D}[\psi,\bar{\psi},U]{O}(x_1,...,x_n)e^{-S[\psi,\bar{\psi},U]}, \\
\mathcal{Z}&=\int\mathcal{D}[\psi,\bar{\psi},U]e^{-S[\psi,\bar{\psi},U]}.
\end{align}
This is equivalent to expectation values in statistical mechanics with a Boltzmann factor of $e^{-S[\psi,\bar{\psi},U]}$. The action can be decomposed into its gluon and fermion components $S[\psi,\bar{\psi},U]=S_{\textrm G}[U]+S_{\textrm F}[\psi,\bar{\psi},U]$, and fermion degrees of freedom can be integrated out as
\begin{align}
\left<{O}(x_1,...,x_n)\right>&=\frac{1}{\mathcal{Z}}\int\mathcal{D}[U]e^{-S_{\textrm G}[U]}\mathcal{Z}_{\textrm F}\notag \\
&\times\left[\frac{1}{\mathcal{Z}_{\textrm F}}\int\mathcal{D}[\psi,\bar{\psi}]{O}(x_1,...,x_n)e^{-S_{\textrm F}[\psi,\bar{\psi}]}\right] \\
&=\frac{1}{\mathcal{Z}}\int\mathcal{D}[U]e^{-S_{\textrm G}[U]}\mathcal{Z}_{\textrm F}\left<{O}(x_1,...,x_n)\right>_{\textrm F},
\end{align}
with 
\begin{align}
\mathcal{Z}_{\textrm F}&=\int\mathcal{D}[\psi,\bar{\psi}]e^{-S_{\textrm F}[\psi,\bar{\psi}]}=\Pi_{i=1}^{N_f}{\textrm{det}}\left(D\right).
\end{align}
This fermionic determinant can be expressed as an effective action as
\begin{align}
\left<{O}(x_1,...,x_n)\right>&=\frac{1}{\mathcal{Z}}\int\mathcal{D}[U]e^{-S_{\textrm G}[U]-S_{\textrm{eff}}[U]}\left<{O}(x_1,...,x_n)\right>_{\textrm F}, \\
\mathcal{Z}&=\int\mathcal{D}[U]e^{-S_{\textrm G}[U]-S_{\textrm{eff}}[U]}, \\
S_{\textrm{eff}}[U]&=-\sum_{i=1}^{N_f}{\textrm{log}}\;{\textrm{det}}\left(D\right).
\end{align}

In order to compute meson observables we will use meson interpolators, which are composite fermionic observables that share the same quantum numbers as the desired meson state. A generic meson interpolator has the form
\begin{equation}
{O}_A^{ij}(x)=\bar{\psi}^i(x)\Gamma_A\psi^j(x),
\end{equation}	
with $\Gamma_A$ a Gamma matrix or product of matrices. This way, a meson two-point function reads
\begin{align}
\label{ch_foundation:eq:path_int}
\left<{O}_A^{ij}(x_1){O}_B^{ji}(x_2)\right>&=\frac{1}{\mathcal{Z}}\int\mathcal{D}[U]e^{-S_{\textrm G}[U]-S_{\textrm{eff}}[U]}\notag\\
&\times\left<\bar{\psi}^i(x_1)\Gamma_A\psi^j(x_1)\psi^i(x_2)\Gamma_B\bar{\psi}^j(x_2)\right>_{\textrm F}  \\
&=-\frac{1}{\mathcal{Z}}\int\mathcal{D}[U]e^{-S_{\textrm G}[U]-S_{\textrm{eff}}[U]}\notag\\
&\times{\textrm{tr}}\left(\Gamma_AD_i^{-1}(x_1,x_2)\Gamma_BD_j^{-1}(x_2,x_1)\right),
\end{align}
where the trace is over spin indices and $D_i$ the massive Dirac operator for flavor $i$. In order to perform this integral numerically, using the connection with statistical mechanics, a finite set of $N_{\textrm{cnfg}}$ gauge configurations is generated with Boltzmann distribution $e^{-S_{\textrm G}[U]-S_{\textrm{eff}}[U]}$ following a Markov process (see Appendices~\ref{appex_simulations},~\ref{appex_errors}). Then, measurements of the quantity
\begin{equation}
P=-{\textrm{tr}}\left(\Gamma D_i^{-1}(x_1,x_2)\Gamma D_j^{-1}(x_2,x_1)\right),
\end{equation}
are taken in each of these configurations, and the expectation value is computed as
\begin{equation}
\left<P\right>=\frac{1}{N_{\textrm{cnfg}}}\sum_{i}^{N_{\textrm{cnfg}}}P_i+\mathcal{O}\left(\frac{1}{\sqrt{N_{\textrm{cnfg}}}}\right).
\end{equation}


%%%%%%%%%%%%%%%%%%%%%%%%%%%%%%%%%%%%%%%%%%%%%%%%%%%%%%%%%%%
%%%%%%%%%%%%%%%%%%%%%%%%%%%%%%%%%%%%%%%%%%%%%%%%%%%%%%%%%%%

\section{Continuum limit}
\label{ch_foundation:sec:continuum-limit}

For the discussion in this subsection we follow~\citep{Hernandez:2009zz}. The lattice regularization provides with a natural energy cutoff $a^{-1}$, ensuring that any loop integral is finite in perturbation theory. In perturbative renormalization, it is necessary to take the cutoff to infinity, which on the lattice means taking the lattice spacing to $a\rightarrow0$. If the theory is renormalizable, any physical quantity (e.g. a mass $m_{\textrm{phys}}$) in units of the lattice spacing must vanish in the continuum limit
\begin{equation}
\label{ch_foundation:eq:ma0}
m_{\textrm{phys}}a\rightarrow0,
\end{equation}
since this means that $m_{\textrm{phys}}$ remains finite in this limit. 

Physical quantities are dependent on the couplings of the theory, $m_{\textrm{phys}}(g_0)$, and accordingly change with them. In turn, one can study how the couplings of the theory change on the lattice as one approaches the continuum limit by decreasing $a$. To do so and for simplicity, we assume a single coupling $g_0$, and write the most general local effective action at lattice spacing $a_1$
\begin{equation}
S(a_1)=g_0(a_1)\sum_i{O}_i,
\end{equation}
where ${O}_i$ are all possible local operators respecting the lattice symmetries. At a finer lattice spacing $a_2<a_1$ all the short-range extra degrees of freedom can be integrated out and reabsorbed into a redefinition of the coupling, obtaining an effective action at the original scale $a_1$, $S^{(2)}(a_1)$, that has the same generic form but with different couplings
\begin{align}
S^{(2)}(a_1)&=g_0^{(2)}(a_1)\sum_iO_i, \\
g_0^{(2)}(a_1)&=R(g_0(a_1)).
\end{align}
$R$ here stands for the renormalization group (RG) transformation that defines the change in the couplings when varying the lattice spacing. It can be observed then that renormalizability corresponds to fixed points $g_0^*$ of the RG transformation
\begin{equation}
R(g_0^*)=g_0^*.
\end{equation}

In the context of $SU(N)$ Yang-Mills theory, perturbation theory shows that at a fixed value of the renormalized coupling $g_R$ the bare coupling runs with the lattice spacing as
\begin{equation}
\label{ch_foundation:eq:beta-func}
a\frac{\partial g_0}{\partial a}\equiv\beta(g_0)=-\beta_0g_0^3-\beta_1g_0^5+...,
\end{equation}
where $\beta_{0,1}$ are universal coefficients (do not depend on the renormalization scheme) and positive for $N=3$ colors and $N_f=6$ flavors, as in the case of QCD. This shows that $g_0=0$ is a fixed point of the RG transformations and thus corresponds to the continuum limit. As the fixed point is in the weak coupling regime, this perturbative argument is expected to be valid. Therefore, the continuum limit corresponds to
\begin{equation}
g_0\rightarrow0,
\end{equation}
or in terms of the inverse coupling $\beta$
\begin{equation}
\beta\rightarrow\infty.
\end{equation}

In practice, one cannot numerically simulate at infinite inverse coupling $\beta$. Therefore, physical observables are computed at several finite values of $\beta$. This introduces $\mathcal{O}(a^n)$ cutoff effects in the results, with some power $n$. To obtain results in the continuum, one parameterizes these cutoff effects with some function of the lattice spacing and extrapolates to $a\rightarrow 0$. However, this task is far from trivial, and it has been shown that spectral quantities receive logarithmic corrections on the lattice spacing~\citep{Husung:2022kvi} which could significantly complicate this task. To help in the continuum limit extrapolation, one can systematically reduce lattice artifacts, e.g. from $\mathcal{O}(a)$ to $\mathcal{O}(a^2)$ following the Symanzik improvement program. 

%%%%%%%%%%%%%%%%%%%%%%%%%%%%%%%%%%%%%%%%%%%%%%%%%%%%%%%%%%%
%%%%%%%%%%%%%%%%%%%%%%%%%%%%%%%%%%%%%%%%%%%%%%%%%%%%%%%%%%%

\section{Symanzik improvement program}
\label{ch_foundation:sec:impr}

Symanzik improvement requires improving both the action of the theory and the lattice interpolators that enter the different correlators. 

In order to improve a lattice action, one can describe the target continuum theory in terms of an effective action in powers of the lattice spacing $a$
\begin{equation}
S_{\textrm{eff}}=\int d^4x\sum_kc_k\mathcal{L}_k(x)a^{k-4}.
\end{equation}
Here $\mathcal{L}_0(x)$ is the discretized lattice Lagrangian unimproved, the higher-dimension terms $\mathcal{L}_k$(x) are all possible Lagrangians built from fermion and gluon field operators that preserve the symmetries of the regularized theory, i.e. the lattice theory, with mass dimension $4+k$, and $c_k$ are numerical coefficients.

In the case of Lattice QCD, we saw that in the Wilson gauge action in eq.~(\ref{ch_foundation:eq:YM-latt}) lattice artifacts appear at $\mathcal{O}(a^2)$, and therefore no $\mathcal{O}(a)$ improvement is required. However, these $\mathcal{O}(a^2)$ effects can be further reduced by adding all possible dimension $4+k=6$ operators that preserve the underlying symmetries of the gauge action. These dimension-6 operators are all three possible ways of writing a closed path in a rectangular lattice with 6 gauge links: planar, twisted and L-shaped rectangles. The action then reads
\begin{equation}
\label{ch_foundation:eq:SG_impr}
S_G=\frac{\beta}{3}\sum_{\mu\nu}\left[c_0\sum_p{\textrm{Re}}\left({\textrm{tr}}\left(1-U_{\mu\nu}(p)\right)\right)+\sum_{i=1}^3c_i\sum_r{\textrm{Re}}\left({\textrm{tr}}\left(1-U^{(i)}(r)\right)\right)\right],
\end{equation}
with $U^{(i)}$ said dimension-6 operators. Tuning the coefficients $c_i$ properly leads to $\mathcal{O}(a^2)$ improvement. The CLS ensembles that we employ in this thesis (see Sec.~\ref{ch_ma:sec:Sea}) use the so called Lüscher-Weisz gauge action~\citep{Luscher:1984xn,Luscher:1985zq}, with these coefficients computed at tree-level
\begin{gather}
\label{ch_foundation:eq:LW}
c_0=\frac{5}{3}, \quad
c_1=-\frac{1}{12}, \quad
c_2=c_3=0.
\end{gather}
Thus, in the Lüscher-Weisz gauge action the only dimension-6 operators that survive are planar rectangles $U^{(1)}$.

We also need to improve the fermion action. Wilson fermions have $\mathcal{O}(a)$ cutoff effects. In order to improve the Wilson fermion action to $\mathcal{O}(a^2)$ we need to look for all possible operators with dimension $4+k=5$ that preserve the lattice symmetries. These are
\begin{gather}
\label{ch_foundation:eq:L1}
\mathcal{L}_{k=1}^{(1)}=i\bar{\psi}(x)\sigma_{\mu\nu}\hat{F}_{\mu\nu}(x)\psi(x),\\
\mathcal{L}_{k=1}^{(2)}=\boldsymbol{m}{\textrm{tr}}\left(\hat{F}_{\mu\nu}(x)\hat{F}_{\mu\nu}(x)\right),\\
\mathcal{L}_{k=1}^{(3)}=\boldsymbol{m}^2\bar{\psi}(x)\psi(x),
\end{gather}
with
\begin{align}
\label{ch_foundation:eq:dim5-op}
\sigma_{\mu\nu}&=\frac{\left[\gamma_{\mu},\gamma_{\nu}\right]}{2i},\\
\hat{F}_{\mu\nu}(x)&=\frac{-i}{8a^2}\left(Q_{\mu\nu}(x)-Q_{\nu\mu}(x)\right),\\
Q_{\mu\nu}&=U_{\mu\nu}(x)+U_{\nu,-\mu}(x)+U_{-\mu,-\nu}(x)+U_{-\nu,\mu}(x).
\end{align}
$\mathcal{L}_{k=1}^{(2),(3)}$ are already present (up to numerical factors) in the original Wilson fermion action and can therefore be reabsorbed in those terms. The $\mathcal{O}(a)$ improved Wilson Dirac operator appearing in the improved fermion action reads
\begin{equation}
\label{ch_foundation:eq:DW_impr}
D_{\textrm W}+\boldsymbol{m}+c_{\textrm{sw}}a\frac{1}{2}\sum_{\mu<\nu}\sigma_{\mu\nu}\hat{F}_{\mu\nu},
\end{equation}
with $c_{\textrm{sw}}$ the Sheikholeslami-Wohlert coefficient determined non perturbatively in~\citep{Sheikholeslami:1985ij}.

Improving the lattice action ensures improvement of on-shell quantities such as meson masses. However, if one is interested in matrix elements mediated by some current $\mathcal{J}_{\mu}$, it is also necessary to improve the lattice interpolators that enter into the definition of those currents. In analogy with the improvement of the action, a local operator $O$ is expressed in the Symanzik effective theory as
\begin{equation}
O_{\textrm{eff}}(x)=\sum_kc_kO_k(x)a^k.
\end{equation}
Again, $O_k$ are gauge invariant local operators with the right mass dimensions and $c_k$ some parameter properly tuned to cancel $a^k$ cutoff effects. Following this, a generic n-point function reads
\begin{equation}
\label{ch_foundation:eq:Oimpr}
\left<\Phi\right>=\left<\Phi_0\right>-a\int d^4y\left<\Phi_0\mathcal{L}_1(y)\right>+a\left<\Phi_1\right>+...,
\end{equation}
with 
\begin{align}
\left<\Phi_0\right>&=\left<O_0(x_1)...O_0(x_n)\right>, \\
\left<\Phi_1\right>&=\sum_{i=1}^n\left<O_0(x_1)...O_1(x_i)...O_0(x_n)\right>,
\end{align}
and vacuum expectation values taken in the continuum. In Sec~\ref{ch_observables} we discuss the details of operator improvement for the observables of interest.

The $\mathcal{O}(a)$ improved Wilson tm fermion action is analogous to the Wilson case, with the improved Dirac operator given by 
\begin{equation}
D_{\textrm W}+\boldsymbol{m}+i\gamma_5\boldsymbol{\mu}+c_{\textrm{sw}}a\frac{1}{2}\sum_{\mu<\nu}\sigma_{\mu\nu}\hat{F}_{\mu\nu}.
\end{equation}
The advantage of Wilson tm fermions is that at maximal twist (vanishing renormalized standard quark mass) one achieves automatic $\mathcal{O}(a)$ improvement~\citep{Frezzotti:2003ni,Shindler:2007vp}. This means that physical quantities are automatically improved without the need of any improvement coefficients for lattice operators. The following argument is based on the original work~\citep{Frezzotti:2003ni} to which we refer for a complete proof.

At maximal twist, the Wilson tm Dirac operator reads
\begin{equation}
D_{\textrm W}+i\boldsymbol{\mu}\gamma_5.
\end{equation}
Working in the twisted basis, this action in the continuum is invariant under a discrete chiral symmetry 
\begin{equation}
\mathcal{R}_5^{1,2}=\left\{\begin{matrix}
\psi(x)\rightarrow i\gamma_5\tau^{1,2}\psi(x) \\ 
\bar{\psi}(x)\rightarrow \bar{\psi}(x)i\gamma_5\tau^{1,2}
\end{matrix}\right.,
\end{equation}
while $\mathcal{L}_{k=1}^{(1)}$ in eq.~(\ref{ch_foundation:eq:L1}) is not 
\begin{equation}
\mathcal{L}_{k=1}^{(1)}\rightarrow-\mathcal{L}_{k=1}^{(1)}.
\end{equation}
This is key for automatic $\mathcal{O}(a)$ improvement. For correlation functions like eq.~(\ref{ch_foundation:eq:Oimpr}), we have that operators may be even or odd under $\mathcal{R}_5$, $\left<\Phi_0\right>$ and $\left<\Phi_1\right>$ having opposite $\mathcal{R}_5$-chirality
\begin{gather}
\left<\Phi_0\right>\rightarrow\pm\left<\Phi_0\right>, \quad \left<\Phi_1\right>\rightarrow\mp\left<\Phi_1\right>.
\end{gather}
This means that for even $\left<\Phi_0\right>$
\begin{gather}
\left<\Phi_0\right>=\left<\Phi_0\right>,
\quad \left<\Phi_0\mathcal{L}^{(1)}_{k=1}\right>=-\left<\Phi_0\mathcal{L}^{(1)}_{k=1}\right>=0, \quad \\ \left<\Phi_1\right>=-\left<\Phi_1\right>=0,
\end{gather}
and thus even operators are automatically $\mathcal{O}(a)$ improved. On the other hand, for odd operators what we have is
\begin{gather}
\left<\Phi_0\right>=-\left<\Phi_0\right>=0, \quad
\left<\Phi_0\mathcal{L}^{(1)}_{k=1}\right>=\left<\Phi_0\mathcal{L}^{(1)}_{k=1}\right>, \quad \\ \left<\Phi_1\right>=\left<\Phi_1\right>,
\end{gather}
and thus they vanish in the continuum. Summing up, the only tuning required for Wilson tm fermions to achieve $\mathcal{O}(a)$ improvement is to set the bare quark mass $m$ to its critical value $m_{\textrm{cr}}$ in order to obtain maximal twist.

In our particular case, we will be working with a mixed action setup employing standard Wilson quarks in the sea and fully twisted Wilson tm quarks in the valence (see Sec~\ref{ch_ma}). This means valence observables still get residual $\mathcal{O}(a)$ cutoff effects from the sea sector, and thus improvement is still needed. However, these effects are expected to be $\mathcal{O}(g_0^4)$ in perturbation theory.

Finally, we also need to improve the bare gauge coupling, which at $\mathcal{O}(a)$ reads
\begin{equation}
\tilde{g}_0^2=g_0^2\left(1+ab_g{\textrm{tr}}\left(M_q^{\textrm{(s)}}\right)\right),
\end{equation}
with $M_q^{\textrm{(s)}}$ the sea quark mass matrix (see Sec.~\ref{ch_ma}), and $b_g$ the improvement coefficient, whose value at one-loop is given in~\citep{Luscher:1996sc}.

%%%%%%%%%%%%%%%%%%%%%%%%%%%%%%%%%%%%%%%%%%%%%%%%%%%%%%%%%%%
%%%%%%%%%%%%%%%%%%%%%%%%%%%%%%%%%%%%%%%%%%%%%%%%%%%%%%%%%%%

\section{Scale setting}
\label{ch_foundation:sec:ss}

on the lattice, all physical observables are computed in units of the lattice spacing $a$. Consequently, in order to make any prediction, it is necessary to determine $a$ in physical units. This task is called scale setting. It involves the precise determination of a reference observable, called the scale, in physical units, to which any other observable is compared to in order to extract the value of the latter in physical units. As mentioned in the introduction, in ``precision era'' lattice calculations, high precision scale setting is of the utmost importance in order to extract predictions whose uncertainty is not dominated by the scale. 

As an example of the scale setting procedure, we could use the proton mass $m_{\textrm{proton}}$ as a reference scale, and calculate the ratio of it to a given mass $m_i$
\begin{equation}
R_i=\frac{m_i}{m_{\textrm{proton}}}.
\end{equation}
After computing the continuum limit of $R_i$, we can extract the physical mass $m_i$ as
\begin{equation}
\label{ch_ss:eq:R}
m_i^{\textrm{ph}}=R_i(a=0)\times m_{\textrm{proton}}^{\textrm{exp}}.
\end{equation}
Here, the proton mass is used as a reference scale, and comparing any lattice observable to it allows to extract the latter in physical units, once the continuum limit is performed. This procedure is equivalent to finding the value of the lattice spacing in physical units, since it can be extracted as 
\begin{equation}
a=\frac{(am_{\textrm{proton}})^{\textrm{latt}}}{m_{\textrm{proton}}^{\textrm{exp}}}.
\end{equation}
From eq.~(\ref{ch_ss:eq:R}) it is clear that when aiming for precise lattice calculations of any physical observable like $m_i$, a reliable and precise scale setting is of the utmost importance. In this example this means being able to determine $m_{\textrm{proton}}$ with high accuracy on the lattice in order to compute the ratios $R_i$, controlling the continuum limit of $R_i$ and having a high precision determination of $m_{\textrm{proton}}^{\textrm{exp}}$.

In this context, baryon masses like the proton or the $\Omega$ baryon mass are popular choices to set the scale~\citep{RQCD_scale}. The former is determined with high accuracy experimentally~\citep{ParticleDataGroup:2020ssz} but suffers from the signal-to-noise problem~\citep{Lepage:1989hd,Luscher:2010ae} on the lattice determination. This problem is also present in the $\Omega$ baryon mass, but the statistical precision is better there~\citep{RQCD_scale}. Furthermore, the $\Omega$ baryon mass has a weak dependence on the light quark masses and a strong one in the strange quark mass. This makes it an interesting scale for trajectories with constant strange quark mass. Another choice is using meson masses. The pion and kaon meson masses are used to define the line of constant physics along which the continuum limit is taken, and therefore are not available to set the scale. In the past, the $\rho$ meson mass was used to set the scale of quenched simulations~\citep{Mawhinney:1996jk,Irving:1998yu,Bornyakov:2015plz}, but it is not suited for dynamical quarks simulations. The $\Upsilon$ meson mass is also used~\citep{HPQCD:2011qw,Gray:2005ur} thanks to its precise experimental determination. However, large discretization effects due to the $b$ quark are expected. 

Instead of using a phenomenological scale like the ones listed above, another choice is to use intermediate scales, like the gradient flow scale $t_0$~\citep{Luscher:2010we,1006.4518} this thesis is based on and that we introduce in Sec.~\ref{ch_observables:sec:Flow}. This quantity is a popular choice~\citep{Bruno:2016plf,Strassberger:2023xnj,RQCD_scale,Kostrzewa:2021syw,Hollwieser:2020qri,MILC:2015tqx} since it can be computed to a very high precision on the lattice, though it is not a physical quantity and as such cannot be measured experimentally. To obtain its value in the continuum and physical quark masses, one builds a dimensionless quantity $(\sqrt{t_0}\Lambda)^{\textrm{latt}}$ with some phenomenological quantity $\Lambda$ on the lattice. After performing the continuum limit, the physical value of $t_0$ can be extracted as
\begin{equation}
\label{ch_foundation:eq:Lambda}
\sqrt{t_0^{\textrm{ph}}}=\frac{\left.\begin{matrix}
\left(\sqrt{t_0}\Lambda\right)^{\textrm{latt}}
\end{matrix}\right|_{a=0}}{\Lambda^{\textrm{exp}}}.
\end{equation}
In addition to the continuum limit, on the lattice often unphysical quark masses are simulated since they are computationally cheaper. This means one needs to perform chiral extrapolations/interpolations of lattice observables to reach physical quark masses. Both chiral and continuum limits are discussed in Sec.~\ref{ch_ss} for the scale setting we perform in this thesis.

Once the physical value of $t_0$ is found, it can be used as an intermediate scale against which any other quantity $\Lambda'$ on the lattice can be compared in order to extract the latter in physical units. For this purpose, one performs a continuum extrapolation of $\sqrt{t_0}\Lambda'$ and obtains the physical value of $\Lambda'$ as
\begin{equation}
\Lambda^{\textrm{' ph}}=\frac{\left.\begin{matrix}
\left(\sqrt{t_0}\Lambda'\right)^{\textrm{latt}}
\end{matrix}\right|_{a=0}}{\sqrt{t_0^{\textrm{ph}}}}.
\end{equation}
This quantity is already a prediction of the lattice.

A popular choice~\citep{Brown:2018jtv,BMW:2012hcm,BMW:2012hcm,Bruno:2016plf,Strassberger:2023xnj} for $\Lambda$ in eq.~(\ref{ch_foundation:eq:Lambda}) and the one used in this work is a linear combination of the pion and kaon decay constants. These exhibit large plateaux on the lattice, indicating that excited states contributions decay fast and therefore they can be determined to a high precision on the lattice. On the other hand, their experimental values are extracted from the weak processes $\pi/K\to l\nu$, which leads to the measurement of $V_{ud(us)}f_{\pi(K)}$, with $V_{ud,us}$ CKM matrix elements. This leads to an increase in the uncertainty of the experimental values of $f_{\pi,K}$ coming from the determination of said CKM matrix elements~\citep{FlavourLatticeAveragingGroupFLAG:2021npn}.

Finally, other popular intermediate scales to $t_0$ are $\omega_0$~\citep{BMW:2012hcm,Kostrzewa:2021syw,MILC:2015tqx} which is closely related to $t_0$, and the force scale $r_0$~\citep{Sommer:1993ce,Necco:2001xg,Bernard:2000gd} which is derived from the static quark-antiquark potential extracted from the evaluation of Wilson loops. This potential shows early plateaux~\citep{Sommer:2014mea} which again indicates that excited states contributions are small.

%%%%%%%%%%%%%%%%%%%%%%%%%%%%%%%%%%%%%%%%%%%%%%%%%%%%%%%%%%%
%%%%%%%%%%%%%%%%%%%%%%%%%%%%%%%%%%%%%%%%%%%%%%%%%%%%%%%%%%%
