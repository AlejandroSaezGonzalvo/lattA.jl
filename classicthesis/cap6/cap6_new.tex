\chapter{Impact of the scale setting in Lattice QCD computations}
\label{ch_charm}

%%%%%%%%%%%%%%%%%%%%%%%%%%%%%%%%%%%%%%%%%%%%%%%%%%%%%%%%%%%
%%%%%%%%%%%%%%%%%%%%%%%%%%%%%%%%%%%%%%%%%%%%%%%%%%%%%%%%%%%
%%%%%%%%%%%%%%%%%%%%%%%%%%%%%%%%%%%%%%%%%%%%%%%%%%%%%%%%%%%
%%%%%%%%%%%%%%%%%%%%%%%%%%%%%%%%%%%%%%%%%%%%%%%%%%%%%%%%%%%

%\section{Introduction}
%\label{ch_qm:sec:introduction}

In this Chapter we will discuss the role of the determination of $t_0$, described in Chapter \ref{ch_ss}, in lattice QCD calculations of other observables. In particular, we will see that the precision of the result quoted in eq.~(\ref{ch_ss:eq:t0ph_c}) leads to determinations of the renormalized charm quark mass and $D_{(s)}$ charmed mesons decay constants for which the scale $t_0$ is not the dominant source of uncertainty. 

For the extraction of charmed observables we rely entirely on the mixed-action approach with Wilson twisted mass fermions at maximal twist, as reported in Sec.~\ref{ch_ma:sec:matching}, exploiting the absence of leading lattice artifacts of $\mathcal{O}(a\mu_c)$ that would otherwise play a dominant role at the scale of the charm quark mass $\mu_c$. This provides a way to approach the continuum limit of charmed observables complementary to that based on Wilson fermions that require explicit inclusion of Symanzik improvement counterterms.

In Sec. \ref{sec:matching_charm} we discuss the details of our strategy to match the charm quark mass to its physical value. In Sec.~\ref{sec:mc} we discuss chiral-continuum extrapolations of the renormalized charm quark mass and present our results for this quantity at the physical point after performing a model average over the set of considered functional forms. In Sec.~\ref{sec:fDs} we summarize our results for the charmed mesons $D_{(s)}$ decay constants, showing the contribution to the final uncertainty coming from the determination of the scale $t_0$. For a complete discussion of these results we refer to~\citep{charm}. 

In addition to these charmed mesons computations, in Appendix~\ref{apex_light_qm} we report about the status of an analysis of the light and strange quark masses. 


\section{Matching of the charm quark mass}
\label{sec:matching_charm}

In Sec.~\ref{ch_ma} we performed the matching of the sea and valence sectors of our mixed action for the light and strange quark flavors, in addition to tuning to maximal twist. Once the valence parameters were determined to ensure these conditions, an independent set of computations of heavy propagators was performed for the study of charm physics. Heavy propagators are computed at three different values of the twisted  mass $\mu_c^{(i)}$ around the physical charm region for most of the considered ensembles, while for a subset of them two masses have been used, so that in all cases observables are interpolated at the physical value of the charm quark mass. In order to fix the charm quark mass to its physical value, we use different combinations of mesons masses $m_H$ matched to their physical values. Since the charm quark is partially quenched, this matching procedure involves observables with only charm quarks in the valence sector. 
%

We study two different charm quark matching conditions based on two choices of $m_H^{(i)},~i=1,2$, and will often be expressed in units of $\sqrt{8t_0}$ as $\phi_H^{(i)} = \sqrt{8t_0}m_H^{(i)}$.
%

The first possibility we explore, corresponding to $\phi_H^{(1)}$, consists in using the flavor average meson mass combination
\begin{equation}
        m_H^{(1)} = m_{\overline{H}} \equiv \frac{2}{3} m_H + \frac{1}{3}m_{H_s},
        \label{eq:fl_av_matching}
\end{equation}
built from heavy-light $H$ and heavy-strange $H_s$ pseudoscalar meson masses with heavy-quark masses in the neighborhood of the charm.
%
Since we mass shifted\footnote{In the case of the charmed observables considered in this Chapter, the mass shift was performed in a similar manner to that discussed in Sec. \ref{ch_ma:sec:chiral_traj}, but this time using the dedicated measurements of the mass derivatives for each ensemble, instead of parametrizing them as a function of $\phi_2$ and of the lattice spacing.} the considered CLS ensembles in order to impose a constant value of $\phi_4$ (see eq.~(\ref{ch_ma:eq:phi4})), we expect the flavor average combination $\phi_H^{(1)}$ to remain fairly constant along the chiral trajectory. The physical value of $m_H^{(1), \mathrm{ph}}$ is obtained by setting $m_{H_{(s)}}$ to the following prescription for the isoQCD values of $D_{(s)}$ meson masses,
%
\begin{equation}
    m_D^{\mathrm{isoQCD}} = 1867.1(2.6)  \ \mathrm{MeV}, \qquad m_{D_s}^{\mathrm{isoQCD}} = 1967.1(1.3) \ \mathrm{MeV}.
\label{eq:DsisoQCDinputs}
\end{equation}
%
The uncertainties in these isoQCD values are chosen to cover the deviation with respect to the experimental values~\cite{ParticleDataGroup:2022pth} of the $D^{\pm}$ and $D_s^{\pm}$ meson masses, $m_{D^\pm}^{\mathrm{exp}} = 1869.66(5) \ \mathrm{MeV}$ and $m_{D_s^\pm}^{\mathrm{exp}} = 1968.35(7) \ \mathrm{MeV}$, respectively. We observe that the larger uncertainty in the isoQCD inputs of the $D$ and $D_s$ meson masses in eq.~(\ref{eq:DsisoQCDinputs}) --- as compared to the corresponding experimental values --- does not induce a significant increase in the uncertainties of our target results. The input values in eq.~(\ref{eq:DsisoQCDinputs}) lead to the following flavor averaged meson mass,
%
\begin{equation}
         m_H^{(1), \mathrm{ph}} = m_{\overline{D}} = 1900.4(1.8) \ \mathrm{MeV}\,.
\end{equation}
% 

The second strategy, corresponding to $\phi_H^{(2)}$, is to consider the mass-degenerate pseudoscalar meson mass $m_{\eta_h}^{\mathrm{conn}}$ extracted from the quark-connected two-point correlation function made of heavy quark propagators with a mass in the neighborhood of the charm mass,
%
\begin{equation}
  m_H^{(2)} = m_{\eta_h}^{\mathrm{conn}}\,.
\label{eq:etac_matching}
\end{equation}
%
The physical value for this mass, $m_H^{(2), \mathrm{ph}}$,  is set from the experimental value of the $\eta_c$ meson mass~\cite{ParticleDataGroup:2022pth}, $m_{\eta_c}^{\mathrm{exp}} = 2983.9(4)\,$ MeV, from which a correction of about 6 MeV, with 100\% error, is subtracted to account for the absence of quark-disconnected diagrams and QED effects~\cite{deForcrand:2004ia, Donald:2012ga,Colquhoun:2015oha,Hatton:2020qhk,Colquhoun:2023zbc}. Specifically, we employ, 
%
\begin{equation}
  m_H^{(2), \mathrm{ph}} = m_{\eta_c}^{\mathrm{conn}} = 2978(6) \ \mathrm{MeV}\,.
\end{equation}
%
One potential advantage of this choice of matching observable is that the statistical precision of the $\eta_c^{\mathrm{conn}}$ meson mass is substantially better than the one for heavy-light meson masses, as it does not suffer from the increase in noise-to-signal ratio with Euclidean time.
%

%
Any of these matching conditions can in principle be imposed ensemble by ensemble, even away from the physical point.
%
However, by doing so we would as a result build in the charm quark mass a dependence on the value of the reference scale $t_0^{\mathrm{ph}}$, as well as $\mathcal{O}(a^2)$ effects coming from the specific choice of $m_H$.
%
To avoid this, we have opted instead for setting the physical charm quark mass jointly with the chiral-continuum extrapolation, in a similar way as the one we employed to reach the physical point in the light and strange sector.
%
What this means in practice is that the charm quark mass dependence of any given observable is parameterized as $\mathcal{O}(a, \phi_2, \phi_H^{(i)})$, and we perform a global fit to obtain its physical value $\mathcal{O}(0, \phi_2^{\mathrm{ph}}, \phi_H^{(i),\mathrm{ph}})$.
%
This will be the procedure applied below in the determination of the physical value of the charm quark mass and of the decay constants $f_D$ and $f_{D_s}$.
%

%%%%%%%%%%%%%%%%%%%%%%%%%%%%%%%%%%%%%%%%%%%%%%%%%%%%%%%%%%%
%%%%%%%%%%%%%%%%%%%%%%%%%%%%%%%%%%%%%%%%%%%%%%%%%%%%%%%%%%%
%%%%%%%%%%%%%%%%%%%%%%%%%%%%%%%%%%%%%%%%%%%%%%%%%%%%%%%%%%%
%%%%%%%%%%%%%%%%%%%%%%%%%%%%%%%%%%%%%%%%%%%%%%%%%%%%%%%%%%%

\section{Determination of the charm quark mass}
\label{sec:mc}

\subsection{Renormalized charm quark masses}

%
As discussed in Sec.~\ref{ch_foundation:subsec:tm}, in the Wilson tm regularization, renormalized quark masses can be retrieved from bare Lagrangian twisted masses through a multiplicative renormalization. In our mixed-action setup, due to residual effects coming from the sea, the resulting $\mathcal{O}(a)$ improved expression for the renormalized  charm mass $\mu^{\textrm{R}}_c$ reads
\begin{equation}
	\mu^{\textrm{R}}_c=Z_P^{-1}(g_0^2,a\mu_{\textrm{ren}})\left[1+a\overline{b}_\mu\textrm{tr}\left(M_q^{(s)}\right)\right]\mu_c\,,
	\label{eq:renormalized_charm_mass}
\end{equation}
where $Z_P$ is the renormalization constant for the non-singlet
pseudoscalar density at some renormalization scale $\mu_{\textrm{ren}}$ as discussed in Sec.~\ref{ch_foundation:subsec:tm}.
%
The term depending on the improvement coefficient  will be neglected since it is expected to induce a small correction as it is a sea quark mass effect such that $\overline{b}_{\mu} = \mathcal{O}(g_0^4)$ in perturbation theory and, moreover, the sea quark mass matrix $\textrm{tr}\left(M_q^{(s)}\right)$ depends only on the  relatively light (u,d,s) quark masses. Thus, renormalized quark masses can be obtained by simply applying the renormalization constants $Z_P$ to the twisted masses $\mu_i$ in the Lagrangian.
%

The values of $Z_P$ are listed in Table~\ref{ch_observables:tab:Z} and were computed at a fixed renormalization scale $\mu_{\textrm{had}}=233(8)$ MeV in the Schrödinger functional renormalization scheme~\citep{Campos:2018ahf}. They allow to obtain the renormalized quark masses on each of the ensembles considered in the chiral continuum extrapolation used to determine the physical value of the charm quark mass.
%
The conversion into the renormalization group invariant (RGI) quark mass $M_c^{\mathrm{RGI}}$ is performed by means of the continuum (flavor-independent) ratio also computed non-perturbatively in~\cite{Campos:2018ahf}
\begin{equation}
	\frac{M}{\overline{m}(\mu_{\mathrm{had}})} = 0.9148(88)\,.
	\label{eq:rgi_running_factor}
\end{equation}
%
The renormalized quark masses in other renormalization schemes -- such as the $\overline{\textrm{MS}}$ scheme -- are obtained by a perturbative running from the RGI mass down to the desired renormalization scale $\mu_{\textrm{ren}}$.
%

%%%%

\subsection{Charm quark mass chiral-continuum fits}
\label{subsec:mc_chiral_continuum}

Having determined the  renormalized charm quark masses in the Schr\"odinger Functional scheme at the hadronic renormalization scale $\mu_{\mathrm{had}}$, $\mu_c^{\textrm{R}}$, for all the ensembles listed in Table \ref{apex_ensembles:tab:ens}, we can perform the chiral-continuum fits to obtain results in the continuum limit and at the physical point. The matching procedure of the light
and strange sectors is already devised so that the physical value of the kaon mass is recovered
at $\phi_2 = \phi_2^{\mathrm{ph}}$, where the physical value of $\phi_2$ is computed
with the isoQCD values of the pion mass quoted 
in~\cite{FlavourLatticeAveragingGroupFLAG:2021npn} (see eqs.~(\ref{ch_ss:eq:isoQCD})), and the physical scale $t_0^{\mathrm{ph}}$
is the one determined in eq.~(\ref{ch_ss:eq:t0ph_c}). The charm scale is matched through the two different
prescriptions described in Sec.~\ref{sec:matching_charm}. All quantities entering the fit
are made dimensionless through the appropriate power of the factor $\sqrt{8t_0}$,
and physical units for the final result are restored by using our value for $t_0^{\mathrm{ph}}$.

We parameterize the continuum dependence of the renormalized charm quark mass on $\phi_2$
and any of the $\phi_H^{(i)}$ with the functional form
\begin{equation}
	\sqrt{8t_0}\, \mu_c^{\textrm{R}}(a=0, \phi_2, \phi_H) = p_0 + p_1\phi_2 + p_2\phi_H\,.
	\label{eq:mc_continuum_parameterization}
\end{equation}
Based on the heavy quark effective theory expansion~\cite{Georgi:1990um} at lowest order,
we expect a linear dependence of the charmed meson masses as a function of the charm quark 
mass, hence the latter term in the ansatz. This assumption is supported by our data that show indeed a 
linear behavior in the charmed meson masses, as illustrated in Figure \ref{fig:mc_mh_dependence}. Note that this functional form is used to interpolate the dependence within a small interval around the physical value of the charm quark mass. When considering the pion dependence of the charm quark mass, we assume that the  leading order contributions exhibit a linear behavior in $\phi_2$. As illustrated in Fig.~\ref{fig:mc_pion_dependence}, we observe a mild light-quark mass dependence  which is well characterized by a  linear term in $phi_2$.

Regarding the lattice spacing dependence of the charm quark mass, we assume the leading cutoff effects to 
be $\mathcal{O}(a^2)$, as discussed above. Higher order lattice artifacts are explored by including terms of $\mathcal{O}(a^4)$, as expected when employing twisted mass fermions at maximal twist. The impact of lattice artifacts of $\mathcal{O}(a^3)$ arising from the sea sector and/or from the renormalization factors will be incorporated in a forthcoming version of the analysis. Finally, we allow for lattice artifacts  proportional to $m_{\pi}^2$ and to various powers of the charm mass. The generic ansatz to parameterize lattice spacing dependence thus take the following form
\begin{equation}
	c_{\mu_c}(a, \phi_2, \phi_H) = \frac{a^2}{8t_0} \big(
	c_1 + c_2\phi_2 + c_3 \phi_H^2
	\big)
	+
	\frac{a^4}{(8t_0)^2}\big(
	c_4 + c_5\phi_H^2 + c_6 \phi_H^4
	\big).
	\label{eq:lattice_spacing_dependence}
\end{equation} 

In order to estimate the systematic effects arising from the model variation, we consider all the possible 
combinations where some of the $c_i$ coefficients vanish, save for $c_1$ which is always kept in the fits.
Furthermore, following~\cite{Heitger:2021apz}, we allow for cutoff effects to enter either linearly or 
non-linearly, viz.,
  \begin{align} 	\label{eq:tot_model}
 	\sqrt{8t_0}\mu_c^{R,\text{linear}}(a, \phi_2,\phi_H) &=
 	\sqrt{8t_0}\mu_c^{R,\text{cont}} + c_{\mu_c}(a, \phi_2,\phi_H),
 	\\
 	\sqrt{8t_0}\mu_c^{R,\text{non-lin}}(a, \phi_2,\phi_H) &=
 	\sqrt{8t_0}\mu_c^{R,\text{cont}} \times\big(1+ c_{\mu_c}(a, \phi_2,\phi_H)\big), \nonumber
 \end{align}
where $\sqrt{8t_0}\mu_c^{R,\text{cont}}=\sqrt{8t_0}\, \mu_c^{\textrm{R}}(a=0, \phi_2, \phi_H)$. We thus end up with a total of 64 functional forms for each of the two charm matching conditions,
i.e., a total of 128 models.

As in the analysis of the scale setting in Chapter \ref{ch_ss}, we perform a model average as introduced in Sec.~\ref{ch_observables:sec:MA} in order to study the different choices for the chiral-continuum limit extrapolations, assigning to each fit a model weight through the Takeuchi's Information Criterion (TIC), obtaining thus a final weighted average result, as well as a systematic uncertainty coming from the model variation. For a complete discussion of the models considered and their relative weight we refer to~\citep{charm}.

In Table \ref{tab:mc_results_all_matching} we report the results for $\mu_c^{\textrm{R}}$
in units of $\sqrt{8t_0}$ obtained with each of the two matching conditions independently,
as well as for the combined model average.  

\begin{longtable}{c | c c c}
\toprule
&  $\phi_{H}^{(1)}$ & $\phi_{H}^{(2)} $  &   \text{combined} \\
\midrule
$\sqrt{8t_0}\mu_c^{\textrm{R}}$ & 3.349(24)(6) & 3.366(22)(6)  &   3.365(23)(7)  \\
\bottomrule
\caption{Preliminary results of the model average for the renormalized charm quark mass  in units of $\sqrt{8t_0}$ based on the two
		 charm quark mass matching conditions --- $\phi_H^{(1)}$ denotes the flavor-averaged matching 
		 condition in eq.~(\ref{eq:fl_av_matching}) and  $\phi_H^{(2)}$ the $\eta_h^{\mathrm{conn}}$ matching prescription in eq.~(\ref{eq:etac_matching}). The last column reports the combined result from these two matching procedures according to our model average prescription. The first error is 
		 statistical, while the second is the systematic uncertainty arising from the model variation.
                }
		\label{tab:mc_results_all_matching}
\end{longtable}

Figure~\ref{fig:mc_continuum_limit} illustrates typical fits for each of the matching conditions, chosen 
among those with higher weights according to the TIC prescription. The plot shows  the continuum limit behavior of 
the charm quark mass in units of $\sqrt{8t_0}$. Results coming from the two matching strategies coincide in the continuum, in spite of displaying a qualitatively different structure regarding cutoff effects. We observe that the linear dependence of $\mathcal{O}(a^2)$ has to be supplemented by higher order terms to properly describe the lattice data.

Note also the overall small size of scaling violations, which are at the few percent level.
Finally, Figure~\ref{fig:mc_pion_dependence} shows the pion  mass dependence of the charm quark mass, while Figure~\ref{fig:mc_mh_dependence} shows the heavy-quark mass dependence of the charm quark mass. As 
expected, we observe a mild dependence of the charm mass on the light quark masses and a smooth linear interpolation in the heavy-quark mass.
 
\begin{figure}
	\centering 
	\includegraphics[scale=0.45]{./cap6/figs/mc/mc_cl_all_cat.pdf}
	\caption{Comparison of the continuum limit approach for the two  charm matching 
	prescriptions. Shown are two of the fits with the highest weights from the TIC, projected onto the lattice 
	spacing dimension. In yellow we show results for the $\eta_h^{\mathrm{conn}}$ matching condition, while  the blue 
	points illustrate  the flavor-averaged matching. Each data-point in this plot is projected to the 
	physical pion mass and the physical charm quark mass, in order to properly visualize the lattice 
	spacing dependence. }
	\label{fig:mc_continuum_limit}
\end{figure}

\begin{figure}
	\centering
	\includegraphics[scale=0.42]{./cap6/figs/mc/fit_phi2_muc_fl_ave.pdf}
	\caption{Pion mass dependence of the charm quark mass for one of the best  fits according to the TIC criteria. Results are shown for the flavor-averaged matching condition. Each point corresponds to the  value for a given ensemble, projected to the physical charm quark mass. The dashed lines represent the chiral trajectories at finite lattice spacing, while the blue shaded band is a projection to the continuum limit. The red point shows the result extrapolated at the physical point in the continuum. }
	\label{fig:mc_pion_dependence}
\end{figure}

\begin{figure}
 	\centering
 	%\hspace{-15mm}
 	\includegraphics[scale=0.5]{./cap6/figs/mc/fit_phih_interp_muc_fl_ave.pdf}
 	\caption{ Heavy-quark mass dependence of the renormalized charm quark mass $\mu_c^{R}$ in units of $\sqrt{8t_0}$ for one of the fits with larger weights according to the TIC criteria. Results shown for the flavor-averaged matching condition $\phi_{H}^{(1)} = \sqrt{8t_0} m_{\overline{H}}$. Dependencies other than $\phi_H^{(i)}$ in the chiral-continuum extrapolation have been projected to the physical point. The red square symbols indicate the continuum results at the physical value $\phi_H^{\mathrm{ph}}$. We observe a linear dependence of the charm quark mass on $\phi_{H}^{(1)} = \sqrt{8t_0} m_{\overline{H}}$ in the neighborhood of the physical point. }
 	\label{fig:mc_mh_dependence}
 \end{figure}


\subsection{Results for the charm quark mass}

The renormalized charm quark mass 
$\mu_c^{\textrm{R}}$ can be obtained once we combine the results collected in Table~\ref{tab:mc_results_all_matching} with our determination of $\sqrt{t_0^{\mathrm{ph}}}$ in eq.~(\ref{ch_ss:eq:t0ph_c}). As discussed at the beginning of this section, the knowledge of the renormalization group running factors allows  to quote
results for the RGI and $\overline{\textrm{MS}}$ values of the charm quark mass.

After combining the results from our 128 fitting models through the model average procedure,
and using the running factor in eq.~(\ref{eq:rgi_running_factor}), we quote for the three-flavor theory
the value for the RGI quark mass
\begin{equation}
  M_c^{\mathrm{RGI}}(N_f=3) &=& 1.486(8)_{\textrm{stat}}(3)_{\textrm{syst}}(14)_{\textrm{RGI}}\ \mathrm{GeV}\,,
	\label{eq:rgi_charm_mass_result}
\end{equation}
where the first error is statistical, including the uncertainty from  $t_0^{\mathrm{ph}}$,  the second accounts for the systematic uncertainty, derived from the model average, and the third is the error contribution from the RGI running factor in eq.~(\ref{eq:rgi_running_factor}). 

Figure~\ref{fig:mc_error_contributions} illustrates the relative contribution of various sources of error to the
uncertainty of our determination of $M_c^{\mathrm{RGI}}$. The dominant source of error comes from the 
renormalization group running of eq.~(\ref{eq:rgi_running_factor}), while the second most relevant 
contribution arises from the statistical error of  the correlation functions computed in each ensemble.  
The  error coming from  the uncertainty on $t_0^{\mathrm{ph}}$ based on our  scale setting  procedure, as well as the 
systematic error from the model average  are subleading contributions. We therefore expect
that the 
inclusion in this charm quark mass analysis of further ensembles or increased statistics will only have a significant impact if combined with improved determinations of the RGI running factor.
%
\begin{figure}
	\centering
	\includegraphics[scale=0.5]{./cap6/figs/mc/mc_error_pie.pdf}
	\caption{Relative contributions to the total variance of our result for $M_c^{\mathrm{RGI}}(N_f=3)$. The dominant piece comes from the error in the non-perturbative determination of the renormalization group running factor to the RGI mass quoted in eq.~(\ref{eq:rgi_running_factor}). The label statistical plus $\chi$-continuum limit stands for the error arising from the statistical accuracy of our data and the chiral-continuum extrapolation, while the scale setting piece comes from the physical value of the gradient flow scale $t_0^{\mathrm{ph}}$. Finally, the model average piece illustrates the systematic error arising from the set of models considered in this work.
          }
	\label{fig:mc_error_contributions}
\end{figure}
%

In order to quote results in the $\overline{\textrm{MS}}$ scheme, we use five-loop perturbation theory for the quark
mass anomalous dimension~\cite{Baikov:2014qja,Luthe:2016xec,Baikov:2017ujl} and the beta function~\cite{Baikov:2016tgj,Herzog:2017ohr,Luthe:2017ttc}.
The matching between the $N_f=3$ and $N_f=4$ theories uses the four-loop decoupling effects~\cite{Liu:2015fxa}
incorporated into the RunDec package~\cite{Chetyrkin:2000yt,Schmidt:2012az,Herren:2017osy}. Renormalization group equations are solved using as input the value 
$\Lambda^{(3)}_{\overline{\mathrm{MS}}} = 341(12)\ \mathrm{MeV}$ from~\cite{Bruno:2017gxd}. The correlation arising from the fact that a common subset of gauge field configuration ensembles were employed in the computation of $\Lambda^{(3)}_{\overline{\mathrm{MS}}}$ and the non-perturbative running factor in eq.~(\ref{eq:rgi_running_factor}) is taken into account. Our result is shown in Figure~\ref{fig:mc_comparison}, where we compare our determination of the charm quark mass in the $\overline{\textrm{MS}}$ scheme with the results from other lattice QCD calculations also based on $N_f=2+1$ dynamical simulations and with the corresponding FLAG average~\cite{FlavourLatticeAveragingGroupFLAG:2021npn}. We observe in particular a good agreement with the results from \cite{Heitger:2021apz} which are also based on CLS ensembles but employ Wilson fermions in the valence sector.

\begin{figure}
	\centering
	\hspace{-0mm}
	\includegraphics[scale=0.6]{./cap6/figs/mc/mc_comparison_3gev.pdf}
	\caption{Comparison of our charm quark mass determinations in the $\overline{\textrm{MS}}$ scheme with the FLAG average~\cite{FlavourLatticeAveragingGroupFLAG:2021npn} and the results from other lattice QCD calculations based on $N_f=2+1$ dynamical simulations. In our results, shown in blue, we indicate both the total uncertainty and the error when excluding the uncertainty arising from $\Lambda^{(3)}_{\overline{\mathrm{MS}}}$. \textit{Top}: comparison for the  $\overline{m}_c(\mu=3\ \mathrm{GeV}, N_f=4)$. \textit{Bottom}: comparison for $\overline{m}_c(\mu=\overline{m}_c, N_f=4)$.  Starting from the bottom, results are taken from: PDG \cite{ParticleDataGroup:2022pth}, HPQCD 08B \cite{HPQCD:2008kxl}, HPQCD 10 \cite{McNeile:2010ji}, $\chi$QCD \cite{Yang:2014sea}, JLQCD 16 \cite{Nakayama:2016atf}, Maezawa 16 \cite{Maezawa:2016vgv}, Petreczky 19 \cite{Petreczky:2019ozv}, ALPHA 21 \cite{Heitger:2021apz}.
       }
	\label{fig:mc_comparison}
\end{figure}

%%%%%%%%%%%%%%%%%%%%%%%%%%%%%%%%%%%%%%%%%%%%%%%%%%%%%%%%%%%
%%%%%%%%%%%%%%%%%%%%%%%%%%%%%%%%%%%%%%%%%%%%%%%%%%%%%%%%%%%
%%%%%%%%%%%%%%%%%%%%%%%%%%%%%%%%%%%%%%%%%%%%%%%%%%%%%%%%%%%
%%%%%%%%%%%%%%%%%%%%%%%%%%%%%%%%%%%%%%%%%%%%%%%%%%%%%%%%%%%


\section{Determination of decay constants of charmed mesons}
\label{sec:fDs}

For the determination of the decay constants of the charmed mesons $D_{(s)}$ we employ a similar methodology to the one for the renormalized charm quark mass. We match the charm quark mass to its physical value following the same prescription as in Sec.~\ref{sec:matching_charm}, and we explore different ways of performing the chiral-continuum limit extrapolations in order to obtain $f_{D_{(s)}}$ at the physical point. For a detailed discussion we refer to our work~\citep{charm}, here we will only show our main results emphasizing the impact on these of our determination of the scale $t_0$ in Chapter \ref{ch_ss}.

\subsection{Computation of decay constants}

The quantity we employ to extract $f_{D_{(s)}}$ in the continuum and at physical quark masses is
\begin{equation}
  \Phi_{D_{(s)}} = (8t_0)^{3/4}f_{D_{(s)}} \sqrt{m_{D_{(s)}}},
  \label{eq:defphiD}
\end{equation}
for which a Heavy Quark Effective Theory (HQET) scaling law in powers of the inverse
heavy quark mass exists.
The general continuum heavy and light quark mass dependence can be expressed as the product of the individual contributions to arrive at the generic expression 
\begin{equation}
	\Phi_{D_{(s)}} = \Phi_{\chi} \left[
	1 + \delta\Phi_{\chi\mathrm{PT}}^{D_{(s)}}
	\right]
	\left[
	1 + \delta\Phi_a^{D_{(s)}}
	\right]\,.
	\label{eq:fds_different_pieces}
\end{equation}
Here $\Phi_\chi$ governs the heavy-quark mass dependence while  $\delta\Phi_{\chi\mathrm{PT}}^{D_{(s)}}$ controls the light quark behavior as approaching the physical point. Finally, the lattice spacing dependence describing cut-off effects is regulated by $\delta\Phi_a^{D_{(s)}}$. 

For an analysis of each of the terms appearing in eq.~(\ref{eq:fds_different_pieces}) we refer to our work~\citep{charm}. In particular, we refer to eq. (5.13) in the previously cited work. For $\Phi_{\chi}$ we use expressions motivated by HQET, while the light-quark dependence in $\delta\Phi_{\chi\mathrm{PT}}^{D_{(s)}}$ admits an expression in Heavy Meson $\chi$PT (HM$\chi$PT). For cutoff effects, we consider $\mathcal{O}(a^2)$, $\mathcal{O}(a^2\phi_2)$ and $\mathcal{O}(a^2\phi_H)$ terms.

Similarly to the case of the charm quark mass, we scan over various functional forms by including/excluding some of the fit parameters. We furthermore match the charm scale using
the two different procedures described in Sec.~\ref{sec:matching_charm}. The result is a total
of 57 different models  for each matching condition,
and we use the TIC criterion to estimate the systematic uncertainty associated to the variation
within the full set of fits.

In Table~\ref{tab:dec_res_all_matching} we show our determinations of $\Phi_D$
and $\Phi_{D_s}$ for each of the two procedures to match the charm scale, as well
as the result from their combination. Using this combination we arrive at the following results for the $D_{(s)}$ meson decay constants,
\begin{align}
	f_D&=211.1(1.8)_{\textrm{stat}}(0.5)_{\textrm{syst}}\; \textrm{MeV},\\
	f_{D_s}&=248.1(1.5)_{\textrm{stat}}(0.3)_{\textrm{syst}}\; \textrm{MeV},
\end{align}
where the first error is statistical and the second the systematic uncertainty from the model average. The different contributions to the variance of $D_{(s)}$ meson decay constants are 
shown in Figure~\ref{fig:fds_error_sources}. Finally, in  Figure~\ref{fig:fds_comparison} we show a comparison between our results and other $N_f=2+1$ lattice QCD determinations.
%

\begin{longtable}{c | c c c}
\toprule
&  $\phi_{H}^{(1)}$ & $\phi_{H}^{(2)} $  &  \text{combined} \\
\midrule
$\Phi_D$ &  0.8625(60)(16) & 0.8641(68)(48) &   0.8627(58)(19) \\
$\Phi_{D_s}$ & 1.0373(52)(6) & 1.0375(59)(34) &  1.0373(48)(10) \\
\bottomrule
\caption{Preliminary model average results for the observables $\Phi_D$ and $\Phi_{D_s}$ --- defined in eq.~(\ref{eq:defphiD}) ---  which are related to the $f_D$ and $f_{D_s}$ decay constants, respectively, for
		the two different matching quantities $\phi_H^{(i)}$. The last column reports the result of the combination of these two matching conditions. The first error is statistical while the second is the estimate of systematic uncertainty arising from the model averaging procedure. }
		\label{tab:dec_res_all_matching}
\end{longtable}

\begin{figure}
\begin{center}
\begin{minipage}{.8\linewidth}
\includegraphics[width=\linewidth]{././cap6/figs/fds/error_pie_fd.pdf}
\end{minipage}
\hspace{10mm}
\begin{minipage}{.8\linewidth}
\includegraphics[width=\linewidth]{././cap6/figs/fds/error_pie_fds.pdf}
\end{minipage}
\end{center}
\vspace{-5mm}
	\caption{Relative contributions to the total error of our determinations of $f_D$ (\textit{top}) and $f_{D_s}$ (\textit{bottom}). The label statistical plus $\chi$-continuum limit represents the error arising from the statistical accuracy of our data and the chiral-continuum extrapolations. The scale setting label denotes the error coming from the physical value $t_0^{\mathrm{ph}}$ as determined in Chapter \ref{ch_ss}, while the model average represents the systematic error arising from the model variation according to the TIC procedure.	}
	\label{fig:fds_error_sources}
\end{figure}


\begin{figure}
	\centering
	\includegraphics[scale=0.70]{./cap6/figs/fds/fds_comparison.pdf}
	\caption{Comparison of our results for $f_D$ and $f_{D_s}$  with those from lattice QCD collaborations based on simulations with $N_f=2+1$ dynamical flavors as well as with FLAG21 averages~\cite{FlavourLatticeAveragingGroupFLAG:2021npn}.
          Only data points with filled symbols contribute to  the FLAG averages. Starting from the bottom, results are taken from: HPQCD 10 \cite{Davies:2010ip}, PACS-CS 11 \cite{PACS-CS:2011ngu}, FNAL/MILC 11 \cite{FermilabLattice:2011njy}, HPQCD 12A \cite{Na:2012iu}, $\chi$QCD 14 \cite{Yang:2014sea}, RBC/UKQCD 17 \cite{Boyle:2017jwu},  $\chi$QCD 20A \cite{Chen:2020qma}, RQCD/ALPHA 24~\citep{Kuberski:2024pms}.
          }
	\label{fig:fds_comparison}
\end{figure}


\subsection{Direct determination of $f_{D_s}/f_D$}

In addition to the determination of $f_D$ and $f_{D_s}$, we investigate the direct determination
of the ratio $f_{D_s}/f_D$ from a dedicated fit. This allows for a consistency check, since
the ratio is dimensionless and thus does not require normalization with a reference scale
such as $\sqrt{8t_0}$. In this ratio, the scale setting dependence is therefore mainly associated to the matching of the quark masses to their physical values. Another advantage
is that the ratio is exactly~1 by construction when $m_s=m_l$, i.e., at the symmetric
point of our $\phi_4={\rm const.}$ trajectory. We can thus perform a fit that is highly constrained in the unphysical masses
region, at the cost of reducing the total number of ensembles entering in the study of the approach to the physical point.

A first set of fit ansätze is derived from HM$\chi$PT expressions as in the case
for $\Phi_{D_{(s)}}$. The generic form is
\begin{equation}
	\frac{\Phi_{D_s}}{\Phi_D} = \left[
	1 + \left(
	\delta\Phi_{\chi\mathrm{PT}}^{D_s} - \delta\Phi_{\chi\mathrm{PT}}^{D}
	\right)
	\right]
	\left[
	1 + \left(
	\delta\Phi_{a}^{D_s} - \delta\Phi_{a}^{D_s}
	\right)
	\right].
	\label{eq:ratio_fds_expansion}
\end{equation}
Here $\delta\Phi_{\chi\mathrm{PT}}^{D_{(s)}}$ labels the light quark mass dependence of the ratio, while $\delta\Phi_a^{D_{(s)}}$ controls the continuum limit approach. For more details we refer to eq. (5.18) in~\citep{charm}.
In the expression for $\frac{\Phi_{D_s}}{\Phi_D}$ we consider all the possible combinations of non-vanishing fit parameters,
and perform our TIC-weighted model average among the different functional forms tested to
quote a systematic uncertainty.  

We further explore the systematic uncertainties by considering  also functional forms based on a Taylor expansion of $\Phi_{D_{(s)}}$. The generic
expression then reads
\begin{align}
	\Phi_{D_{(s)}}= \left( \Phi_{D_{(s)}}\right)_{\chi} \left[ 1 + \delta \Phi_{{h,\mathrm{Taylor}}} \right] \left[ 1 + \delta \Phi_{{m,\mathrm{Taylor}}}^{D_{(s)}} \right] \left[ 1 + \delta \Phi_a^{D_{(s)}}  \right]
	\,,
	\label{eq:phiqcontT}
\end{align}
where $ \left( \Phi_{D_{(s)}}\right)_{\chi}$ is the value in
the chiral limit and at the physical value of the heavy-quark mass. More concretely, we refer to eq. (5.21) in~\citep{charm}.


Then, in order to arrive at our determination of $f_{D_s}/f_D$ we perform a model average among all the HM$\chi$PT and Taylor functional forms, considering all the possible combinations of non-vanishing fit parameters, for the two different matching conditions simultaneously. In Table~\ref{tab:ratio_res_all_matching} we report our results for the
ratio of decay constants from the model average separately for each charm matching
condition, as well as their combination. Also for the ratio we observe good agreement for the two different $\phi_H^{(i)}$ tested in this work. 

\begin{longtable}{c | c c c}
\toprule
&  $\phi_{H}^{(1)}$ & $\phi_{H}^{(2)} $  &  \text{combined} \\
\midrule
$f_{D_s}/f_D$   &  1.1651(91)(15)& 1.1650(91)(16) &  1.1649(90)(16) \\
\bottomrule
\caption{Preliminary results of the model average for $f_{D_s}/f_D$ for the two charm-quark matching conditions. The last column reports the combined result. The first error is statistical while the second is the systematic uncertainty arising from the model variation procedure. }
		\label{tab:ratio_res_all_matching}
\end{longtable}

  In Figure \ref{fig:fds_ratio_error}  we show the major error sources contributing to our final determination of the ratio, where we notice that the major contribution is given by the statistical and chiral-continuum error.


\begin{figure}
\begin{center}
\begin{minipage}{.4\linewidth}
\includegraphics[width=\linewidth]{././cap6/figs/fds/error_pie_ratio_fds.pdf}
\end{minipage}
\hspace{15mm}
\begin{minipage}{.8\linewidth}
\includegraphics[width=\linewidth]{././cap6/figs/fds/error_pie_ratio_fds_statonly.pdf}
\end{minipage}
\end{center}
\vspace{-5mm}
	\caption{\textit{Top}: Relative contributions to the total error on the determination of the ratio $f_{D_s}/f_D$. The label statistical plus $\chi$-continuum limit represents the error arising from the statistical accuracy of our data and the chiral-continuum extrapolation. The scale setting label denotes the error coming from the physical value $t_0^{\mathrm{ph}}$, while the model average represents the systematic error arising from the model variation according to the TIC procedure. \textit{Bottom}: Details of the relative contributions to the statistical and chiral-continuum extrapolation error arising from specific gauge field configuration ensembles. 
          }
	    \label{fig:fds_ratio_error}
\end{figure}


%%%%%%%%%%%%%%%%%%%%%%%%%%%%%%%%%%%%%%%%%%%%%%%%%%%%%%%%%%%
%%%%%%%%%%%%%%%%%%%%%%%%%%%%%%%%%%%%%%%%%%%%%%%%%%%%%%%%%%%
