\chapter*{Conclusions and outlook}\addcontentsline{toc}{chapter}{Conclusions and outlook}
\markboth{CONCLUSIONS}{CONCLUSIONS}
\label{ch_conclu}

In this Ph.D. thesis we have reported on a scale setting procedure that provides a new lattice QCD determination of the gradient flow scale $t_0$ and the lattice spacing for CLS ensembles. Accurate scale setting determinations are paramount to reach the sub-percent precision level required for some of the lattice QCD calculations aimed at improving the precision of Standard Model predictions. The results of the scale setting procedure are being used in an ongoing study aimed at improving the determination of quark masses and $D_{(s)}$ decay constants. These quantities are necessary to improve the determination of some of the fundamental parameters of the Standard Model and to strengthen the consistency checks of its validity.

In this work we employed lattice gauge field configurations generated by the CLS initiative~\citep{Bruno:2014jqa,Mohler:2017wnb} with lattice spacings ranging from $a\approx0.085$ fm to $a\approx0.038$ fm, and pion masses from $m_{\pi}\approx420$ MeV down to the physical point $m_{\pi}\approx130$ MeV. We have used a mixed action lattice regularization based on CLS gauge ensembles with $N_f=2+1$ $\mathcal{O}(a)$ improved sea Wilson quarks and $N_f=2+1+1$ valence Wilson twisted mass quarks. We performed the matching of the mixed action through the pseudoscalar pion and kaon masses, which equates physical masses for the up/down and strange quarks in the sea and valence sectors, treating the additional charm quark as a partially quenched flavor. This ensures the unitarity of the theory in the continuum limit. Furthermore, we tuned the parameters of the Wilson twisted mass Dirac operator in order to impose maximal twist, ensuring automatic $\mathcal{O}(a)$ improvement~\citep{Frezzotti:2003ni,Shindler:2007vp} for valence observables up to subleading effects coming from the sea sector.

We employ the $\Gamma$--method to compute the errors of the Monte Carlo data together with automatic differentiation to perform error propagation that is accurate to machine precision. This allows arbitrarily complex derived observables to be considered while retaining adequate control of autocorrelations. These techniques are implemented within the ADerrors.jl Julia library~\citep{Ramos:2018vgu,Ramos:2020scv}. 

For the scale setting procedure based on a combination of the Wilson and Wilson twisted mass quark regularizations we employed the pion and kaon decay constants as physical input. We obtain the following result for $\sqrt{t_0}$
\begin{align}
\sqrt{t_0}&=0.1441(6)_{\textrm{stat}}(4)_{\textrm{syst}}\;\textrm{fm},\;[f_{\pi K}].
\end{align}
Using the kaon decay constant to set the scale relies on the determination of the CKM matrix element $V_{us}$ which has a larger uncertainty than $V_{ud}$. The uncertainty from $|V_{us}|$ amounts to about $6.5\%$ of the total squared error of $\sqrt{t_0}$. In addition, $f_K$ receives larger QED corrections than $f_{\pi}$, whose uncertainty amounts to a $\sim1.6\%$ contribution to the total squared error. It is therefore desirable to consider also the case where only the pion decay constant is used as an external input in the scale setting procedure. The use of physical point ensembles with various values of the lattice spacing is expected to play a decisive role in such an analysis. This would be a natural extension of the analysis presented in this work, together with the determination of the up/down and strange quark masses from a combination of the Wilson unitary and mixed action regularizations, of which we provide a preliminary analysis in Appendix \ref{apex_light_qm}.

Furthermore, following our work in~\citep{charm} we have presented the current status of the determination of the physical charm quark mass and charmed mesons decay constants based on this mixed action setup, exploiting automatic $\mathcal{O}(a)$ improvement to reduce lattice artifacts associated with the heavy quark mass. Using our determination of the scale $t_0$ we quote as result for the RGI charm quark mass in the three flavor theory
\begin{equation}
  M_c^{\mathrm{RGI}}(N_f=3) &=& 1.486(8)_{\textrm{stat}}(3)_{\textrm{syst}}(14)_{\textrm{RGI}}\ \mathrm{GeV}\,.
\end{equation}
The error of the RGI quark mass is completely dominated by the computation of the non-perturbative renormalization group running factor, and therefore, no substantial improvement can be achieved until a more precise calculation of this quantity is obtained. In particular, the uncertainty in the scale $t_0$ accounts for $\sim3\%$ of the squared total error in $M_c^{\mathrm{RGI}}(N_f=3)$.

For the $D_{(s)}$ decay constants we quote
\begin{align}
	f_D &= 211.1(1.8)_{\textrm{stat}}(0.5)_{\textrm{syst}} \ \mathrm{MeV},
	\\
	f_{D_s} &= 248.1(1.5)_{\textrm{stat}}(0.3)_{\textrm{syst}} \ \mathrm{MeV}.
\end{align}
In this case, the error is completely dominated by the statistical uncertainty of the gauge ensembles and the chiral-continuum extrapolations, and the scale setting accounts for the second largest contribution.

The results quoted in this work were obtained in the isosymmetric limit of QCD, defined in~\citep{FlavourLatticeAveragingGroupFLAG:2021npn}. As the accuracy of lattice results continues to improve, the inclusion of QED and strong isospin breaking effects will become increasingly relevant for constraining precision physics observables. Another avenue for future developments consists in the extension of a setup combining Wilson and twisted Wilson mass fermions to approach  the b-quark sector, following a step-scaling strategy \cite{Sommer:2023gap}.


