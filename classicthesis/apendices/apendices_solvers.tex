%%%%%%%%%%%%%%%%%%%%%%%%%%%%%%%%%%%%%%%%%%%%%%%%%%%%%%%%%%%
%%%%%%%%%%%%%%%%%%%%%%%%%%%%%%%%%%%%%%%%%%%%%%%%%%%%%%%%%%%
%%%%%%%%%%%%%%%%%%%%%%%%%%%%%%%%%%%%%%%%%%%%%%%%%%%%%%%%%%%
%%%%%%%%%%%%%%%%%%%%%%%%%%%%%%%%%%%%%%%%%%%%%%%%%%%%%%%%%%%

\chapter{Solvers}
\label{appex_solvers}

\section{Stochastic methods}

For the computation of correlation functions of fermions in the lattice (e.g. a two-point function, see eq.~(\ref{ch_foundation:eq:path_int})) the inversion of the Dirac operator is required. In particular one needs to compute the inverse of $D(x,y)$ for all spatial positions $\vec{x},\vec{y}$. This is referred to as inverting the all-to-all Dirac operator. This is computationally very expensive, and stochastic methods can be employed to reduce the computational cost~\citep{Luscher:2010ae}. A set of stochastic noise sources $\eta$ are introduced such that
\begin{gather}
\left<\eta_i(x)\right>_{\eta}=0, \quad \left<\eta_i^{\dagger}(x)\eta_j(y)\right>_{\eta}=\delta_{x,y}\delta_{i,j},
\end{gather}
with $\left<.\right>_{\eta}$ meaning average over the $N_{\eta}$ samples of some noise distribution. Some common choices are Gaussian, $Z_2$ or $U(1)$. From these we define
\begin{gather}
\xi_i^q(x)=\sum_yD^{-1}_q(x,y)\eta_i(y), \quad \zeta_{i,B}^r(x)=\sum_yD^{-1}_r(x,y)\gamma_5\Gamma_B^{\dagger}\eta_i(y),
\end{gather}
with $\Gamma_B$ some Gamma matrix. Now, two-point functions like the one in eq.~(\ref{ch_foundation:eq:path_int}) can be expressed as
\begin{align}
\left<O^{rq}_A(x_0)O^{qr}_B(y_0)\right>&\approx -\frac{a^6}{L^3}\frac{1}{N_{\eta}}\sum_{i=1}^{N_{\eta}}\sum_{\vec{x}}\left<(\Gamma_A\gamma_5\zeta^r_{i,B}(x))^{\dagger}\xi^q_i(x)\right>,
\end{align}
without the need to compute the all-to-all inverted Dirac operator, therefore reducing significantly the computational effort.

\section{Iterative solvers}

The inversion of the Dirac operator is still needed to compute correlation functions. This means solving the Dirac equation
\begin{equation}
D_q(x,y)\psi_r(y)=\delta_{x,y}\delta_{q,r}\equiv\eta_{x,y,q,r},
\end{equation}
where $q$ and $r$ are flavor indices. This is usually done by an iterative procedure. The basic idea is to start from an initial approximate solution $\psi_0$ and define the residue $\rho$ (we suppress indices for simplicity)
\begin{equation}
\rho=D\psi_0-\eta.
\end{equation}
Then, one solves
\begin{equation}
D\psi_1=\rho,
\end{equation}
finds the new residue and iterates the process. The algorithm stops when some convergence criterion is met
\begin{equation}
|\rho|<\epsilon,
\end{equation}
with the final approximate solution given by
\begin{equation}
\psi=\psi_0+\psi_1+...
\end{equation}
The difference between the true and approximate solutions is
\begin{equation}
|\psi-\psi_{\textrm{true}}|<\epsilon\kappa(D),
\end{equation}
with $\kappa(D)$ the condition number of matrix $D$
\begin{equation}
\kappa(D)=|D||D^{-1}|.
\end{equation}

The main solvers used in modern lattice simulations use the iterative procedure outlined above to solve the Dirac equation. Some popular choices are the CG, BiCGstab and GCR algorithms, which are all based on the Krylov subspace method and belong to the class of conjugate gradient solvers. For a pedagogical introduction we refer to ~\citep{Gattringer_Lang_2010,Luscher:2010ae}.

\section{Preconditioning}

The smaller the condition number of the Dirac operator, the less iterative steps one needs to perform in order to find the solution to the Dirac equation. Thus, convergence can be improved by suitably transforming the system into one with a smaller $\kappa(D)$. This can be done by finding some similarity transformations easily invertible such that one can write
\begin{gather}
LDR\psi'=L\eta, \quad \psi=R\psi'.
\end{gather}
This is called preconditioning, and there are many different variations. One of the most used is even-odd preconditioning~\citep{DEGRAND1988161}. Lattice sites can be categorized as even or odd depending on the sum of their space-time coordinates. If the points are ordered such that all the even ones come first, the Dirac operator takes the block form
\begin{equation}
D=\begin{pmatrix}
D_{ee} & D_{eo} \\ 
D_{oe} & D_{oo}
\end{pmatrix},
\end{equation}
where $D_{eo}$ captures the terms which couple the odd to the even sites. The block on the diagonal do not couple different lattice sites, and as such $D_{ee,oo}$ are easily invertible. Choosing the preconditioners
\begin{gather}
L=\begin{pmatrix}
1 & -D_{eo}D_{oo}^{-1} \\ 
0 & 1
\end{pmatrix}, \quad 
R=\begin{pmatrix}
1 & 0 \\ 
-D_{oo}^{-1}D_{oe} &  1
\end{pmatrix},
\end{gather}
we get
\begin{gather}
LDR=\begin{pmatrix}
\hat{D} & 0 \\ 
0 & D_{oo}
\end{pmatrix}, \quad
\hat{D}=D_{ee}-D_{eo}D_{oo}^{-1}D_{oe}.
\end{gather}
The condition number of $\hat{D}$ is usually less than half that of $D$, and thus even-odd preconditioning leads to an acceleration of the solver by a factor $\geq2$.

Another widely used preconditioning of particular relevance for heavy flavors is distance preconditioning~\citep{deDivitiis:2010ya}. 



%%%%%%%%%%%%%%%%%%%%%%%%%%%%%%%%%%%%%%%%%%%%%%%%%%%%%%%%%%%
%%%%%%%%%%%%%%%%%%%%%%%%%%%%%%%%%%%%%%%%%%%%%%%%%%%%%%%%%%%
%%%%%%%%%%%%%%%%%%%%%%%%%%%%%%%%%%%%%%%%%%%%%%%%%%%%%%%%%%%
%%%%%%%%%%%%%%%%%%%%%%%%%%%%%%%%%%%%%%%%%%%%%%%%%%%%%%%%%%%

