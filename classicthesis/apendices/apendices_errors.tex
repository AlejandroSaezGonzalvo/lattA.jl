%%%%%%%%%%%%%%%%%%%%%%%%%%%%%%%%%%%%%%%%%%%%%%%%%%%%%%%%%%%
%%%%%%%%%%%%%%%%%%%%%%%%%%%%%%%%%%%%%%%%%%%%%%%%%%%%%%%%%%%
%%%%%%%%%%%%%%%%%%%%%%%%%%%%%%%%%%%%%%%%%%%%%%%%%%%%%%%%%%%
%%%%%%%%%%%%%%%%%%%%%%%%%%%%%%%%%%%%%%%%%%%%%%%%%%%%%%%%%%%

\chapter{Error analysis}
\label{appex_errors}

In this appendix we discuss how to perform the data analysis of correlation functions and the different lattice observables extracted from lattice simulations. 

Lattice data is measured from Monte Carlo (MC) sampling, and estimates of expectation values of physical observables are extracted from means over the MC time. A crucial step is to assign a proper statistical and systematic uncertainties to these mean values, for which it is needed to take into account the autocorrelated nature of MC measurements. This autocorrelation arises form the fact that each gauge configuration is proposed from the previous one (Markov chain). Some popular methods to deal with these correlations are binning, bootstrap and the jack-knife methods.

A recent technique which we will use in this work was proposed by the ALPHA collaboration~\citep{Schaefer:2010hu,Wolff:2003sm,Ramos:2018vgu} and is known as $\Gamma$-method, which explicitely computes the autocorrelation function to estimate the statistical uncertainty.

In lattice simulations tipically one measures a primary observable $p_i$ over several ensembles (defined by the simulation parameters like e.g. the inverse coupling $\beta$ and $\kappa$ parameter) 
\begin{equation}
p_i^{\alpha}(k), k=1,...,N_{\alpha},
\end{equation}
where $\alpha$ labels the ensemble and $k$ is the MC time running over the total number of gauge configurations $N_{\alpha}$ of the given ensemble. In our context, primary observable means a correlation function for some given euclidean time. An estimate for the true value $P_i^{\alpha}$ is given by the mean value
\begin{equation}
\bar{p}_i^{\alpha}=\frac{1}{N_{\alpha}}\sum_{k=1}^{N_{\alpha}}p_i^{\alpha}(k)\rightarrow_{N_{\alpha}\rightarrow\infty}P_i^{\alpha}.
\end{equation}
This is an unbiased estimator. Fluctuations over the MC time can be computed as
\begin{equation}
\delta_i^{\alpha}(k)=p_i^{\alpha}(k)-\bar{p}_i^{\alpha}.
\end{equation}
Due to the Central Limit theorem, we are ensured that $\bar{p}_i^{\alpha}$ behaves as a Gaussian distribution independently of the distribution of $p_i^{\alpha}(k)$, and so the statistical uncertainty associated to $\bar{p}_i^{\alpha}$ is simply given by the standard deviation $\sigma_i^{\alpha}$,
\begin{equation}
P_i^{\alpha}\approx\bar{p}_i^{\alpha}\pm\sigma_i^{\alpha}.
\end{equation}
This standard deviation can be computed from the autocorrelation $\Gamma$ function
\begin{equation}
(\sigma_i^{\alpha})^2=\frac{1}{N_{\alpha}}\sum_{k=-\infty}^{\infty}\Gamma_{ii}^{\alpha\alpha}(k),
\end{equation}
where the $\Gamma$ function is defined as
\begin{equation}
\Gamma_{ij}^{\alpha\beta}=\frac{\delta_{\alpha\beta}}{N_{\alpha}-k}\sum_{k'=1}^{N_{\alpha}-k}\delta_i^{\alpha}(k+k')\delta_j^{\alpha}(k').
\end{equation}

From the primary observable $p_i^{\alpha}$ we can compute derived observables $F=f(p_i^{\alpha})$, such as meson masses coming from pseudoscalar two point functions. As in the primary observable case, we can estimate this derived observable as
\begin{equation}
\bar{F}=f(\bar{p}_i^{\alpha}).
\end{equation}
To compute the statistical uncertainty, we can expand $f$ around the true value $P_i^{\alpha}$
\begin{equation}
f(P_{i}^{\alpha}+\epsilon_{i}^{\alpha})=f(P_{i}^{\alpha})+\bar{f}_i^{\alpha}\epsilon_{i}^{\alpha}+\mathcal{O}((\epsilon_{i}^{\alpha})^2),
\end{equation}
with
\begin{equation}
\bar{f}_i^{\alpha}=\frac{\partial f(x)}{\partial x}|_{x=P_{i}^{\alpha}}.
\end{equation}
Now the autocorrelation function of the derived observable $F$ for ensemble $\alpha$ can be defined as
\begin{equation}
\Gamma_F^{\alpha}(k)=\sum_{ij}\bar{f}_i^{\alpha}\bar{f}_j^{\alpha}\Gamma_{ij}^{\alpha\alpha}(k),
\end{equation}
from which the standard deviation of $F$ can be derived
\begin{equation}
\sigma_F^2=\sum_{\alpha}\frac{\Gamma_F^{\alpha}(0)}{N_{\alpha}}2\tau_{\textrm{int}}^{\alpha}(F),
\end{equation}
where we assumed that several ensembles contribute to $F$, and hence the sum $\sum_{\alpha}$ over the subset of them which contribute. The integrated autocorrelation time $\tau_{\textrm{int}}^{\alpha}(F)$ is defined as
\begin{equation}
\label{app_errors:eq:taui}
\tau_{\textrm{int}}^{\alpha}(F)=\frac{1}{2}+\sum_{k=1}^{\infty}\frac{\Gamma_F^{\alpha}(k)}{\Gamma_F^{\alpha}(k)}.
\end{equation}
To estimate it, a truncation in MC time $k$ is needed. The autocorrelation function admits the following expansion~\citep{Luscher:2011kk,Schaefer:2010hu}
\begin{equation}
\Gamma(k)\approx\sum_{n=0}^{\infty}e^{-k/\tau_n}.
\end{equation}
The slowest mode $\tau_0\equiv\tau_{\textrm{exp}}$ is called the exponential autocorrelation time and it gives the decay rate of $\Gamma(k)$. Truncating eq.~(\ref{app_errors:eq:taui}) at MC time $k=W_F^{\alpha}$ introduces a systematic uncertainty of $\mathcal{O}(\exp(-W_F^{\alpha}/\tau_{\textrm{exp}}^{\alpha}))$. The $\Gamma$-method proposes as optimal window that which minimizes the sum of statistical (estimated in~\citep{Madras:1988ei}) and systematic contributions
\begin{equation}
W_F^{\alpha}=\textrm{min}_W\left(\sqrt{\frac{2(2W+1)}{N_{\alpha}}}+e^{-W/\tau_{\textrm{exp}}^{\alpha}}\right).
\end{equation}
In~\citep{Wolff:2003sm} it was proposed to set $\tau_{\textrm{exp}}=S_{\tau}\tau_{\textrm{int}}$, with $S_{\tau}$ some value between 2 and 5. One can also vary $W_F^{\alpha}$ until saturation in $\tau_{\textrm{int}}^{\alpha}$ is reached. Finally it was also proposed to add an exponential tail~\citep{Wolff:2003sm}
\begin{equation}
\tau_{\textrm{exp}}^{\alpha}\frac{\Gamma_F^{\alpha}(W_F^{\alpha}+1)}{\Gamma_F^{\alpha}(0)},
\end{equation}
to eq.~(\ref{app_errors:eq:taui}) to account for the systematic effect of truncating the sum over MC time. For this an estimate of $\tau_{\textrm{exp}}^{\alpha}$ is needed for each ensemble. In the case of CLS ensembles an estimation is given in~\citep{Bruno:2014jqa}
\begin{equation}
\tau_{\textrm{exp}}^{\alpha}=14(3)\frac{t_0}{a^2}.
\end{equation}

In this thesis we use the $\Gamma$-method explained above as it is implemented by the ADerrors.jl julia package~\citep{Ramos:2020scv}.


%%%%%%%%%%%%%%%%%%%%%%%%%%%%%%%%%%%%%%%%%%%%%%%%%%%%%%%%%%%
%%%%%%%%%%%%%%%%%%%%%%%%%%%%%%%%%%%%%%%%%%%%%%%%%%%%%%%%%%%
%%%%%%%%%%%%%%%%%%%%%%%%%%%%%%%%%%%%%%%%%%%%%%%%%%%%%%%%%%%
%%%%%%%%%%%%%%%%%%%%%%%%%%%%%%%%%%%%%%%%%%%%%%%%%%%%%%%%%%%

