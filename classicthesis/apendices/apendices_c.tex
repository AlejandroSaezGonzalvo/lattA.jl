%%%%%%%%%%%%%%%%%%%%%%%%%%%%%%%%%%%%%%%%%%%%%%%%%%%%%%%%%%%
%%%%%%%%%%%%%%%%%%%%%%%%%%%%%%%%%%%%%%%%%%%%%%%%%%%%%%%%%%%
%%%%%%%%%%%%%%%%%%%%%%%%%%%%%%%%%%%%%%%%%%%%%%%%%%%%%%%%%%%
%%%%%%%%%%%%%%%%%%%%%%%%%%%%%%%%%%%%%%%%%%%%%%%%%%%%%%%%%%%

\chapter{Finite Volume Effects}
\label{apex_fv}

Simulating QCD in a finite box introduces finite volume effects which can be a source for systematic uncertainties. In Table~\ref{apex_ensembles:tab:ens} we show the volume of each ensemble in terms of $m_{\pi}L$. It is conventional to consider finite volume effects under control if $m_{\pi}L\geq4$.

ChPT provides us with formulae to account for these finite volume effects. In particular, to NLO the pion mass and decay constant, as well as the kaon decay constant, receive the following corrections~\citep{Colangelo:2003hf},\citep{Colangelo:2005gd}
\begin{equation}
X^{(\infty)}=X^{(L)}\frac{1}{1+R_X},
\end{equation}
where $X^{(\infty)}$ is observable $X$ at infinite volume and $X^{(L)}$ is said observable at a finite volume $L^3$, with $X=m_{\pi},m_K,f_{\pi},f_K$,
\begin{align}
R_{m_{\pi}}&=\frac{1}{4}\xi_{\pi}\tilde{g}_1(\lambda_{\pi})-\frac{1}{12}\xi_{\eta}\tilde{g}_1(\lambda_{\eta}), \\
R_{m_K}&=\frac{1}{6}\xi_{\eta}\tilde{g}_1(\lambda_{\eta}), \\
R_{f_K}&=-\xi_{\pi}\tilde{g}_1(\lambda_{\pi})-\frac{1}{2}\xi_{K}\tilde{g}_1(\lambda_{K}), \\
R_{f_{\pi}}&=-\frac{3}{8}\xi_{\pi}\tilde{g}_1(\lambda_{\pi})-\frac{3}{4}\xi_{K}\tilde{g}_1(\lambda_{K})-\frac{3}{8}\xi_{\eta}\tilde{g}_1(\lambda_{\eta}), \\
\xi_{PS}&=\frac{m_{PS}^2}{(4\pi f_{\pi})^2}, \\
\lambda_{PS}&=m_{PS}L, \\
\tilde{g}_1(x)&=\sum_{n=1}^{\infty}\frac{4m(n)}{\sqrt{n}x}K_1(\sqrt{n}x), \\
m_{\eta}^2&=\frac{4}{3}m_K^2-\frac{1}{3}m_{\pi}^2,
\end{align}
where $K_1(x)$ is a Bessel function of the second kind, and the multiplicities $m(n)$~\citep{Colangelo:2003hf} are listed in Table~\ref{apex_fv:tab:mn}. It is manifest that the lighter the pion mass and the smaller the volume, the stronger the volume corrections. We find these corrections to be less than half a standard deviation for the ensembles with the smallest volumes and lightest pion masses. We nonetheless apply the corrections to all the ensembles.

PCAC quark masses are short distance observables and as such do not receive any infinite volume correction.

\vspace{2cm}

\begin{longtable}{c c c c c c c c c c c c c c c c c c c c c}
\toprule
$n$ & 1 & 2 & 3 & 4 & 5 & 6 & 7 & 8 & 9 & 10 & 11 & 12 & 13 & 14 & 15 & 16 & 17 & 18 & 19 & 20 \\
\midrule
$m(n)$ & 6 & 12 & 8 & 6 & 24 & 24 & 0 & 12 & 30 & 24 & 24 & 8 & 24 & 48 & 0 & 6 & 48 & 36 & 24 & 24 \\
\bottomrule
\caption{Multiplicities $m(n)$ calculated in~\citep{Colangelo:2003hf} for $n\leq20$.}
\label{apex_fv:tab:mn}
\end{longtable}

%%%%%%%%%%%%%%%%%%%%%%%%%%%%%%%%%%%%%%%%%%%%%%%%%%%%%%%%%%%
%%%%%%%%%%%%%%%%%%%%%%%%%%%%%%%%%%%%%%%%%%%%%%%%%%%%%%%%%%%
%%%%%%%%%%%%%%%%%%%%%%%%%%%%%%%%%%%%%%%%%%%%%%%%%%%%%%%%%%%
%%%%%%%%%%%%%%%%%%%%%%%%%%%%%%%%%%%%%%%%%%%%%%%%%%%%%%%%%%%

