%%%%%%%%%%%%%%%%%%%%%%%%%%%%%%%%%%%%%%%%%%%%%%%%%%%%%%%%%%%
%%%%%%%%%%%%%%%%%%%%%%%%%%%%%%%%%%%%%%%%%%%%%%%%%%%%%%%%%%%
%%%%%%%%%%%%%%%%%%%%%%%%%%%%%%%%%%%%%%%%%%%%%%%%%%%%%%%%%%%
%%%%%%%%%%%%%%%%%%%%%%%%%%%%%%%%%%%%%%%%%%%%%%%%%%%%%%%%%%%

\chapter{Least-squares fitting}
\label{apex_chisq}

We employ a least-squares method to fit our data to some fit function. This method is based on finding the  minimum of the $\chi^2$ function
\begin{equation}
\label{apex_chisq:eq:chisq}
\chi^2=\sum_{i,j=1}^{N_{\textrm{dat}}}\left(y_i-f(x_i;\vec{p})\right)\mathcal{W}_{ij}\left(y_j-f(x_j;\vec{p})\right),
\end{equation}
where $\{x_i,y_i\}_{i=1,...,N_{\textrm{dat}}}$ are the data points we want to fit, $x$ being the independent variable and $y$ the abscissa. $\mathcal{W}$ is a matrix which gives different weights to the different data points entering the fit. When $\mathcal{W}$ is chosen to be the inverse of the covariance matrix of the $y$-data, $C^{-1}$, the fit is said to be fully correlated. For fits employing a large number of data points, inverting the covariance matrix can be challenging. Alternatively, an uncorrelated fit corresponds to the case in which the weight matrix $\mathcal{W}$ is set to the  inverse of the matrix including only the diagonal part of $C$. $f(x;\vec{p})$ is the fit function with fit parameters $\vec{p}=(p_1,...,p_{N_{\textrm{param}}})$. For a given fit function $f(x;\vec{p})$, the method finds the parameters values that minimize eq.~(\ref{apex_chisq:eq:chisq}) for given data points $\{x_i,y_i\}_{i=1,...,N_{\textrm{dat}}}$.

In our case we perform fits to extract the ground state signal of lattice observables, fitting e.g. an effective mass to a constant plus exponential signals along the lattice time extent. In this case, Euclidean time plays the role of the $x$. The Euclidean-time fit intervals may include $\mathcal{O}(100)$ correlated data points, which in general precludes the possibility of inverting the covariance matrix. We therefore have to rely on uncorrelated fits. With the exception of the definition of the $chi^2$ function, correlations present in the data are retained in the statistical analysis and propagated to the target observables.

In~\citep{Bruno:2022mfy} a method to measure the goodness of fits was proposed in terms of p-values, irrespective of the choice of the weight matrix $\mathcal{W}$. Also a definition of the expected value of the minimum of $\chi^2$, $\left<\chi^2\right>$ is provided. In the case of a fully correlated fit it holds that $\left<\chi^2\right>={\textrm{dof}}$ (number of degrees of freedom).

We also perform fits for the chiral-continuum extrapolation of $\sqrt{8t_0}f_{\pi K}$ to set the scale. In this case, the $y$ variable is $\sqrt{8t_0}f_{\pi K}$ while the $x$ is $\phi_2$, and thus the latter has its own uncertainty. In this situation, a generalized $\chi^2$ function can be defined to include uncertainties of $x$ as
\begin{gather}
\label{apex_chisq:eq:chisq_generalized}
\chi^2=\sum_{i,j=1}^{2N_{\textrm{dat}}}\left(Y_i-F(X_i;\vec{p},\vec{q})\right)\mathcal{W}_{ij}\left(Y_j-F(X_j;\vec{p},\vec{q})\right), \\
Y=(x_1,...,x_{N_{\textrm{dat}}},y_1,...,y_{N_{\textrm{dat}}}), \quad
X=(x_1,...,x_{N_{\textrm{dat}}},x_1,...,x_{N_{\textrm{dat}}}), \\
F(X_i;\vec{p},\vec{q})=\left\{\begin{matrix}
q_i & \textrm{ if $1\leq i\leq N_{\textrm{dat}}$} \\ 
f(x_i;\vec{p}) & \textrm{ if $N_{\textrm{dat}}+1\leq i\leq 2N_{\textrm{dat}}$}
\end{matrix}\right..
\end{gather}
A fully correlated fit in this context corresponds to setting $\mathcal{W}$ to the inverse covariance matrix of the generalized data vector $Y$, $\mathcal{C}$. In practice, the dimension of the full covariance matrix $\mathcal{C}$ can reach $\mathcal{O}(50)$ and, in general it is therefore not possible to invert it. We consider, however, a block structure for $\mathcal{C}$. The block corresponding to the correlation among  the $\sqrt{8t_0}f_{\pi K}$ data is maintained while the correlations associated to the other blocks are neglected in the definition of the $\chi^2$ function. However, all other steps in the analysis chain take full account  of the correlations and, in particular, those associated with $\phi_2$, $t_0/a^2$ and $\sqrt{8t_0}f_{\pi K}$. Including only the correlations from $\sqrt{8t_0}f_{\pi K}$ in the $chi^2$ of the fits leads to an expectation value of the $chi^2$ that deviates only slightly from the number of degrees of freedom
\begin{equation}
\frac{\left<\chi^2\right>}{{\textrm{dof}}}\sim0.98.
\end{equation}
This indicates that the bulk of the correlations are effectively incorporated in the fit.



%%%%%%%%%%%%%%%%%%%%%%%%%%%%%%%%%%%%%%%%%%%%%%%%%%%%%%%%%%%
%%%%%%%%%%%%%%%%%%%%%%%%%%%%%%%%%%%%%%%%%%%%%%%%%%%%%%%%%%%
%%%%%%%%%%%%%%%%%%%%%%%%%%%%%%%%%%%%%%%%%%%%%%%%%%%%%%%%%%%
%%%%%%%%%%%%%%%%%%%%%%%%%%%%%%%%%%%%%%%%%%%%%%%%%%%%%%%%%%%

