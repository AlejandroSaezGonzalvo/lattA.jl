%%%%%%%%%%%%%%%%%%%%%%%%%%%%%%%%%%%%%%%%%%%%%%%%%%%%%%%%%%%
%%%%%%%%%%%%%%%%%%%%%%%%%%%%%%%%%%%%%%%%%%%%%%%%%%%%%%%%%%%
%%%%%%%%%%%%%%%%%%%%%%%%%%%%%%%%%%%%%%%%%%%%%%%%%%%%%%%%%%%
%%%%%%%%%%%%%%%%%%%%%%%%%%%%%%%%%%%%%%%%%%%%%%%%%%%%%%%%%%%

\chapter{Conventions}
\label{apex_conventions}

In this appendix we set some useful notation used throughout this work. We begin with the Dirac or Gamma matrices $\gamma_{\mu}$, which are $4\times 4$ complex matrices defined by the anticommutator relation
\begin{equation}
\{\gamma_{\mu},\gamma_{\nu}\}=2g_{\mu\nu}1_{4\times 4},
\end{equation}
with $g_{\mu\nu}$ the metric tensor of 4-dimensional space-time. We will work in the Euclidean and flat space, so
\begin{equation}
g_{\mu\nu}=\textrm{diag}(+1,+1,+1,+1).
\end{equation}
Some useful properties of the Gamma matrices are
\begin{itemize}
\item Hermiticity: $\gamma_{\mu}^{\dagger}=\gamma_{\mu}$.
\item They are traceless: $\textrm{tr}(\gamma_{\mu})=0$.
\item Involutory: $\gamma_{\mu}^{-1}=\gamma_{\mu}$.
\end{itemize}
A fifth Gamma matrix can be defined as
\begin{equation}
\gamma_5=\gamma_0\gamma_1\gamma_2\gamma_3,
\end{equation}
which fullfils the same properties as above, and anticommutes with all other Gamma matrices
\begin{equation}
\{\gamma_5,\gamma_{\mu}\}=0.
\end{equation}

These matrices control the flavor content of hadrons, and as such appear in the definition of the lattice hadron interpolators. The relevant quark bilinears needed for this work are
\begin{itemize}
\item Scalar density: $S^{ij}=\bar{\psi}^i\psi^j$.
\item Pseudoscalar density: $P^{ij}=\bar{\psi}^i\gamma_5\psi^j$.
\item Axial current: $A_{\mu}^{ij}=\bar{\psi}^i\gamma_{\mu}\gamma_5\psi^j$.
\item Vector current: $V_{\mu}^{ij}=\bar{\psi}^i\gamma_{\mu}\psi^j$.
\end{itemize}
These bilinears are defined in the physical basis $\{\psi,\bar{\psi}\}$. By the change of variables
\begin{gather}
\psi\rightarrow e^{i\frac{\pi}{2}\gamma_5T/2}\psi, \quad \bar{\psi}\rightarrow\bar{\psi}e^{i\frac{\pi}{2}\gamma_5T/2},
\end{gather}
we define the twisted basis, with $T$ a diagonal matrix in flavor space. With this change of variables and at full twist
\begin{equation}
T=\textrm{diag}(+1,-1,-1,+1),\;N_f=2+1+1,
\end{equation}
the bilinears are rotated as 
\begin{align}
S^{ij}&\rightarrow S^{ij}, \\
P^{ij}&\rightarrow P^{ij}, \\
A_{\mu}^{ij}&\rightarrow iV_{\mu}^{ij}, \\
V_{\mu}^{ij}&\rightarrow -iA_{\mu}^{ij},
\end{align}
for $(i,j)=(u,d),(u,s),(c,d),(c,s)$, and
\begin{align}
S^{ij}&\rightarrow -iP^{ij}, \\
P^{ij}&\rightarrow iS^{ij}, \\
A_{\mu}^{ij}&\rightarrow A_{\mu}^{ij}, \\
V_{\mu}^{ij}&\rightarrow V_{\mu}^{ij},
\end{align}
for $(i,j)=(u,u),(u,c),(d,d),(d,s),(s,s),(c,c)$.

%%%%%%%%%%%%%%%%%%%%%%%%%%%%%%%%%%%%%%%%%%%%%%%%%%%%%%%%%%%
%%%%%%%%%%%%%%%%%%%%%%%%%%%%%%%%%%%%%%%%%%%%%%%%%%%%%%%%%%%
%%%%%%%%%%%%%%%%%%%%%%%%%%%%%%%%%%%%%%%%%%%%%%%%%%%%%%%%%%%
%%%%%%%%%%%%%%%%%%%%%%%%%%%%%%%%%%%%%%%%%%%%%%%%%%%%%%%%%%%

