\chapter*{Introduction}\addcontentsline{toc}{chapter}{Introduction}

%%%%%%%%%%%%%%%%%%%%%%%%%%%%%%%%%%%%%%%%%%%%%%%%%%%%%%%%%%%
%%%%%%%%%%%%%%%%%%%%%%%%%%%%%%%%%%%%%%%%%%%%%%%%%%%%%%%%%%%
%%%%%%%%%%%%%%%%%%%%%%%%%%%%%%%%%%%%%%%%%%%%%%%%%%%%%%%%%%%
%%%%%%%%%%%%%%%%%%%%%%%%%%%%%%%%%%%%%%%%%%%%%%%%%%%%%%%%%%%

\label{ch_introduction}

%%%%%%%%%%%%%%%%%%%%%%%%%%%%%%%%%%%%%%%%%%%%%%%%%%%%%%%%%%%
%%%%%%%%%%%%%%%%%%%%%%%%%%%%%%%%%%%%%%%%%%%%%%%%%%%%%%%%%%%
%%%%%%%%%%%%%%%%%%%%%%%%%%%%%%%%%%%%%%%%%%%%%%%%%%%%%%%%%%%
%%%%%%%%%%%%%%%%%%%%%%%%%%%%%%%%%%%%%%%%%%%%%%%%%%%%%%%%%%%

The Standard Model (SM) of particle physics is the theory that describes and unifies three of the four fundamental interactions in Nature: electromagnetism, the weak interaction, and the strong interaction. The theoretical framework in which the SM is formulated is that of Quantum Field Theory (QFT), and the particular theory of the strong interaction is Quantum Chromodynamics or QCD.

\section{Fields, Special Relativity and Quanta}

Quantum Field Theory is the framework which unifies quantum mechanics and special relativity. It is based on the concept of a field, which is an object that fills all of space and time. Mathematically, the field is a function that takes values in each point of space-time. The concept of fields was made popular in physics in the mid-nineteen century with the development of electromagnetism. It contrasts the newtonian view of forces as depending on the position of particles at a given time. This would imply instantaneous transmission of forces from one point of space to another, which is at odds with the relativistic principle that nothing can travel faster than light. In order to reconcile this with special relativity, one should make the force on one particle at a given time depend on the position of other particles at previous times in a complicated way. This task can be significantly simplified by the introduction of fields filling all space and time. Two such examples of fields are the electric $\vec{E}(\vec{x},t)$ and magnetic fields $\vec{B}(\vec{x},t)$, which are described classically by Maxwell's equations. These are Lorentz invariant and thus compatible with special relativiy.

On the other hand, the basic idea of quantum mechanics is that particles are described by wave equations that represent the density probability of finding said particle at a given position in space. In quantum mechanics, position and momentum are promoted to conjugate operators rather than being degrees of freedom. These operators do not commute and as a result Heisenberg's uncertainty principle is obtained
\begin{equation}
\Delta x\Delta p\geq\hbar.
\end{equation}
The wave equation of a particle is described by Schrödinger's equation, which is not Lorentz invariant, and thus is only usefull to describe non-relativistic quantum systems. In order to bridge the gap between quantum mechanics and special relativity, one can write a version of Schrödinger's equation which respects Lorentz symmetry. Two such versions are the Klein-Fock-Gordon equation
\begin{equation}
\left(\partial^2+m^2\right)\phi=0,
\end{equation}
which describes spin-$0$ particles (scalars), and the Dirac equation
\begin{equation}
\label{ch_intro:eq:Dirac}
\left(i\gamma_{\mu}\partial_{\mu}-m\right)\psi=0,
\end{equation}
which describes spin-$\frac{1}{2}$ particles (e.g. an electron). These equations are Lorentz invariant and thus describe a relativistic quantum system. The wave functions $\phi(x),\psi(x)$ are the fields of the particles described, which take values in all of space-time. Thinking of $\psi(x)$ as the field of the electron, in analogy to $\vec{E}$ being the elecric field, makes manifest that what is truly fundamental is not the electron as a particle itself, but its field. This solves another puzzle of Nature: how can two electrons separated a space-like distance (causally disconected) look exactly the same yet be different particles? Thinking of the field of the electron helps answering this, since all electrons are simply manifestations of a single unique field filling all of space-time. Once the transition to Quantum Field Theory is done, particles appear as excitations of the underlying field.

In order to make the transition to Quantum Field Theory, one needs to promote the classical fields $\psi(x),\psi(x)$ and whatever other fields in the theory to quantum operators, just like position and momentum were in quantum mechanics. This way, a quantum field is an operator valued function of space-time. Classically, eq.~(\ref{ch_intro:eq:Dirac}) has as solution
\begin{equation}
\label{ch_intro:eq:psi}
\psi_s(x)=\frac{1}{(2\pi)^3}\int d^3p\left(u_s(p)e^{-ipx}+v_s(p)e^{ipx}\right),
\end{equation}
where $s$ is the spin state of the particle. In QFT, we promote the amplitudes $u_s(p),v_s(p)$ to quantum operators. Furthermore, the second term in eq.~(\ref{ch_intro:eq:psi}) appears to have negative energy. In QFT one identifies $v_s(p)$ with the operator that creates an antiparticle with momentum $p$ and spin $s$, with positive energy. These have opposite electric charge to their corresponding particles, and are equivalent to positive energy particles propagating backwards in time, solving the problem of the negative energy modes in the Dirac equation.

Another advantage of QFT is that it provides with a framework in which a quantum system may not have a conserved number of particles. In the context of relativistic systems this is of relevance, since collisions can create and destroy particles. Furthermore, according to Heisenberg's uncertainty principle, if we place a particle in a box of size $L$ we have and uncertainty in its momentum of
\begin{equation}
\Delta p\geq\hbar/L.
\end{equation}
This is turn means that there's an uncertainty in the energy of the particle of order $\Delta E\geq\hbar c/L$. However, when the energy exceeds $2mc^2$ we have enough energy to create out of the vacuum a particle-antiparticle pair. This will happen at distances of order 
\begin{equation}
L=\lambda=\frac{\hbar}{mc},
\end{equation}
which is the Compton wavelength. At this and smaller distances (or equivalently higher energies) one expects to detect particle-antiparticle pairs swarming around the original particle, breaking down the very concept of a point-like particle. In QFT, writing interaction terms of the type
\begin{equation}
\psi(x)\phi(x)...
\end{equation}
for some generic fields $\psi,\phi,...$ allows for interactions in which particles and antiparticles may be destroyed or created, which can be seen easily by expanding the fields into creation and annihilation operators multiplying modes as in eq.~(\ref{ch_intro:eq:psi}).

\section{Global and Gauge Symmetries}

A key ingredient to QFTs are symmetries. These are described in the mathematical language of groups, which are a collection of elements with a group operation that is associative, has an identity element and for each element there exists an inverse element. Symmetries are said to be global when the group elements do not depend on space-time. These set of symmetries are of utmost importance in physics, since they provide with conservation laws through Noether's theorem. This way, by writing some theory which is invariant under Lorentz transformations one gets conservation of energy and momentum. Another example is the symmetry of the U(1) group, which provides with conservation of electric charge in the context of electromagnetism. The relevant groups in QFT are Lie groups, which are both groups and differentiable manifolds. In particular, the SM is built out of Yang-Mills theories coupled to matter. These are gauge theories whose Lie group is the special unitarity group SU(N).

A generalization of global symmetries consists in letting the group elements describing them to depend on the space-time coordinates. This way one extends the symmetry of the theory to be local, what is referred to as gauge symmetry. Gauge symmetry can be seen as a redundancy of the theory, such that performing some local rotation to the fundamental fields leaves physics unchanged. Though it is rather strange to write our theories of Nature in a redundant way (gauge theories), it turns out it is very useful, since these redundancies allow us to write simple Lagrangians which may have unphysical degrees of freedom, which we can eliminate by use of gauge redundancy. Such is the case of the polarization of the photon. These particles are described by vector fields with four components (four degrees of freedom)
\begin{gather}
A_{\mu}(x), \quad \mu=1,2,3,4,
\end{gather}
though the photon only has two physical polarizations (two degrees of freedom). Invoking gauge symmetry one can eliminate the two unphysical degrees of freedom, while writing a Lagrangian describing only the two physical polarizations of the photon would be extremely difficult. 

Another beautiful property of gauge symmetries is that they allow for a geometical interpretation of interactions. For example, the interaction of a photon and a electron can be understood in terms of the parallel transport of the latter along the manifold of its gauge group. The photon (gauge boson) appears as the connection between two points in the manifold of the U(1) Lie group, and the field strength tensor as the curvature. This way, all fundamental interactions of Nature can be understood in the light of geometry, much like gravity is in General Relativity.

The symmetry group of the SM is
\begin{equation}
SU(3)_{\textrm{c}}\times SU(2)_{\textrm{w}}\times U(1)_{\textrm{Y}},
\end{equation}
with $SU(3)_{\textrm{c}}$ the gauge group of the strong interaction (whose charge is called color), $SU(2)_{\textrm{w}}$ the gauge group of the weak interaction (whose charge is weak-isospin) and $U(1)_{\textrm{Y}}$ that of hypercharge. 

There are also three important discrete symmetries: charge conjugation (C), parity (P) and time reversal (T). The gauge part of the SM conserves the three of them, while the matter content violates C and P, while preserving T. There's a theorem that states that CPT is conserved in the full SM Lagrangian. Antiparticles are understood as the CPT conjugate of particles.

\section{Fundamental particles and the Standard Model}

In order to undertsand the way the fundamental particles are structured in the SM we need to remember that fermions (half-integer spin particles) can be right- or left-handed. In the massless limit this corresponds to their spin aligning or not with their momentum respectively. 

In the SM, there are 6 species or flavors of left-handed leptons (electron and electronic neutrino, muon and muonic neutrino, tau and tauonic neutrino), which are fermionic particles not charged under $SU(3)_{\textrm{c}}$,
\begin{equation}
\begin{pmatrix}
e\\ 
\nu_e
\end{pmatrix}_L,
\begin{pmatrix}
\mu\\ 
\nu_{\mu}
\end{pmatrix}_L,
\begin{pmatrix}
\tau\\ 
\nu_{\tau}
\end{pmatrix}_L,
\end{equation}
and 6 flavors of left-handed quarks (up, down, strange, charm , top and bottom), 
\begin{equation}
\begin{pmatrix}
u\\ 
d
\end{pmatrix}_L,
\begin{pmatrix}
s\\ 
c
\end{pmatrix}_L,
\begin{pmatrix}
t\\ 
b
\end{pmatrix}_L.
\end{equation}
which are fermions charged under $SU(3)_{\textrm{c}}$. In turn, these 6 flavors both of (left-handed) leptons and of quarks are organized into 3 different generations, each of them a doublet under $SU(2)_{\textrm{w}}$, as can be seen in the representation above.

On the other hand, there are the same 6 flavors of right-handed quarks, but only 3 flavors of right-handed leptons (electron, muon and tau), and they are singlets under $SU(2)_{\textrm{w}}$, that is they are not charged under the weak interaction. Finally, there is a scalar boson which is a doublet under $SU(2)_{\textrm{w}}$ called the Higgs doublet $\Phi$, responsible for the Spontaneous Symmetry Breaking (SSB) of
\begin{equation}
SU(2)_{\textrm{w}}\times U(1)_{\textrm{Y}}\to U(1)_{\textrm{em}},
\end{equation}
which is responsible for the generation of mass for the fundamental particles through a non-vanishing vacuum expectation value (vev) for the Higgs field. All these particles are charged under $U(1)_{\textrm{Y}}$ too. Finally, we have the gauge bosons mediating the fundamental interactions: the photon, mediating the electromagnetic interaction, the gluon, mediating the strong interaction, and the Z and W bosons, mediating the weak interaction.

The complete Lagrangian of the SM can be written in compact form as
\begin{align}
\mathcal{L}&=-\frac{1}{4}F_{\mu\nu}F^{\mu\nu}\\
&+i\bar{\psi}\gamma_{\mu}D_{\mu}\psi+\textrm{h.c.}\\
&+\psi_iy_{ij}\psi_j\Phi+\textrm{h.c.}\\
&+|D_{\mu}\Phi|^2-V(\Phi).
\end{align}
The first term contains the kinetic energy of the gauge bosons, while the second that of matter fields (leptons and quarks). The third term are the Yukawa couplings, responsible for the mass of elementary particles once the Higgs field $\Phi$ gets a non-vanishing vev. These are responsible for the mixing of flavor through the CKM matrix. Finally, the last term is the Lagrangian for the scalar Higgs. The pure gauge interactions depend only on the three coupling constants, which are the only free parameters of the SM, the matter fields not introducing any other free parameter. The Higgs???

Along the decades the SM has proved to be extremely successful in passing experimental tests. In 1973, neutral weak currents caused by the Z boson were discovered at CERN, causing Glashow, Salam and Weinberg to win the Nobel Prize in 1979 for formulating the electroweak theory based on the gauge symmetry $SU(2)_{\textrm{w}}\times U(1)_{\textrm{Y}}$ years before. The Z and W bosons themselves were discovered experimentally in 1983. Furthermore, the ratio of their masses was found to be consistent with the SM prediction. In 1977 the bottom quark was experimentally discovered by the Fermilab E288 experiment and the top quark in 1995 by CDF and DØ at Fermilab. This granted Kobayashi and Maskawa the Nobel Prize in 2008 for introducing these quarks into the theory in order to explain CP violation in kaon decay. Finally, the only other not observed particle of the SM, the Higgs boson, was discovered in 2012 at CERN, granting Englert and Higgs the Nobel Prize in 2013.

Despite the astonishing success of the SM, we now it cannot be the whole story. To begin with, it does not explain one of the four fundamental interactions of Nature, gravity. On the other hand, there's no candidate particle in the SM for dark matter, which we know accounts for about 85\% of the matter in the Universe. In addition to this, there are other theoretical puzzles, like the hierarchy problem of the Higgs mass, triviality of the Higgs coupling, the flavor puzzle or the strong CP problem, to quote a few. All this points to the fact that the SM is an effective theory that describes extremely well the Universe at the energy scales probed by modern day colliders, but that there must be some new physics lurking at high energy. Search for New Physics (NP) is the holy grail of modern day particle physics. 

One way to search for NP is to perform precission tests of the SM. This involves making high precission theoretical predictions and comparing them to high precission experiments. This allows to disentangle subtle NP effects that may affect processes accesible to nowadays colliders. In this respect, low-energy QCD is a rich arena to look for NP effects. One good place where to look is at Flavor Changing Neutral Currents (FCNC). These are heavily suppressed by the GIM mechanism in the SM, and thus are expected to be more sensitive to NP effects. On the other hand, in recent years some anomalies have been observed in processes involving B meson decays (...). In all these processes, QCD plays a crucial role, and thus precise theoretical predictions in this sector of the SM are of utmost importance in order to be able to discern possible signals of NP in experiments. 

\section{Why Lattice Field Theory?}

One difficulty arising in QFT is that when computing physical observables, like a scattering amplitude, divergences appear. These are removed from physical observables by the renormalization program. The latter consists in subtracting the divergent pieces appearing in physical observables by redefining the quantities of the theory which are not observables. For example, one can redefine field normalizations, masses and couplings appearing in the Lagrangian in such a way that though these quantities may diverge, a physical observable like a transition amplitude remains finite. This renormalization program has been applied successfully to the three fundamental interactions that the SM describes.

Renormalization introduces the concept of running of coupling constants. These are taken to run with the energy in such a way that physical observables remain finite and independent of the energy scale at which infinites are subtracted in the renormalization program.  In the case of electromagnetism, the coupling (the electron electric charge) decreases at low energies. However, in the case of Yang-Mills theories the opossite is true, and the coupling becomes stronger at lower energies. QCD is a Yang-Mills theory coupled to matter. Indeed, at low-energies, the strong coupling grows logarithmically. 

The technology usually employed to study Quantum Field Theory (QFT) is perturbation theory. It consists in expanding expectation values in the coupling of the theory, which must be smaller than 1 in order for the series to be convergent. Perturbation theory can thus be applied for the study of Quantum Electrodynamics at low energies, but not to the study of QCD at these energies. The only other known first-principles method to study QFTs is Lattice Field Theory. It consists in discreticing space-time into a grid or lattice, with space-time points separated by a non-zero lattice spacing $a$, and Wick rotating to the Euclidean. This allows to treat the theory as a statistichal physics system, computing integrals and expectation values numerically. This aproach allows to make reliable predictions of non-perturbative phenomena present in the SM such as low-energy QCD, and thus is of utmost relevance for precission tests of the SM and search of NP. 

Non-perturbative treatment of QCD is not only needed for the search of NP. It is also crucial to understand QCD and thus the SM itself. Most of the interesting phenomena of QCD appear due to its being strongly coupled at low energies. This is responsible for the phenomena of confinement, by which no color charged particles are observed in Nature at low energies. Spontaneous chiral symmetry breaking is also caused by non-perturbative effects, and is responsible for the light mass of pions. It is also expected that a mass gap is dynamically generated by the theory due to its non-perturbative character. This means that the spectrum of QCD (or any other Yang-Mills theory) contains no arbitrarily light particles. This is confirmed experimentally but there is no definite theoretical proof that QCD predicts such a mass gap, though numerical simulatiosn using Lattice Field Theory indicate the existence of said mass gap. A definite theoretical proof of it is one of the famous Millenium Prize Problems. Another important aspect of QCD is to understand its vacuum and the role of the $\theta$-term and topology of the gauge group. Again, for this is crucial a non-perturbative treatment of the theory.

Non-perturvative treatment of QFT is also crucial for other theoretical reasons beyond QCD. In many popular BSM theories, non-perturvative effects play a central role: in SUSY, non-perturvative effects are invoked to break SUSY at low energies. Also, nearly conformal field theories and technicolor models (which are up-scale version of QCD) require a non-perturbative treatment. More importantly, the SM version of the Higgs potential suffers from the triviality problem. This means that the renormalized Higgs coupling vanishes after perturbative renormalization except if there is a finite energy cutoff in the theory, meaning that the SM is nothing but an EFT valid up to some energy cutoff. If this is the case, it is expected that the Higgs mass receives large contributions from the high-energy scales, making it naturally heavy, as oppossed to what is measured at CERN. This is referred to as the hierarchy problem. Non-perturbative numerical approaches show triviality of any scalar quartic theory (which is the case of the Higgs potential in the SM). However, coupling this scalar to other particle content as in the SM could change the triviality behaviour of the coupling. Again, these issues can only be tackled with using non-perturbative approaches.

\section{A Mixed Action Lattice approach to light, strange and charm physics}

We have already motivated the need for precission calculations of SM physics involving the strong interaction in order to constrain the search for NP in experiments. In this thesis we are interested in the definition and setting of a mixed action approach for the study of charm physics with Lattice QCD. This mixed action uses the Wilson fermion regularization for quarks in the sea, with up/down and strange quarks, and the Wilson twisted mass regularization for quarks in the valence, with up/down, strange and charm quarks. When tuning the twisted mass quarks at maximal twist, systematic effects of order $\mathcal{O}(a)$ coming from the discretization of space-time are expected to cancel, improving the scaling of physical observables towards the continuum. This is of relevance for the study of charm physics, since the discretization effects associated to the charm quark $m_c$ are of order $\mathcal{O}(am_c)$ and large due to the heavy mass of the charm. Thus our mixed action is expected to substantially help in the extraction of precision charm observables with a controlled continuum limit.

The use of this mixed action involves matching the sea and valence sectors of the theory. This is so because we use different lattice regularizations for both. In order to ensure unitarity of the theory in the continuum we must match the physical quark masses in both sectors. Since the sea contains only light (up and down, which are treated as mass degenerate) and strange quarks, we need to tune the parameters of the mixed action in order to impose that the valence up/down and strange physical quark masses are the same as the ones in the sea. This involves precise calculations in the light and strange sectors of QCD, which is what we focus on in this thesis.

Furthermore, in Lattice Field Theory any physical quantity is computed in units of the lattice spacing $a$. Thus in order to make predictions, one needs to find the value of $a$ in physical units to convert any prediciton in the lattice to physical units. This task is called scale setting, and is one of the main focus of this thesis. As computations in Lattice Field Theory have become more nd more precise in the recent years, setting the scale with a high accuracy has become one of the main tasks for the community. This is so because its determination affects any prediction of the theory, and in particular its uncertainty affects any other quantity we want to express in physical units.

In addition to setting the scale in our mixed action, we will determine the physical value of the up/down and strange quark masses, which can only be determined from the lattice approach (...)

The thesis is structured as follows. In Chapter \ref{ch_foundation} we introduce the QCD action in the continuum and its gauge structure. Then we review how it can be formulated in a lattice with finite lattice spacing $a$. We explain how expectation values are computed numerically bridging the gap between the path integral formalism and statistical mechanics. We explain how the continuum limit is taken and its relation to renormalizability in the continuum. We review the Symanzik improvement program thought to reduce the discretization systematic effects and help in the task of taking the continuum limit. We also explain the program of setting the scale. In Chapter \ref{ch_observables} we define all the relevant physical observables that we will need in this thesis and how they are extracted in the lattice. These are meson masses, decay constants, PCAC quark masses and the Wilson gradient flow scale $t_0$. We also explain how do we extract the ground state signals of these observables, isolating them from excited states, using model variation techniques. In Chapter \ref{ch_ma} we introduce our mixed action regularization. We explain the differences between the sea and valence sectors, and perform the matching between both to impose equality of the physical quark masses in both sectors. We simultaneously tune the valence to maximal twist in order to obtain $\mathcal{O}(a)$ improvement. We also introduce the line of constant physics followed by the ensembles under study and the mass shifting procedure needed to correct for small mistunings along it. In Chapter \ref{ch_ss} we perform the scale setting of our mixed action by computing the gradient flow scale $t_0$ in physical units, using as external physical input the decay constants of the pion and kaon. We introduce a set of models we explore to take the chiral extrapolation to the physical pion mass and the continuum limit to $a\to0$. We use model averaging techniques to compute a final average result of $t_0$ in physical units accounting for the systematic uncertainty coming from the model variation. Treating $t_0$ as an intermediate scale allows to extract the lattice spacing in fm. Finally we present our conclusions in Chapter \ref{ch_conclusion}. 

This thesis is supplemented with a set of appendices. In Appendix \ref{apex_conventions} we present some conventions regarding the Gamma matrices, quark bilinears and the twisted and physical basis used in the different lattice regularizations. In Appendix \ref{apex_simulations} we review some useful simulation details of Lattice Field Theories. In Appendix \ref{apex_solvers} we briefly discuss how the Dirac operator needed to compute n-point functions is inverted in the lattice. We give details on the error propagation and treatment of (auto)correlations in Appendix \ref{apex_errors}. In Appendix \ref{apex_CLS} we review the gauge ensembles used in this work. In Appendix \ref{apex_fit} we give details on the fitting strategy we follow throughtout this work. In Appendix \ref{apex_fve} we give expressions for correcting finite volume effects as given by Chiral Perturbation Theory. Finally, in Appendix \ref{apex_ma} we sum up all the results for $t_0$ in physical units in the continuum and physical pion mass for each model explored for the chiral-continuum extrapolation.

