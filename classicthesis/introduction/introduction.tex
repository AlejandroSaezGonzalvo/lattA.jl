\chapter*{Introduction}\addcontentsline{toc}{chapter}{Introduction}

%%%%%%%%%%%%%%%%%%%%%%%%%%%%%%%%%%%%%%%%%%%%%%%%%%%%%%%%%%%
%%%%%%%%%%%%%%%%%%%%%%%%%%%%%%%%%%%%%%%%%%%%%%%%%%%%%%%%%%%
%%%%%%%%%%%%%%%%%%%%%%%%%%%%%%%%%%%%%%%%%%%%%%%%%%%%%%%%%%%
%%%%%%%%%%%%%%%%%%%%%%%%%%%%%%%%%%%%%%%%%%%%%%%%%%%%%%%%%%%

\label{ch_introduction}

%%%%%%%%%%%%%%%%%%%%%%%%%%%%%%%%%%%%%%%%%%%%%%%%%%%%%%%%%%%
%%%%%%%%%%%%%%%%%%%%%%%%%%%%%%%%%%%%%%%%%%%%%%%%%%%%%%%%%%%
%%%%%%%%%%%%%%%%%%%%%%%%%%%%%%%%%%%%%%%%%%%%%%%%%%%%%%%%%%%
%%%%%%%%%%%%%%%%%%%%%%%%%%%%%%%%%%%%%%%%%%%%%%%%%%%%%%%%%%%

The Standard Model (SM) of particle physics is the theory that describes and unifies three of the four fundamental interactions in Nature: electromagnetism, the weak nuclear force or interaction, and the strong force, also named Quantum Chromodyamics or QCD. The particle content of the SM is structured in three generations of ... In turn, these particles can be left- or right-handed (...) In Nature fermions of both chiralities exist, except for right-handed neutrinos which have never been observed. One possible reason for this is that neutrinos are only affected by the weak and gravitational interactions. However, their mass being extremely small (...) we can only detect them through the former (...) But the weak interaction only interacts with left-handed fermions (...). These particles complete the matter content of the theory. On the other hand, the three fundamental interactions are mediated by gauge bosons. The weak interaction is mediated by the massive W and Z bosons, electromagnetism by massless photons, and the strong interaction by massless gluons.  

The theory of the Standard Model is founded on the framework of Quantum Field theory. (Special relativity + QM... is a description of particles and interactions founded on the concept of gauge symmetry (...) Renormalization (...))

The Standard Model was developed for decades, its definite formulation in the (...) During the decades it has proved extremely successful in passing experimental tests. (Discovery of the top and bottom quarks, W boson, Higgs boson after decades...)

Despite the astonishing success of the SM, we now it cannot be the whole story. To begin with, it does not explain one of the four fundamental interactions of Nature, gravity. On the other hand, there's no candidate particle in the SM for dark matter, which we know accounts for about 20\% of the matter in the Universe. In addition to this, there are other theoretical puzzles, like the hierarchy problem of the Higgs mass, triviality of the Higgs coupling, the flavor puzzle or the strong CP problem, to quote a few. All this points to the fact that the SM is an effective theory that describes extremely well the Universe at the energy scales probed by modern day colliders, but that there must be some new physics lurking at high energy. Search for New Physics (NP) is the holy grail of modern day particle physics. 

One way to search for NP is to perform precission tests of the SM. This involves making theoretical predictions to a high accuracy, and comparing them to high precission experiments. This allows to disentangle subtle NP effects that may affect processes accesible to nowadays colliders. In this respect, low-energy QCD is a rich arena to look for NP effects (B-anomalies, flavor changing neutral currents, B meson rare decays...).

The framework usually employed to study Quantum Field Theory (QFT) is perturbation theory. It consists in expanding expectation values in the coupling of the theory, which must be smaller than 1 in order for the series to be convergent. However, in QFT the couplings run with the energy. In the case of electromagnetism, the coupling (the electron electric charge) decreases with energy. However, there may be cases in which a coupling is small at high energies, but that grows and becomes greater than 1 at low energies. This is the case of Yang-Mills theories, which are QFT whose gauge symmetry group is SU(N) for some value of N. QCD is a Yang-Mills theory with $N=3$ coupled to matter. Indeed, at low-energies, the strong coupling grows logarithmically and perturbation theory no longer provides a reliable description of the dynamics. The only other known first-principles method to study QFTs is Lattice Field Theory. It consists in discreticing space-time into a grid or lattice and Wick rotating to the Euclidean. This allows to treat the theory as a statistichal physics system, computing integrals and expectation values numerically. This apporach allows to make reliable predictions of non-perturbative phenomena such as low-energy QCD, and thus is of utmost relevance for precission tests of the SM and search of NP. In particular, some NP related problems are expected to involve non-perturbative processes, and in this respect Lattice Field Theory provides an excellent tool for its understanding. 

In this thesis we are interested in the definition and setting of a mixed action approach for the study of charm physics with Lattice QCD. This mixed action uses (...) The motivation is that it is expected that this setup allows to properly control the systematic effects associated to the charm quark mass when regularizing QCD in a lattice of finite spacing $a$. This is of great importance for studying charm physics at the non-perturbative level. The setting of this mixed action involves precise calculations in the light (up and down) and strange quark sector, which is what we focus on in this work. In particular, one needs to tune the parameters of the mixed action to have the same physical quark masses in the sea and valence sectors of the theory. Since the sea we will use employ only two degenerate up/down and a strange flavor, a matching in the light/strange sector is needed. Furthermore, in Lattice Field Theory any physical quantity is computed in units of the lattice spacing $a$. Thus in order to make predictions, one needs to find the value of $a$ in physical units. This task is called scale setting, and is one of the main focus of this thesis. In addition to this, we will determine the physical value of the up/down and strange quark masses, which can only be determined from the lattice approach (...)

The thesis is structured as follows. In Chapter (...)

