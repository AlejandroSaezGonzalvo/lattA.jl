%*******************************************************
% Abstract
%*******************************************************
%\renewcommand{\abstractname}{Abstract}
\pdfbookmark[1]{Abstract}{Abstract}
% \addcontentsline{toc}{chapter}{\tocEntry{Abstract}}
\begingroup
\let\clearpage\relax
\let\cleardoublepage\relax
\let\cleardoublepage\relax

\chapter*{Abstract}

In contemporary particle physics, accurate results for Standard Model (SM) observables are needed to improve the determination of its fundamental parameters and to guide the search for the New Physics (NP) by precise comparisons of SM predictions with the corresponding  experimental results. The quark-flavor sector of the Standard Model constitutes a rich arena for such an endeavor. In a large class of processes, the non-perturbative dynamics associated with the strong interaction between quarks and gluons plays a fundamental role. These hadronic effects are governed by Quantum Chromodynamics (QCD), the gauge theory of the strong interaction within the SM framework. The proper control of these effects is one of the main research fronts in theoretical particle physics today. Lattice field theory provides a first-principles method for studying strongly coupled theories such as QCD.

In this work, we study a lattice QCD setup aimed at high-precision calculations of light- and charm-quark physics. We employ a mixed action approach, in which two different regularizations of the fermionic action are used for the sea and valence sectors. More specifically, the sea sector is based on $N_f = 2 + 1$ non-perturbatively $\mathcal{O}(a)$ improved Wilson fermions, while up/down, strange and charm quarks are considered in the valence sector using  Wilson twisted mass quarks at maximal twist. By also considering the case where $\mathcal{O}(a)$ improved Wilson fermions are used in the sea and valence sectors, we have carried out a universality test in the up/down and strange quarks sector. This provides strong evidence of proper control of the approach to the continuum limit in these lattice QCD formulations.

We will describe a scale setting procedure and its impact on charm-quark observables. The use of a mixed action requires an adjustment of the quark masses of the sea and valence sectors to preserve the unitarity of the continuum theory. The external input used in the scale setting procedure corresponds to the use of the pion and kaon masses and decay constants in the isospin symmetric limit of QCD. The gradient flow scale $t_0$ is used as an intermediate scale, whose physical value can  be determined as a result of the scale setting. We employ model variation techniques to evaluate all relevant systematic uncertainties. Finally, the results of the scale setting are applied to  charm-quark sector in which accurate determinations of the charm quark mass and of the decay constants of the $D$ and $D_s$ mesons are obtained. Our results are among the most precise in the community for Wilson-like lattice regularizations.

\vfill

\begin{otherlanguage}{spanish}
\pdfbookmark[1]{Resumen}{Resumen}
\chapter*{Resumen}

En la física de partículas actual, resultados precisos de observables del Modelo Estándar (SM) son necesarios para mejorar la determinación de los parámetros fundamentales del SM y guiar la búsqueda de la Nueva Física (NP) mediante comparaciones precisas de las predicciones del SM con los resultados experimentales correspondientes. El sector de sabores de quarks del Modelo Estándar constituye un rico escenario para tal esfuerzo. En una gran clase de procesos, la dinámica no-perturbativa asociada a la interacción fuerte entre quarks y gluones juega un papel fundamental. Estos efectos hadrónicos se rigen por la Cromodinámica Cuántica (QCD), la teoría gauge de la interacción fuerte en el marco del SM. El control adecuado de estos efectos es uno de los principales frentes de investigación en la física teórica de partículas actual. La teoría de campos en el retículo proporciona un método basado en primeros principios para estudiar teorías fuertemente acopladas como QCD.

En este trabajo, estudiamos un \textit{setup} de QCD en el retículo orientada a cálculos de alta precisión de la física de quarks ligeros y del \textit{charm}. Empleamos un enfoque de acción mixta, en el que se utilizan dos regularizaciones diferentes de la acción fermiónica para los sectores mar y valencia. Más concretamente, el sector mar se basa en $N_f = 2 + 1$ fermiones de Wilson no-perturbativamente $\mathcal{O}(a)$ \textit{improved}, mientras que en el sector de valencia se consideran los quarks \textit{up/down}, \textit{strange} y \textit{charm} utilizando una regularización de Wilson \textit{twisted mass} a máximo \textit{twist}. Al considerar también el caso en el que se utilizan fermiones de Wilson $\mathcal{O}(a)$ \textit{improved} en los sectores mar y valencia, hemos llevado a cabo una prueba de universalidad en el sector de los quarks \textit{up/down} y \textit{strange}. Esto proporciona una fuerte evidencia de un control adecuado de la aproximación al límite al continuo en estas formulaciones de QCD en el retículo.

Describiremos un procedimiento de ajuste de escala o \textit{scale setting} y su impacto en los observables que involucran al quark \textit{charm}. El uso de una acción mixta requiere un ajuste de las masas de quarks de los sectores mar y valencia para preservar la unitariedad de la teoría en el continuo. El \textit{input} externo utilizado en el procedimiento de \textit{scale setting} corresponde al uso de las masas y constantes de desintegración de piones y kaones en el límite simétrico de isospín de QCD. La escala $t_0$ se utiliza como escala intermedia, cuyo valor físico puede determinarse como resultado del \textit{scale setting}. Empleamos técnicas de variación sobre modelos para evaluar todas las incertidumbres sistemáticas relevantes. Por último, los resultados del \textit{scale setting} se aplican al sector del quark \textit{charm}, en el que se obtienen determinaciones precisas de la masa del quark \textit{charm} y de las constantes de desintegración de los mesones $D$ y $D_s$. Nuestros resultados se encuentran entre los más precisos de la comunidad para regularizaciones reticulares tipo Wilson.

\end{otherlanguage}

\endgroup

\vfill
